\documentclass[12pt,a4paper]{article}
\usepackage{amsmath,amsthm,amsfonts,amssymb,amscd}
\usepackage{times}              
% Use Times New Roman
\usepackage{graphicx}           
% Enhanced support for images
\usepackage{float}              
% Improved interface for floating objects
\usepackage{booktabs}           
% Publication quality tables
\usepackage{xcolor}             
% Driver-independent color extensions
\usepackage{geometry}           
% Customize document dimensions
\usepackage{fullpage}           
% all 4 margins to be either 1 inch or 1.5 cm
\usepackage{comment}            
% Commenting
\usepackage{mathtools}
\usepackage{minted}             
% Highlighted source code. Syntax highlighting
\usepackage{listings}           
% Typeset programs (programming code) within LaTeX
\usepackage{lastpage}           
% Reference last page for Page N of M type footers.
\usepackage{fancyhdr}           
% Control of page headers and footers
\usepackage{hyperref}           
% Cross-referencing 
\usepackage[small,bf]{caption}  
% Captions
\usepackage{multicol}
\usepackage{cancel}
\usepackage{tikz}               
% Creating graphic elements
\usepackage{circuitikz}         
% Creating circuits
\usepackage{verbatim}          
% Print exactly what you type in
\usepackage{cite}               
% Citation
\usepackage[us]{datetime} 
% Various time format
\usepackage{blindtext}
% Generate blind text
\usepackage[utf8]{inputenc}
\usepackage{array}
\usepackage{makecell}
\usepackage{tabularx}
\usepackage{titlesec}
\usepackage[T1]{fontenc}
\usepackage[utf8]{inputenc}
\usepackage{amsmath,amsthm,amsfonts,amssymb,amscd}
\usepackage[lining,semibold,scaled=1.05]{ebgaramond}     
%\usepackage[cmintegrals,cmbraces]{newtxmath}
% Use FUCKING GARAMOND
\usepackage{hyperref}
\hypersetup{
	colorlinks,
	linkcolor={red!50!black},
	citecolor={blue!50!black},
	urlcolor={blue!80!black}
}
\urlstyle{same}
\usepackage[ebgaramond]{newtxmath}
\usepackage{xparse}
\usepackage{minted}
\usepackage{graphicx}                    
% Publication quality tables          
% Driver-independent color extensions
\usepackage[margin=1in]{geometry}           
% Customize document dimensions        
% all 4 margins to be either 1 inch or 1.5 cm
\usepackage{mathtools}           

% Creating circuits        
% Print exactly what you type in
\usepackage[us]{datetime} 
% Various time format
\usepackage{blindtext}
\usepackage{bbding}
\usepackage[scr=euler,cal=dutchcal,frak=euler,bb=dsfontserif,bbsymbols]{mathalfa}
\usepackage{bbm}
\usepackage{bm}
\usepackage{multirow}
\usepackage{multicol}
\usepackage[x11names,table]{xcolor}
\usepackage[shortlabels]{enumitem}   
\usepackage{standalone}
\usepackage{microtype}
\usepackage{censor}
\usepackage[backend=biber,style=alphabetic]{biblatex}
\usepackage{courier}
\usepackage{movement-arrows}
\usepackage{algorithm}
\usepackage{algpseudocode}
\usepackage[makeroom]{cancel}
\usepackage{tcolorbox}
\tcbuselibrary{skins,raster,theorems,listings,minted}
\usepackage{mathtools}
\usepackage{soul}
\usepackage{multicol}
\usepackage{caption}
\usepackage{wrapfig}
\usepackage{authblk}
\usepackage{float}
\usepackage{fontawesome5}
\usepackage{tikz}
\usetikzlibrary{positioning}
\usetikzlibrary{shapes.geometric,arrows,patterns,arrows.meta, bending,calc,matrix, shadings,intersections,shadows,decorations,fit,backgrounds,circuits.ee.IEC,shapes.callouts}
\newcounter{dummy}



\definecolor{dkgreen}{rgb}{0,0.6,0}
\definecolor{dred}{rgb}{0.545,0,0}
\definecolor{dblue}{rgb}{0,0,0.545}
\definecolor{lgrey}{rgb}{0.9,0.9,0.9}
\definecolor{gray}{rgb}{0.4,0.4,0.4}
\definecolor{darkblue}{rgb}{0.0,0.0,0.6}
\DeclareMathOperator*{\argmax}{arg\,max}
\DeclareMathOperator*{\argmin}{arg\,min}


\renewcommand{\labelitemi}{$\textcolor{SteelBlue4}{\bullet}$}
\renewcommand{\labelitemii}{$\textcolor{blue}{\cdot}$}
\renewcommand{\labelitemiii}{$\textcolor{SkyBlue4}{\diamond}$}
\renewcommand{\labelitemiv}{$\textcolor{SkyBlue4}{\ast}$}

\newcommand{\enf}[1]{\textcolor{DodgerBlue3}{\textbf{#1}}} %enf sta per enfasi
\let\emph\enf
\newcommand{\sott}[1]{\setulcolor{SkyBlue3}\ul{#1}}
\let\underline\sott

\newcommand{\prob}{\mathbb{P}}
\newcommand\independent{\protect\mathpalette{\protect\independenT}{\perp}}
\newcommand{\ev}{\mathbb{E}}
\def\independenT#1#2{\mathrel{\rlap{$#1#2$}\mkern2mu{#1#2}}}
\newcommand{\Z}{\mathbb{Z}}
\newcommand{\R}{\mathbb{R}}
\newcommand{\N}{\mathbb{N}}
\newcommand{\T}{\mathbb{T}}
\newcommand{\Tbar}{\overline{\T}}
\newcommand{\A}{\mathscr{A}}
\newcommand{\Nstar}{\N^{*}}
\newcommand{\Rext}{\overline{\R}}
\newcommand{\io}{\text{ i.o.}}
\newcommand{\normale}{\mathcal{N}}
\newcommand{\equalexpl}[1]{%
	\underset{\substack{\uparrow\\\mathrlap{\text{\vspace{-3cm}\hspace{-1em}#1}}}}{=}}
\newcommand{\dif}{\mathop{}\!\mathrm{d}}
\newcommand{\convas}{\xrightarrow[]{\text{a.s.}}}
\newcommand{\convpr}{\xrightarrow[]{\pr}}
\newcommand{\convd}{\xrightarrow[]{\text{d}}}
\newcommand{\convlp}{\xrightarrow[]{\lp}}
\newcommand{\lone}{L^{1}}
\newcommand{\convw}{\xrightarrow[]{\mathrm{weak}}}
\newcommand{\as}{\text{ a.s.}}
\newcommand{\asstnr}{\sim N(0,1)}
%\def\checkmark{\tikz\fill[scale=0.4](0,.35) -- (.25,0) -- (1,.7) -- (.25,.15) -- cycle;} 
\newcommand{\dx}{\dif x}
\newcommand{\dy}{\dif y}
\newcommand{\dt}{\dif t}
\newcommand{\du}{\dif u}
\newcommand{\ds}{\dif s}
\newcommand{\dw}{\dif\omega}
\newcommand{\dz}{\dif z}
\newcommand{\dmu}{\dif\mu}
\newcommand{\dpr}{\dif\pr}
\newcommand{\E}{\mathscr{E}}
\newcommand{\B}{\mathscr{B}}
\newcommand{\F}{\mathscr{F}}
\newcommand{\G}{\mathscr{G}}
\newcommand{\HS}{\mathscr{H}}
\newcommand{\Zn}{\mathscr{Z}}
\newcommand{\D}{\mathscr{D}}
\newcommand{\xbar}{\overline{X}}
\newcommand{\rbar}{\overline{\R}}
\newcommand{\ybar}{\overline{Y}}
\newcommand{\xxbar}{\overline{\xbar}}
\newcommand{\xtilde}{\widetilde{X}}
\newcommand{\ifonly}{\underline{if and only if}}
\newcommand{\ang}[1]{\langle#1\rangle}

\newcommand{\ubracketthin}[1]{\underbracket[0.3pt]{#1}}

\newcommand*\circled[1]{\tikz[baseline=(char.base)]{
		\node[shape=circle,draw,
		shading=ball, ball color=SkyBlue1!70,
		,inner sep=1.5pt] (char) {\scriptsize\bfseries #1};}}
\newcommand{\circnum}{\protect\circled{\arabic*}}
\newcommand{\circlet}{\protect\circled{\alph*}}	
\newcommand{\bbg}[1]{%
	\ooalign{$#1$\cr\raisebox{-.2pt}{$#1$}\cr\raisebox{.2pt}{$#1$}\cr\textcolor{white}{$\mkern0.2mu#1$}}%
}
\newcommand{\leb}{\bbg{\lambda}}

%i comandi di MERDA di stefano
\newcommand{\ra}{\rightarrow}
\newcommand{\iy}{\infty}
\newcommand{\mE}{\mathbb{E}}
\newcommand{\mP}{\mathbb{P}}
\newcommand{\mB}{\mathcal{B}(\mathbb{R})}
\newcommand{\mL}{\mathbb{L}}
\newcommand{\ninfty}{n\to$\infty$}
\newcommand{\tinfty}{t\to$\infty$}
\newcommand{\xinfty}{x\to$\infty$}
\newcommand{\yinfty}{y\to$\infty$}
\newcommand{\mLL}{\mathbb{L}^2(0,T)}
\newcommand{\mR}{\mathbb{R}}
\newcommand{\mC}{\mathcal{C}}
\newcommand{\mRd}{\mathbb{R}^d}
\newcommand{\mN}{\mathbb{N}}
%\newcommand{\m1}{\mathbf{1}}
\newcommand{\mF}{\mathcal{F}}
\newcommand{\mM}{\mathcal{M}}
\newcommand{\mW}{\mathcal{W}}
\newcommand{\mV}{\mathcal{V}}
\newcommand{\mA}{\mathcal{A}}
\newcommand{\mez}{\frac{1}{2}}
\newcommand{\intT}{\int_{0}^{T}}
\newcommand{\nN}{n \in \mathbb{N}}
\newcommand{\mmN}{\mathcal{N}}
\newcommand{\BM}{(B_t)_{t\geq 0}}
\newcommand{\PX}{(X_t)_{t\geq 0}}
\newcommand{\var}{\mathbb{V}\!\mathrm{ar}\,}
\newcommand{\cov}{\mathbb{C}\mathrm{ov}\,}
\newcommand{\lp}{L^{p}}
\newcommand{\BMF}{\left((B_t)_{t\geq 0},\mathcal{F}_t \right)}
\newcommand{\mFt}{\mathcal{F}_t}
\newcommand{\every}{\forall\,}
\newcommand{\dyadic}{\mathcal{d}}
\newcommand{\norm}[1]{\left|\negthinspace\left|#1\right|\negthinspace\right|}
\newcommand{\pr}{\mathbb{P}}
\newcommand{\indi}{\mathbb{1}}
\newcommand{\trns}[1]{{#1}^{\scriptscriptstyle\mathsf{T}}}
\newcommand{\iid}{\stackrel{\mathrm{iid}}{\sim}}
\newcommand{\unmezz}{\frac{1}{2}}
\newcommand{\sa}{$\sigma$-algebra}	\newcommand{\rv}{random variable}
\newcommand{\prefacename}{Preface}
\newcommand{\cinlar}{Çinlar }
\newenvironment{preface}{
	\vspace*{\stretch{2}}
	{\noindent \bfseries \huge\color{SkyBlue4} \prefacename}
	\begin{center}
		% \phantomsection \addcontentsline{toc}{chapter}{\prefacename} % enable this if you want to put the preface in the table of contents
		\thispagestyle{plain}
	\end{center}%
}
{\vspace*{\stretch{5}}}


\newenvironment{closethedeal}{
	\vspace*{\stretch{2}}
	{\noindent \bfseries \huge\color{SkyBlue4} That's all, folks}
	\begin{center}
		% \phantomsection \addcontentsline{toc}{chapter}{\prefacename} % enable this if you want to put the preface in the table of contents
		\thispagestyle{plain}
	\end{center}%
}
{\vspace*{\stretch{5}}}
\pgfkeys{%
	/calloutquote/.cd,
	width/.code                   =  {\def\calloutquotewidth{#1}},
	position/.code                =  {\def\calloutquotepos{#1}}, 
	author/.code                  =  {\def\calloutquoteauthor{#1}},
	/calloutquote/.unknown/.code   =  {\let\searchname=\pgfkeyscurrentname
		\pgfkeysalso{\searchname/.try=#1,                                
			/tikz/\searchname/.retry=#1},\pgfkeysalso{\searchname/.try=#1,
			/pgf/\searchname/.retry=#1}}
}  


\newcommand\calloutquote[2][]{%
	\pgfkeys{/calloutquote/.cd,
		width               = 5cm,
		position            = {(0,-1)},
		author              = {}}
	\pgfqkeys{/calloutquote}{#1}                   
	\node [rectangle callout,draw,callout relative pointer={\calloutquotepos},text width=\calloutquotewidth,/calloutquote/.cd,
	#1] (tmpcall) at (0,0) {#2};
	\node at (tmpcall.pointer){\calloutquoteauthor};    
}  		


\newcommand\lemmethink[2][]{%
	\pgfkeys{/calloutquote/.cd,
		width               = 5cm,
		position            = {(0,-1)},
		author              = {}}
	\pgfqkeys{/calloutquote}{#1}                   
	\node [cloud callout,draw,callout relative pointer={\calloutquotepos},text width=\calloutquotewidth,/calloutquote/.cd,
	#1] (tmpcall) at (0,0) {#2};
	\node at (tmpcall.pointer){\calloutquoteauthor};    
}  											


\renewcommand\leq\leqslant
\renewcommand\geq\geqslant		
\setlength\parindent{0pt}

%%%%%%%%%%%%%%%%%%%%%%%%%%%%%%%%%%%%%%%%%%%%%%%%%%%%%%%%%%%%%%
%%%%%%%%%%%%%%%%%%%%%%%%%%%%%%%%%%%%%%%%%%%%%%%%%%%%%%%%%%%%%%
\usemintedstyle{monokai}
\definecolor{codeBg}{HTML}{282822}
\newenvironment{py}
{\VerbatimEnvironment
	\begin{minted}[
		bgcolor=codeBg,
		breaklines,
		fontsize=\footnotesize
		]
		{python}}
	{\end{minted}}

	%\newmintinline[bluecode]{c++}{\color{codeblue}}
	
\newcommand{\pyinl}[1]{\mintinline[bgcolor=black!40]{python}{#1}}
%%%%%%%%%%%%%%%%%%%%%%%%%%%%%%%%%%%%%%%%%%%%%%%%%%%%%%%%%%%%%%

%%%%%%%%%%%%%%%%%%%%%%%%%%%%%%%%%%%%%%%%%%%%%%%%%%%%%%%%%%%%%%%
\definecolor{UM_Brown}{HTML}{3D190D}
\definecolor{UM_DarkBlue}{HTML}{2264B0}
\definecolor{UM_LightBlue}{HTML}{1CA9E1}
\definecolor{UM_Orange}{HTML}{fEB415}
\addbibresource{biblio.bib}




\setminted{style=monokai}
\usepackage{pygmentex}


\begin{document}
	\textcolor{UM_Brown}{
		\begin{center}
			\textbf{\Large Simulations}\\
			\vspace{5pt}
			Homework 4 \\
			\vspace{5pt}
			\textbf{M.S. in Stochastics and Data Science}\\
			\vspace{20pt}
			\textit{Andrea Crusi, Lorenzo Sala} \\
			\vspace{5pt}
			\today
		\end{center}
		\vspace{10pt}
		\hrule
	}
	
	
	
	%%%%%%%%%%%%%%% NEW SECTION %%%%%%%%%%%%%%% 
	\section*{Exercise 1}
	\begin{enumerate}
		\item To obtain the explicit expression for the distribution we need to substitute $m=2$ in the probability distribution formula:
		\begin{equation*}
			p(n)=\begin{cases}
				p(0)\left(\dfrac{\lambda}{\mu}\right)^{n}\dfrac{1}{n!}&\text{for }0\leqslant n\leqslant 2\\
				p(0)\left(\dfrac{\lambda}{\mu}\right)^{n}\dfrac{1}{2!2^{n-2}}=	p(0)\left(\dfrac{\lambda}{\mu}\right)^{n}\dfrac{1}{2^{n-1}}&\text{for }n> 2.\\
			\end{cases}
		\end{equation*}
		\item We can use the fact that 
		\[\sum_{n=0}^Np(n)=1\]
		to find an analytical solution for $p(0)$. A simple way to do this is to check how $p(n)$ behaves for $N\to\infty$. 
		\begin{equation*}
			\sum^2_{n=0}p(0)\left(\frac{\lambda}{\mu}\right)^{n}\frac{1}{n!}+\sum_{n=3}^{\infty}p(0)\left(\frac{\lambda}{\mu}\right)^{n}\frac{1}{2!2^{n-2}}=1.
		\end{equation*}
		By explicitly computing the first sum and rearranging the second one so that it starts from $n=0$ we can get
		\begin{align*}
			&p(0)\left[1+\frac{\lambda}{\mu}+\frac{\lambda^{2}}{2\mu^{2}}+\sum_{n=3}^{\infty}\left(\frac{\lambda}{\mu}\right)^{n}\frac{1}{2^{n-1}}\right]=1\\
			\implies p(0)&=\frac{1}{1+\frac{\lambda}{\mu}+\frac{\lambda^{2}}{2\mu^{2}}+\sum_{n=3}^{\infty}\left(\frac{\lambda}{\mu}\right)^{n}\frac{1}{2^{n-1}}}=\\
			&=\frac{1}{1+\frac{\lambda}{\mu}+\frac{1}{2}\left(\frac{\lambda}{\mu}\right)^{2}+\sum_{n=3}^{\infty}\left(\frac{\lambda}{\mu}\right)^{n}\frac{1}{2^{n-1}}}=\\
			&=\frac{1}{1+\frac{\lambda}{\mu}\left[1+\frac{\frac{\lambda}{\mu}}{2}+\sum_{n=3}^{\infty}\left(\frac{\frac{\lambda}{\mu}}{2}\right)^{n-1}\right]}=\\
			&=\frac{1}{1+\frac{\lambda}{\mu}\left[\underbracket{\sum_{n=0}^{\infty}\left(\frac{\frac{\lambda}{\mu}}{2}\right)^{n-1}}_{\text{geometric series}}\right]}=\\
			&=\frac{1}{1+\frac{\lambda}{\mu}\left[\frac{1}{1-\frac{\frac{\lambda}{\mu}}{2}}\right]}=\\
			&=\frac{1}{1+\frac{\lambda}{\mu}\left[\frac{2}{2-\frac{\lambda}{\mu}}\right]}=\\
			&=\frac{2-\frac{\lambda}{\mu}}{2+\frac{\lambda}{\mu}}.
		\end{align*}
		So our new distribution becomes
		\begin{align*}
			p(n)&=\begin{cases}
				\frac{2-\frac{\lambda}{\mu}}{2+\frac{\lambda}{\mu}}\left(\dfrac{\lambda}{\mu}\right)^{n}\dfrac{1}{n!}&\text{for }0\leqslant n\leqslant 2\\
				\frac{2-\frac{\lambda}{\mu}}{2+\frac{\lambda}{\mu}}\left(\dfrac{\lambda}{\mu}\right)^{n}\dfrac{1}{2^{n-1}}&\text{for }n> 2\\
			\end{cases}\\
			&=\begin{cases}
				\frac{2-\frac{\lambda}{\mu}}{2+\frac{\lambda}{\mu}}\left(\dfrac{\lambda}{\mu}\right)^{n}\dfrac{1}{n!}&\text{for }0\leqslant n\leqslant 2\\
				\frac{2-\frac{\lambda}{\mu}}{2+\frac{\lambda}{\mu}}\left(\dfrac{\lambda}{\mu}\right)^{n}\dfrac{1}{2^{n-1}}&\text{for }n> 2\\
			\end{cases}
		\end{align*}
		which can also be expressed, by denoting $\frac{\lambda}{\mu}=\rho$ as the load of the server, as
		\begin{equation*}
			p(n)=\begin{cases}
				\frac{2-\rho}{2+\rho}\left(\rho\right)^{n}\dfrac{1}{n!}&\text{for }0\leqslant n\leqslant 2\\
				\frac{2-\rho}{2+\rho}\left(\rho\right)^{n}\dfrac{1}{2^{n-1}}&\text{for }n> 2.\\
			\end{cases}
		\end{equation*}
	\end{enumerate}
	\section*{Exercise 2}
	We know that when a network comprises load independent stations the recursive equation for utilization in a station $i$ is
	\begin{equation*}
		\overline{n_{i}}(n)=U_{i}[1+\overline{n_{i}}(n-1)].
	\end{equation*}
	We are interested in what happens when $n$ goes to $\infty$ so
	\begin{equation*}
		\overline{n_{i}}=U_i[1+\overline{n_{i}}]\implies\overline{n_{i}}-U_i\overline{n_{i}}=U_i\implies \overline{n_{i}}=\frac{U_i}{1-U_i}.
	\end{equation*}
	In bottleneck stations $b$, where the utilization $U_b$ tends to 1 due to the saturation of the system, the queue length will tend to infinity.
	Moreover, we know that, by consistency laws,
	\begin{equation*}
		\frac{U_b}{U_i}=\frac{V_bS_b}{V_iS_i}=\frac{D_{b}}{D_b}\implies U_i=\frac{D_iU_b}{D_b}
	\end{equation*}
	and therefore
	\begin{equation*}
		\overline{n_{i}}=\frac{\frac{D_iU_b}{D_b}}{\frac{D_b-D_iU_b}{D_b}}=\frac{D_iU_b}{D_b-D_iU_b}
	\end{equation*}
	but since as load increases $U_b\to1$ we get
	\begin{equation*}
		\overline{n_{i}}=\frac{D_i}{D_b-D_i}
	\end{equation*}
	for non-bottleneck stations.\\
	Let's turn to average waiting time. Little's Law tells us that 
	\begin{equation*}
		\overline{n_{i}}=\overline{w_i}X_i\implies\overline{w_i}=\frac{\overline{n_i}}{X_i}
	\end{equation*}
	but we know that as the load increases $\overline{n_{i}}\to U_i[1+\overline{n_{i}}]$ so 
	\begin{equation*}
		\overline{w_i}=\frac{U_i[1+\overline{n_{i}}]}{X_i}.
	\end{equation*}
	In bottleneck stations, as the queue goes to $\infty$, so does the average waiting time. Remember that for load independent stations we have $U_i=X_i(n)\cdot S_i$, so
	\begin{equation*}
		\overline{w_i}=\frac{\cancel{X_i(n)}S_i[1+\overline{n_{i}}]}{\cancel{X_i(n)}}=S_i[1+n_i].
	\end{equation*}
	Now let's take into account the throughput. As we said before, by Little's Formula we know that 
	\begin{equation*}
		\overline{n_{i}}=\overline{w_i}X_i\implies X_i(n)=\frac{\overline{n_i}}{\overline{w_i}}
	\end{equation*}
	and by substituting we get
	\begin{equation*}
		X_i(n)=\frac{\overline{n_{i}}}{S_i[1+n_i]}.
	\end{equation*}
	In bottleneck stations, as $\overline{n_{i}}$ becomes increasingly larger, $[1+\overline{n_i}]$ becomes closer and closer to $\overline{n_{i}}$ so that the throughput for bottleneck stations will tend to
	\begin{equation*}
		\frac{1}{S_i}.
	\end{equation*}
	\section*{Exercise 3}
	The normalization constant is defined in two ways:
	\begin{equation*}
		\underbracket[0.6pt]{g(n,m)=\sum_{\mathbf{n}\in S(n,m)}\prod_{i=1}^{m}f_i(n_i)}_{\text{definition}}\qquad\underbracket[0.6pt]{\sum_{k=0}^{n}f_{m}(k)g(n-k,m-1).}_{\text{manipulation}}
	\end{equation*}
	The service function $f_i(k)$ is defined as
	\begin{equation*}
		f_{i}(k)=\begin{cases}
			1&k=0\\
			V_iS_if_i(k-1) &k>0.
		\end{cases}
	\end{equation*}
	We know that the distribution of $p_i(k)$, which is the fraction of time the station $i$ spends with $k$ customers inside, can be computed as
	\begin{align*}
		p_i(k)&=\sum_{\mathclap{\stackrel{\mathbf{n}\in S(n,i)}{n_i=k}}}P(\mathbf{n})\\
		&=\sum_{\stackrel{\mathbf{n}\in S(n,i)}{n_i=k}}\frac{1}{g(n,i)}\prod_{j=1}^{M}f_j(n_j)\\
		&=\frac{1}{g(n,i)}\sum_{\stackrel{\mathbf{n}\in S(n,i)}{n_i=k}}\prod_{j=1}^{M}f_j(n_j)\\
		&=\frac{f_i(k)}{g(n,i)}\underbrace{\sum_{\mathbf{n}\in S(n-k,i-1)}\prod_{j=1}^{i-1}f_j(n_j)}_{g(n-k,i-1)}\qquad\text{\tiny since $n_i=k$ for all states we can take it out of the sum}\\
		&=\frac{f_i(k)g(n-k,i-1)}{g(n,i)}
	\end{align*}
	To get $p_i(k,N)$ we must sum the probabilities for station $i$ to have $k$ customers over all possible distribution in the other station with a maximum of $N$ customers. This happens over the reduced state space 
	\begin{equation*}
		S^{[-i]}(N,M)
	\end{equation*}
	(which is basically the whole state space of $N$ maximum clients and $M$ stations but with the number of customers in $i$-th station fixed to 0) and with the reduced normalization constant with the $i$-th station missing over the reduced state space
	\begin{equation*}
		g^{[-i]}(N,M)=\sum_{\mathbf{n}\in S^{[-i]}(N,M)}\prod_{{j\neq i}}f_j(n_j).
	\end{equation*}
	So, in a similar manner as before, we get
	\begin{align*}
		p_i(k,N)&=\sum_{\mathclap{\mathbf{n}\in S^{[-i]}(N,i)}}P(\mathbf{n})\\
		&=\frac{f_i(k)}{g(N,i)}\underbrace{\sum_{\mathbf{n}\in S^{[-i]}(N-k,i-1)}\prod_{j=1}^{i-1}f_j(n_j)}_{g^{[-i]}(N-k,i-1)}\\
		&=\frac{f_i(k)g^{[-i]}(N-k,i-1)}{g(N,i)}.
	\end{align*}
	The service function $f_i(k)$ is defined as
	\begin{equation*}
		f_{i}(k)=\begin{cases}
			1&k=0\\
			V_iS_if_i(k-1) &k>0.
		\end{cases}
	\end{equation*}
	This allows us to write
	\begin{align*}
		p_i(k,N) &= \frac{f_i(k)g^{[-i]}(N-k,i-1)}{g(N,i)} \\
		&= 	V_iS_if_i(k-1)\cdot\frac{g^{[-i]}(N-k,i-1)}{g(N,i)}\\
		&=V_iS_if_i(k-1)\cdot\frac{g^{[-i]}(N-k,i-1)}{g(N,i)}\cdot\color{red}\frac{g(N-1,i)}{g(N-1,i)}\\
		&=\underbrace{V_i\frac{g(N-1,i)}{g(N,i)}}_{X_i(N)}\cdot S_i\cdot\underbrace{\frac{ f_i(k-1)g^{[-i]}(\overbrace{(N-1)-(k-1)}^{N-k},i-1)}{g(N-1,i)}}_{p_i(k-1,N-1)}\\
		&=X_i(N)S_ip_i(k-1,N-1).
	\end{align*}
	\section*{Exercise 4}
	Sto coso è ancora senza input, però
	\begin{minted}[linenos,autogobble,breaklines,bgcolor=black!90!blue!80]{c}
	#include <stdio.h>
	
	// Constants
	#define M 4     // Number of servers (stations)
	#define N 80  // Maximum number of customers
	
	// Input parameters
	double Z = 10.0;  // Think time for delay station (there is only one)
	double S[M] = {0, 0.04, 0.06, 0.04};  // Service times
	char ST[M] = {'D','C', 'C', 'C',};  // Station types ('C' for computational, 'D' for delay)
	//delay station is reference
	// Visiting ratios (Vi)
	double V[M] = {1.0, 10, 5.5, 3.5};
	
	// Arrays for results
	double X[M][N+1], U[M][N+1], n[M][N+1], w[M][N+1];
	
	int main() {
		// Initialization
		for (int i = 0; i < M; i++) {
			n[i][0] = 0.0;  // put 0 as the value for all stations
		}
		
		// Compute performance measures for each population size k (from 1 to N)
		for (int k = 1; k <= N; k++) {
			
			// Compute the waiting time w_i[k] for each station i
			for (int i = 0; i < M; i++) {
				if (ST[i] == 'D') {
					w[i][k] = Z;  // Delay station
				} else {
					w[i][k] = S[i] * (1 + n[i][k - 1]);  // Queue station
				}
			}
			double sum = 0.0; // initialize sum
			
			// Compute the sum of Vi * wi[k] across all stations
			for (int i = 0; i < M; i++) {
				sum += V[i] * w[i][k];
			}
			
			// Compute throughput for reference job
			double Xref = k / sum;
			
			// Compute performance metrics for each station i
			for (int i = 0; i < M; i++) {
				X[i][k] = V[i] * Xref;  // Throughput for station i
				
				if (ST[i] == 'D') {
					// Delay station
					n[i][k] = Z * X[i][k];
					U[i][k] = n[i][k] / k;
				} else {
					// Computational station
					U[i][k] = S[i] * X[i][k];  // Utilization
					n[i][k] = U[i][k] * (1 + n[i][k - 1]);  // Number of customers
				}
			}
		}
		int n_customers[2] = {1,80};
		// Print results for N = 1 and N = 80;
		for (int j=0; j<=1; j++){
			printf("Simulation with %d customers\n\n", n_customers[j]);
			for (int i = 0; i < M; i++) {
				printf("Station %d results:\n", i);
				printf("Throughput (X[%d])\t\t= %f\n", n_customers[j], X[i][n_customers[j]]);
				printf("Utilization (U[%d])\t\t= %f\n", n_customers[j], U[i][n_customers[j]]);
				printf("Avg number of jobs (n[%d])\t= %f\n", n_customers[j], n[i][n_customers[j]]);
				printf("\n");
			}
			printf("-*-*-*-*-*-*-*-*-*-*-*-*-*-*-*-*-*-*-*-*-*-*-*\n\n");
		}
		
		getchar();
		return 0;
	}
	\end{minted}
\end{document}
