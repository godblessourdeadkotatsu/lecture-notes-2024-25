% !TeX spellcheck = en_US
\documentclass{article}
\usepackage{paccoan}

\title{This Should Help Your Lazy Ass In Analisys B}
\author{Charles A. Mongus, world famous jazz composer and bassist}
\date{\today}

\begin{document}

\maketitle
\tableofcontents
\section{Normed Spaces}
\begin{proposition}
    A subset $C\subset X$ is a closed subset of $(X,\norm{\cdot})$ whenever the limit of any convergent sequence ${(x_n)}_n\subset C$ belongs to $C$. 
\end{proposition}
In other words, $C$ is closed if and only if once a sequence of elements of $C$ is converging, the limit cannot escape $C$.
\begin{fancyproof}
    \begin{enumerate}[\circnum]
        \item \textbf{$C$ closed and ${(x_n)}_n\subset C$ $\implies$ $\lim_nx_n=x\in C$}.\\
        We argue by contradiction that $x\not\in C$: then $x$ must be in $X\setminus C$ which must be open. But if that is an open set then there must exist a ball of $x$ for some $r>0$ such that $B(x,r)\subset X\setminus C$. But due to the definition of convergence there must exist a $N$ such that $\norm{x_n-x}<r\;\every n\geq N$, but then that $x_n$ should be outside $C$ which is a contradiction!
        \item \textbf{$\lim_nx_n=x\in C$ $\implies$ $C$ closed  and ${(x_n)}_n\subset C$}. \\
        We argue by contradiction that $C$ is not closed, so $X\setminus C$ must be not open. Not open sets are such that there exist $x\in X\setminus C$ such that for every $r>0$ you can't find a ball that it is entirely in $X\setminus C$ which means $B(x,r)\cap C\neq\emptyset$. Then pick $x_n\in B(x,\frac{1}{n})\cap C$ so that every $x_n$ also belongs to $C$. Clearly this is a sequence of $n$ such that for any $n\geq1$ we have $\norm{x_n9-x}<\frac{1}{n}$ and this means that $\lim_nx_n=x$. We picked $x\in X\setminus B$ so we have found a limit of a sequence of elements of $C$ that doesn't belong to $C$ which is a contradiction!
    \end{enumerate}
\end{fancyproof}
\begin{lemma}
		\emph{Young's inequality.} Let $p>1$ and let $q<1$ be its conjugate exponent. Then, for any nonnegative $a,b\in\R$ it holds
		\begin{equation*}
			ab\leq\frac{1}{p}a^{p}+\frac{1}{b}b^q.
		\end{equation*}
\end{lemma}
\begin{theorem}
    \emph{Holder Inequality.} Let $1\leq p\leq\infty$ and $\frac{1}{p}+\frac{1}{q}=1$. Assume that $f\in \Lp(S,\mu)$ and $g\in\Lq(S,\mu)$. Then $f\cdot g\in\Lone$ and
    \begin{equation*}
        \norm{fg}_1\leq\norm{f}_p\norm{g}_q.
    \end{equation*}
\end{theorem}
\begin{fancyproof}
    The proof is trivial if $p=1$ and $q=1$. Remember Young's inequality 
    \[
    ab\leq\frac{1}{p}a^p+\frac{1}{q}b^q.
    \]
    Now let's say $a=|f(s)|$ and $b=|g(s)|$. Now our inequality becomes
    \[
    |f(s)g(s)|\leq\frac{1}{p}|f(s)|^p+\frac{1}{q}|g(s)|^q\qquad\maes{S}.
    \]
    Now we integrate over $S$ and we get
    \begin{align*}
        &\int_S|f(s)g(s)|\leq\frac{1}{p}\int_S|f(s)|^p+\frac{1}{q}\int_S|g(s)|^q\\
        \implies&\norm{fg}_1\leq\frac{1}{p}\norm{f}_p^p+\frac{1}{q}\norm{g}_q^q
    \end{align*}
    which means that $\norm{fg}_1$ is finite and therefore $fg\in\Lone(S,\mu)$. To end the proof let's substitute $f$ with $\lambda f,\;\every\lambda>0$. We get
    \begin{align*}
        &\lambda\norm{fg}_1\leq\frac{\lambda^p}{p}\norm{f}_p^p+\frac{1}{q}\norm{g}_q^q\qquad\every\lambda>0\\
        \implies&\frac{\lambda}{\color{red}\lambda}\norm{fg}_1\leq\frac{\lambda^p}{p\color{red}\lambda}\norm{f}_p^p+\frac{1}{q\color{red}\lambda}\norm{g}_q^q\qquad\every\lambda>0\\
        \implies&\norm{fg}_1\leq\frac{\lambda^{p-1}}{p}\norm{f}_p^p+\frac{1}{\lambda q}\norm{g}_q^q\qquad\every\lambda>0.\\
    \end{align*}
    Now choose $\lambda=\frac{1}{\norm{f}_p}\cdot\norm{g}^{\frac{q}{p}}_q$. When we substitute this value for \( \lambda \), we get:
    \[
\frac{\lambda^p}{p} \|f\|_p^p = \frac{\left( \frac{1}{\|f\|_p} \cdot \|g\|_q^{\frac{p}{q}} \right)^p}{p} \|f\|_p^p.
\]
Expanding \( \left( \frac{1}{\|f\|_p} \cdot \|g\|_q^{\frac{p}{q}} \right)^p \), we get:
\[
\left( \frac{1}{\|f\|_p} \cdot \|g\|_q^{\frac{p}{q}} \right)^p = \frac{\|g\|_q^p}{\|f\|_p^p}.
\]

Substituting this, we have:
\[
\frac{\lambda^p}{p} \|f\|_p^p = \frac{\frac{\|g\|_q^p}{\cancel{\|f\|_p^p}}}{p} \cancel{\|f\|_p^p} = \frac{\|g\|_q^p}{p}.
\]

Substitute back into the inequality:
\[
\lambda \|fg\|_1 \leq \frac{\|g\|_q^p}{p} + \frac{1}{q} \|g\|_q^q.
\]

Since \( \frac{\|g\|_q^p}{p} + \frac{\|g\|_q^q}{q} = \|g\|_q^p \left( \frac{1}{p} + \frac{1}{q} \right) \) and \( \frac{1}{p} + \frac{1}{q} = 1 \), we get:
\[
\lambda \|fg\|_1 \leq \|g\|_q^p.
\]


Dividing by \( \lambda = \frac{1}{\|f\|_p} \|g\|_q^{\frac{p}{q}} \):
\[
\|fg\|_1 \leq \|f\|_p \|g\|_q.
\]

This is Hölder’s inequality:
\[
\|fg\|_1 \leq \|f\|_p \|g\|_q.
\]

\end{fancyproof}
\begin{theorem}
	For any $1\leq p\leq\infty$, $\Lp(S,\mu)$ is a vector space and $\norm{\cdot}_{p}$ is a norm.
\end{theorem}
\begin{remark}
	For $1<p<\infty$ the triangular inequality
	\begin{equation*}
		\norm{f+g}_p\leq\norm{f}_p+\norm{g}_p\qquad\every f,g\in\Lp(S,\mu)
	\end{equation*}
	is known as \emph{Minkowski's Inequality}.
\end{remark}
\begin{fancyproof}
	We already know that if $f\in\Lp(S,\mu)$ then $\lambda f\in\Lp(S,\mu)$. Homogeneity and uniqueness are also existent for $\norm{\cdot}_p$ so in order to show that $\Lp(S,\mu)$ is a vector space we only need to prove that if $f,g\in\Lp(S,\mu)$ then $f+g\in\Lp(S,\mu)$ \underline{and} $\norm{\cdot}$ is a norm. \\
	Fix $f,g\in\Lp(S,\mu)$. We know that for any $x,y\in\R$ we get
	\begin{equation*}
		\left|\frac{1}{2}x+\frac{1}{2}y\right|^{p}\leq\frac{1}{2}|x|^p+\frac{1}{2}|y|^{p}
	\end{equation*}
	since this mapping $r\to r^p$ is convex. This also means that
	\begin{equation*}
		\left|x+y\right|^p\leq2^{p-1}\left(|x|^{p}+|y|^{p}\right).
	\end{equation*}
	and this implies in particular that
	\begin{equation*}
		\left|f(s)+g(s)\right|^{p}\leq2^{p-1}\left(|f(s)|^{p}+|g(s)|^{p}\right)\qquad\maes{s\in S}.
	\end{equation*}
	If we integrate over $S$ we get:
	\begin{equation*}
		\int_S\left|f(s)+g(s)\right|^{p}\leq2^{p-1}\left(\int_S|f(s)|^{p}+\int_S|g(s)|^{p}\right)
	\end{equation*}
	which means 
	\begin{equation*}
		\norm{f+g}_{p}^{p}\leq2^{p-1}\left(\norm{f}^p_p+\norm{g}^p_p\right)
	\end{equation*}
	which means that $f+g\in\Lp(S,\mu)$. \par
	We now must prove the Minkowski's inequality. We know that
	\begin{equation*}
			\norm{f+g}_{p}^{p}=\int_S\left|f+g\right|^{p}\dmu=\int_S\left|f+g\right|\left|f+g\right|^{p-1}\dmu
	\end{equation*}
	but since we know that $\left|f+g\right|\leq|f|+|g|$ then
	\begin{equation*}
			\norm{f+g}_{p}^{p}\leq \int_S|f||f+g|^{p-1}\dmu+\int_S|g||f+g|^{p-1}\dmu.
	\end{equation*}
	Call $\psi=|f+g|^{p-1}$. It clearly belongs to $\Lq(S,\mu)$ because
	\begin{equation*}
		|\psi|^{q}=\left(|f+g|^{p-1}\right)q=\left|f+g\right|^{p}\qquad\text{since }q(p-1)=p
	\end{equation*}
	so
	\begin{equation*}
		\norm{\psi}_{q}=\left(\int_S|\psi|^q\right)^{\frac{1}{q}}=\left(\int_S\left|f+g\right|^{p}\right)^{\frac{1}{q}}=\norm{f+g}_{p}^{\frac{p}{q}}<\infty
	\end{equation*}
	And this means that $|\psi|^{q}\in\Lone(S,\mu)\implies\psi\in\Lq(S,\mu)$. We also know that $|f|\in\Lp(S,\mu)$ so we can apply Holder's inequality with $f$ and $\psi$ so that
	\begin{equation*}
		\int_S|f||f+g|^{p-1}\dmu=\norm{f\psi}_1\leq\norm{f}_p\norm{\psi}_q=\norm{f}_p\norm{f+g}_p^{\frac{p}{q}}
	\end{equation*}
	and
	\begin{equation*}
		\int_S|g||f+g|^{p-1}\dmu\leq\norm{g}_p\norm{f+g}_p^{\frac{p}{q}}
	\end{equation*}
	So that
	\begin{equation*}
		\norm{f+g}_p^p\leq\norm{f}_p\norm{f+g}_p^{\frac{p}{q}}+\norm{g}_p\norm{f+g}_p^{\frac{p}{q}}.
	\end{equation*}
	Dividing by $\norm{f+g}_p^{\frac{p}{q}}\neq0$ (otherwise the proof is trivial) we get
	\begin{equation*}
		\norm{f+g}_p^{p-\frac{p}{q}}\leq\norm{f}_p+\norm{g}_p.
	\end{equation*}
\end{fancyproof}
\begin{proposition}
	Let $(\Omega,\F,\pr)$ be a probability triplet. then the following holds:
	\begin{equation*}
		L^{\infty}(\Omega,\F,\pr)\subset\Lp(\Omega,\F,\pr)\subset\Lq(\Omega,\F,\pr)\subset\Lone(\Omega,\F,\pr).
	\end{equation*}
\end{proposition}
Of course, this result remains valid for every other measure space $(S,\Es,\mu)$ as long as $\mu(S)<\infty$. In the special case in which 
\begin{equation*}
	S=\N\qquad\F=\mathcal{P}(\N)
\end{equation*}
and $\mu(A)$ is the counting measure $\mu(A)=\sum_{k\in A}\delta_k(A),A\in\N$ then knowing that sequences $n\mapsto f(n)$ can be identified as functions over $\N$ of the type $f:\N\to\R$
we see that 
\begin{equation*}
	\Lone(S,\mu)=\underbracket{\mathscr{L}^1(S,\mu)}_{\mathclap{\text{actual functions, not equivalence classes}}}=\ell^1(\N)=\left\{\mathbf{x}={(x_{n})}_{n};\norm{\mathbf{x}}_{1}:=\sum_{n=1}^{\infty}|x_n|<\infty\right\}.
\end{equation*} 
This means that $\ell^1(\N)$ is a $\Lone$ space for some special choice of $S$ and $\mu$. Since we chose our measure as the counting measure, we get
\begin{equation*}
	\int_\N|f(n)|\dmu(n)=\sum_{n=1}^\infty|f(n)|=\sum_{n=1}^\infty|x_n|.
\end{equation*}
Cool!
\begin{proposition}
	Let $p\geq1$ be given. We define the set
	\begin{equation*}
		\ell^p(\N)=\left\{\mathbf{x}={(x_{n})}_{n\in\N}\subset\R\text{ such that }\sum_{n=1}^\infty\left|x_n\right|^{p}<\infty\right\}.
	\end{equation*}
	Then, $\ell^p(\N)$ is a vector space. Moreover, if
	\begin{equation*}
		\norm{\mathbf{x}}_{p}:=\left(\sum_{n=1}^\infty\left|x_n\right|^p\right)^{\frac{1}{p}}\qquad\every\mathbf{x}={(x_{n})}_{n}\in\ell^p(\N)
	\end{equation*}
	then $\left(\ell^p(\N),\norm{\cdot}_p\right)$ is a normed space.
\end{proposition}
\subsection{The space of linear applications}
\begin{proposition}
	Let $(X,\norm{\cdot}_X)$ and $(Y,\norm{\cdot}_Y)$ be two normed spaces and let $L;X\mapsto Y$ be a linear application. The following are equivalent:
	\begin{enumerate}[\circnum]
		\item\label{prop121}$L$ is continuous on $X$;
		\item \label{prop122}$L$ is continuous at $x=0$;
		\item\label{prop123}there is a positive constant $C>0$ such that
		\[
		\norm{L(x)}_Y\leq C\norm{x}_X\qquad\every x\in X.
		\]
	\end{enumerate}
\end{proposition} 
\begin{fancyproof}
	Of course $\ref{prop121}\implies\ref{prop122}$, but now let's prove \ref{prop123}. Consider the definition of continuity at 0 for the function $L$: this means that for any $\varepsilon>
	0$ there exists a constant $\delta>0$ such that
	\[
	\norm{x-0_X}_X\leq\delta\implies\norm{L(x)-L(0_Y)}_Y<\varepsilon.
	\]
	We know that $L$ is linear, so $L(0_X)=0_Y$ and therefore we get
	\begin{equation*}
		\norm{x}_X\leq\delta\implies\norm{L(x)}_Y<\epsilon.
	\end{equation*}
	Choose $\epsilon=1$ so that we get 
	\begin{equation*}
		\norm{x}_X\leq\delta\implies\norm{L(x)}_Y<1.
	\end{equation*}
	Let $x\in X\setminus\{0_X\}$ and set $y=\frac{\delta}{\norm{x}_X}x$. Now we have that
	\begin{equation*}
		\norm{y}_X=\norm{\frac{\delta}{\norm{x}_X}x}_{X}=\frac{\delta}{\cancel{\norm{x}_X}}\cancel{\norm{x}_X}=\delta
	\end{equation*}
	and, due to the linearity of $L$,
	\begin{equation*}
		\norm{L(y)}_Y=\norm{L\left(\frac{\delta}{\norm{x}_X}x\right)}_Y=\frac{\delta}{\norm{x}_X}\norm{L(x)}_Y=\frac{\delta}{\norm{x}_X}\norm{L(x)}_Y<1\implies\norm{L(x)}_Y<\frac{1}{\delta}\norm{x}_X.
	\end{equation*}
	So, setting $C=\frac{1}{\delta}$ proves $\ref{prop123}$. We still need to prove \ref{prop121}$\implies$\ref{prop123}: 'tis easy, since by linearity we have
	\begin{equation*}
		\norm{L(x)-L(y)}_{X}=\norm{L(x-y)}_X\leq C\norm{x-y}_Y\qquad\every x,y\in X
	\end{equation*}
	and this clearly implies the continuity of $L$ at any $x\in X$ (actually, the uniform continuity).
\end{fancyproof}
\begin{definition}
	If $(X,\dnorm_X)$ and $(Y,\dnorm_Y)$ are two normed spaces, we denote by $\mathscr{L}(X,Y)$ the space of continuous linear applications from $X$ to $Y$. If $X=Y$ we simply denote $\mathscr{L}(X)=\mathscr{L}(X,X)$.
\end{definition}
\begin{proposition}
	If $(X,\dnorm_X)$ and $(Y,\dnorm_Y)$ are two vector spaces and $X$ is of finite dimension, any linear application $L:X\mapsto Y$ is continuous.
\end{proposition}
\begin{remark}
	If $\mathrm{dim}(X)=n$ and $\mathrm{dim}(Y)=p$, the space $\mathscr{L}(X,Y)$ can be identified with the space $\mathscr{M}_{n\times p}(\R)$ of matrices with $n$ lines and $p$ rows.
\end{remark}
\subsection{Compactness}
\begin{definition}
	Let $(X,\dnorm_X)$ be a normed space and let $K\subset X$. We say that $K$ is \emph{compact} if every sequence ${(x_n)}_n$ contains a subsequence which converges to some $x\in K$.
\end{definition}Of course if $K$ is compact then it is closed.\begin{lemma}
If $K$ is a compact subset of a normed space $(X,\dnorm_X)$ then $K$ is closed and there exists $M>0$ such that $\sup_{x\in K}\norm{x}\leq M$ which means that $K$ is bounded.
\end{lemma}\begin{proposition}
\emph{Product of compact spaces}. Let $(X_1,\dnorm_1)$ and $(X_2,\dnorm_2)$ be two compact normed spaces and let $X=X_1\times X_2$. Then $(X,\dnorm_+)$ and $(X,\dnorm_{\max })$ are compact normed spaces.
\end{proposition}
Remember that 
\begin{equation*}
	\norm{\mathbf{x}}_+=\norm{x_1}_1+\norm{x_2}_2
\end{equation*}
and
\begin{equation*}
	\norm{\mathbf{x}}_{\max}=\max(\norm{x_1}_1,\norm{x_2}_2)
\end{equation*}
\begin{fancyproof}
	Let ${(\xi_{n})}_{n}\subset X$ be a given sequence. This means, being in a product space, that there exist two sequences ${(x_{1})}_{n}\subset X_{1}$ and ${(y_{n})}_{n}\subset X_{2}$ such that $\xi_{n}=(x_{n},y_{n})$ for any $n$. Consider the following subsequences:
	\begin{itemize}
		\item since $(X_1,\dnorm_1)$ is compact, there exists a subsequence ${(x_{\varphi(n)})}_{n}\subset X_1$ with a limit $x\in X_{1}$;
		\item take the analogous subsequence ${(y_{\varphi(n)})}_{n}\subset X_2$. Since $(X_2,\dnorm_2)$ is compact there is \textit{another} (sub-)subsequence ${(y_{\phi(\varphi(n))})}_{n}$ of ${(y_{\varphi(n)})}_{n}$ that converges to $y\in X_{2}$;
		\item now take the subsequence  ${(x_{\phi(\varphi(n))})}_{n}$ of ${(x_{\varphi(n)})}_{n}$ which is still convergent in $X_{1}$ to $x$.
	\end{itemize}
	We can easily see that the subsequence
	\begin{equation*}
		{(\xi_{\phi(\varphi(n))})}_{n}={\left((x_{\phi(\varphi(n))}),(y_{\phi(\varphi(n))})\right)}_{n}
	\end{equation*}
	is converging in $X$ to $\mathbf{x}=(x,y)$. So every subsequence $\xi_{n}$ contains a subsequence that converges to a point in $X$ and this means that $X$ is compact.
\end{fancyproof}
Of course, the above result readily extends to any finite product of compact normed spaces. On $\R$ it is easy to describe a large class of compact sets:
\begin{lemma}
	Let $\R$ be endowed with the absolute value, $|\cdot|$. Any interval $[a,b]\subset\R$ is compact.
\end{lemma}
\begin{proposition}
	Let $(X,\dnorm)$ be a normed space and let $K$ be a compact subset of $X$. If $A\subset K$ is a closed subset then $A$ is compact.
\end{proposition}
\begin{corollary}
	\emph{Heine-Borel theorem}. A subset $K$ of $\R^{N}$ (where $\R^{N}$ is endowed with, say, the usual Euclidean norm) is compact \ifonly{} it is closed and bounded.
\end{corollary}
\begin{fancyproof}
	\begin{enumerate}
		\item[$\impliedby$] We know that a compact subset of $\R^{N}$ is closed and bounded.
		\item[$\implies$] Let $K\subset\R^{N}$ be closed and bounded. Being bounded, there exists $R>0$ such that
		\begin{equation*}
			K\subset[-R,R]^{N}.
		\end{equation*}
		Since $K$ is closed, from the previous proposition it is sufficient to prove that $[-R,R]^{N}$ is a compact subset of $\R^{N}$ which is the same as checking that $[-R,R]$ is a compact subset of $\R$, since every closed subset of a compact subset is compact.
	\end{enumerate}
\end{fancyproof}
This corollary can be reformulated as:
\begin{center}
	Every bounded sequence of $\R^{N}$ has a convergent subsequence.
\end{center}
\subsection{Compactness and continuous functions}
\begin{proposition}
	Let $(X,\dnorm_X)$ and $(Y,\dnorm_Y)$ be two normed spaces and let $f:X\to Y$ be continuous. If $K\subset X$ is a compact subset of $X$ then $f(K)$ is a compact subset of $Y$.
\end{proposition}
\begin{fancyproof}
	Let ${(y_{n})}_{n}$ be a sequence of $f(K)$. It means that there is a sequence ${(x_{n})}_{n}\subset K$ such that $y_{n}=f(x_n)$ for any $n$. Since $K$ is compact, then there exists a subsequence ${\left(x_{\varphi(n)}\right)}_{n}$ of ${(x_{n})}_{n}$ which converges to $x\in K$. This means that $\lim_n\norm{x_{\varphi(n)}-x}_{X}=0$, but since $f$ is continuous then we know that 
	\begin{equation*}
		\lim_n\norm{f\left(x_{\varphi(n)}\right)-f(x)}_{Y}=0\iff	\lim_n\norm{y_{\varphi(n)}-f(x)}_{Y}=0.
	\end{equation*}
	Since $y=f(x)\in f(K)$, we get that any sequence of $f(K)$ has a subsequence which converges to a limit in $f(K)$, which means that $f(K)$ is compact.
\end{fancyproof}
This has the following consequence:
\begin{theorem}
	Let $\left(X,\dnorm\right)$ be a normed space and let $K\subset X$ be compact. Let $f:K\to\R$ be continuous. Then, $f$ assumes its maximum and minimum on $K$.
\end{theorem}
\begin{fancyproof}
	We know that $f(K)$ is a compact subset of $\R$ and thanks to the Heine-Borel theorem we know that any compact subset of $\R$ is closed and bounded (and viceversa...). So $f$ is bounded and the fact that it reaches its maximum and minimum inside $K$ is a simple consequence of the fact that $f(K)$ is closed. Let
	\begin{equation*}
		M=\sup\left\{f(x),x\in K\right\}.
	\end{equation*}
	By definition there is a sequence ${(x_{n})}_{n}\subset K$ such that $\lim_{n}f(x_{n})=M$. This sequence ${(f(x_{n}))}_{n}$ lies in $f(K)$ which is compact and is closed. But this means that also its limit $M$ lies in $f(K)$ (i.e.) there is $x\in K$ such that $f(x)=M$. This shows that $M$ is the maximum value of $f$ and by proceeding in an analogous manner we can show the same thing for the minimum.
\end{fancyproof}
\begin{theorem}
	\emph{Heine Theorem}. Let $(X,\dnorm_X)$ and $(Y,\dnorm_Y)$ be two normed spaces and let $K\subset X$ be compact. Assume that $f:K\to Y$ is continuous. Then $f$ is uniformly continuous on $K$.
\end{theorem}
\begin{fancyproof}
	Suppose that $f$ is not uniformly continuous. This means that there exists $\varepsilon_0>0$  such that
	\begin{equation*}
		\every\delta>0,\;\exists x,y\in K,\text{ with }\norm{x-y}_{X}<\delta\text{ and }\norm{f(x)-f(y)}_{Y}\geq\varepsilon_0	.
		\end{equation*}
		Now chossing $\delta=\frac{1}{n},n\in\N$ this allows to build two sequences ${(x_{n})}_{n}$ and ${(y_{n})}_{n}$ such that
		\begin{equation*}
			\norm{x_{n}-y_n}_{X}<\frac{1}{n}\text{ and }\norm{f(x_{n})-f(y_{n})}_{Y}\geq\varepsilon_0\qquad\every n\in\N.
		\end{equation*}
		Since $K$ is compact we can extract a subsequence of $x_n$ that we call ${\left(x_{\varphi(n)}\right)}_{n}$ and that converges to some $x_0\in K$. It follows that ${\left(y_{\varphi(n)}\right)}_{n}$ also converges to $x_0$ (why?). Since $f$ is continuous we get that $\lim_{n\to\infty}{\left(x_{\varphi(n)}\right)}=f(x_0)=\lim_{n\to\infty}f\left(y_{\varphi(n)}\right)$ in $Y$, i.e.
		\begin{equation*}
			\lim_{n \to \infty}\norm{f\left(x_{\varphi(n)}\right)-f\left(y_{\varphi(n)}\right)}_{Y}=0
		\end{equation*}
		but this contradicts the fact that $\norm{f\left(x_{\varphi(n)}\right)-f\left(y_{\varphi(n)}\right)}_{Y}\geq\varepsilon_0$ for each $n\in\N^{+}$. 
\end{fancyproof}
So, in this case taking two sequences that get closer and closer does not correspond to the fact also their functions get closer and closer... and this is not possible.
\subsection{Finite dimensional spaces}
\begin{proposition}
	Let $(X,\dnorm)$ be a finite dimensional normed vector space with $\mathsf{dim}(X)=d$ and let $\{\mathcal{e}_1,\ldots,\mathcal{e}_d\}$ be a basis for $X$. Then, there are positive constants $C_0,C_1>0$ such that
	\begin{equation*}
		C_0\sum_{i=1}^{d}|x_i|\leq\norm{\sum_{i=1}^{d}x_i\mathcal{e}_{i}}\leq C_1\sum_{i=1}^{d}|x_i|\qquad\every(x_1,\ldots,x_d)\in\R^{d}.
	\end{equation*}
\end{proposition}
This proposition asserts that if $\mathsf{dim}(X)=d$ then \textit{any} norm $\dnorm$ is related to the $\dnorm_1$ norm of $\R^{d}$. This translates in the following:
\begin{proposition}
	If $X$ is a finite dimensional vector space, all norms over $X$ are equivalent.
\end{proposition}
So there is no weird norm, but everything is comparable to the simple $\dnorm_1$ norm. This proposition also allows us to identify in a continuous way a finite dimensional space $(X,\dnorm)$ and the space $\R^{d}$ where $d$ is the dimension of $X$. Indeed, introducing a basis $\{\mathcal{e}_1,\ldots,\mathcal{e}_d\}$ of $X$, the mapping
\begin{equation*}
	\Phi:X\to\R^{d}
\end{equation*}
which associates $\Phi(\mathbf{x})=(x_1,\ldots,x_d)$ to some $\mathbf{x}=\sum_{i=1}^{d}x_i\mathcal{e}_i\in X$, is a bijection from $X$ to $\R^{d}$ which is continuous whose inverse is also continuous. This results in the following:
\begin{corollary}
	If $(X,\dnorm)$ is a finite dimensional vector space and $K\subset X$ is closed and bounded then $K$ is compact.
\end{corollary}
Again, this is very specific to finite dimensional spaces and, as we shall see, this actually
characterizes finite dimensional spaces. Indeed, in infinite dimensional normed spaces, the
closed unit ball cannot be compact. This shows that, in infinite dimensional spaces, the
compact subsets do not coincide with closed and bounded subsets!!
We first state the following technical lemma:
\begin{lemma}
	\emph{Riesz Lemma}. Let $(X,\dnorm)$ be a normed vector space and let $Y$ be a closed subspace of $X$ (i.e. $Y$ is closed in $X$ and $Y$ is a linear subspace of $X$). If $Y\neq X$ then for any $\varepsilon\in(0,1)$ there exists $x\in X$ with $\norm{x}=1$ such that
	\begin{equation*}
		\inf_{y\in Y}\norm{x-y}\geq1-\varepsilon.
	\end{equation*}
\end{lemma}
\begin{remark}
	This lemma asserts that if $Y\neq X$ is a closed subspace then for any $\varepsilon\in(0,1)$ there is some unit vector $x\in X$ such that $\mathsf{dist}(x,Y)\geq 1-\varepsilon$.
\end{remark}
\begin{fancyproof}
	Let $z\in X\setminus Y$. Since $Y$ is closed and $z\notin Y$, one has
	\begin{equation*}
		\alpha=\mathsf{dist}(z,Y)=\inf_{y\in Y}\norm{z-y}>0.
	\end{equation*}
	Pick $\varepsilon\in(0,1)$. There exists $\overline{y}\in Y$ such that $\norm{\overline{y}-z}\leq\frac{\alpha}{1-\varepsilon}$ (otherwise we would get $\mathsf{dist}(z,Y)\geq\frac{\alpha}{1-\varepsilon}>\alpha$!). Notice that $\overline{y}\neq z$ so that $r:+\norm{\overline{y}-z}>0$. Set 
	\begin{equation*}
		x:=\frac{1}{r}(z-\overline{y}).
	\end{equation*}
	Clearly, $\norm{x}=1$. Let $y\in Y$ be given. One can write
	\begin{equation*}
		\norm{x-y}=\frac{1}{r}\norm{z-\overline{y}-ry}
	\end{equation*}
	and since $Y$ is a linear subspace $\overline{y}+ry\in Y$ so that $\norm{z-\overline{y}-ry}\geq \alpha$. Therefore $\norm{x-y}\geq\frac{\alpha}{r}\geq 1-\varepsilon$ by assumption on $r=\norm{z-\overline{y}}$. Since this is true for any $y\in Y$, this proves the result.
\end{fancyproof}
\begin{lemma}
	If $\left(X,\dnorm\right)$ is a normed space, any linear subspace of finite dimension is closed. 
\end{lemma}
\begin{theorem}
	\emph{Riesz Theorem}. A normed space $\left(X,\dnorm\right)$ is finite dimensional if and only if the closed unit ball $B_{c}(0,1)=\left\{x\in X;\norm{x}\leq 1\right\}$ of $X$ is compact.
\end{theorem}
\begin{fancyproof}
	We already know that in finite dimensional spaces the closed and bounded subsets are compact. Assume that $B_{c}(0,1)$ is compact. Argue by contradiction that $X$ is infinite dimensional. Let us pick $\mathcal{e}_{0}\in X$ with $\norm{\mathcal{e}_{0}}=1$ and let $$F_0=\mathsf{Span}(\mathcal{e}_{0})=\left\{t\mathcal{e}_{0},t\in\R\right\}.$$
	According to the previous Lemma, $F_{0}$ is a closed linear subspace of $X$ (since it is of dimension 1). Being $X$ infinite dimensional, $F_{0}\neq X$. According to Riesz Lemma, there exists $\mathcal{e}_{1}\notin F_{0}$ such that $\norm{e_{1}}=1$ and $\min_{x\in F_{0}}\norm{x-\mathcal{e}}_{1}\geq\unmezz$. Set then $F_{1}=\mathsf{Span}(\mathcal{e}_{0},\mathcal{e}_{1})$. One constructs inductively a sequence ${(\mathcal{e})}_{n}\subset X$ such that $\norm{\mathcal{e}_{n}}=1$ and such that the finite dimensional space $F_{n}=\mathsf{Span}(\mathcal{e}_{0},\ldots,\mathcal{e}_{n})$ satisfies
	\begin{equation*}
		\mathcal{e}_{n}\notin F_{n-1}\qquad\text{and }\inf_{x\in F_{n-1}}\norm{x-\mathcal{e}_{n}}\geq\unmezz.
	\end{equation*}
	In particular, ${(\mathcal{e}_{n})}_{n}\subset B_{c}(0,1)$ which is compact so that a subsequence of ${(e_{n})}_{n}$ should converge. But, for any $n>m\in\N$, $\norm{\mathcal{e}_{n}-\mathcal{e}_{m}}\geq\unmezz$ (since $\mathcal{e}_{n}\notin F_{m}$). This is a contradiction.
\end{fancyproof}
\section{Banach spaces}
\subsection{Cauchy sequences \& Banach spaces}
\begin{definition}
	Let $(X,\dnorm)$ be a normed space. A sequence ${(x_{n})}_{n}\subset X$ of elements in $X$ is a Cauchy sequence if for any $\varepsilon>0$ there exists $N=N(\varepsilon)\in\N$ such that
	\begin{equation*}
		\norm{x_{n}-x_{m}}\leq\varepsilon\qquad\every n,m\geq\N.
	\end{equation*}
\end{definition}
One easily checks that every Cauchy sequence ${(x_{n})}_{n}\subset X$ is bounded. Indeed, by definition, for any $\varepsilon>0$ there is $N\in\N$ such that
\begin{equation*}
	\norm{x_{n}-x_{m}}<\varepsilon\qquad\every n,m\geq N.
\end{equation*} 
For, say, $\varepsilon=1$ and $m=N$ we see that for $\every n\geq N$
\begin{equation*}
	\norm{x_{n}}=\norm{x_{n}-x_{N}+x_{N}}\leq\norm{x_{n}-x_{N}}+\norm{x_{N}}\leq 1+\norm{x_{N}}.
\end{equation*}
Thus with $C_{1}=1+\norm{x_{N}}$
\begin{equation*}
	\sup_{n\geq N}\norm{x_{n}}\leq C_{1}.
\end{equation*}
Setting now $C_{2}:\max\left\{\norm{x_{1}},\ldots,\norm{x_{N-1}}\right\}$ one sees that $C_{2}$ is finite since it is the maximum of only a finite number of real numbers. By definition
\begin{equation*}
	\norm{x_{n}}\leq C_{2}\qquad\every n<N.
\end{equation*}
Therefore
\begin{equation*}
	\sup_{n}\norm{x_{n}}\leq C=\max\left\{C_1,C_2\right\}<\infty
\end{equation*}
which means $\{(x_{n})\}_{n}$ is bounded.
\begin{lemma}
	Let $(X,\dnorm)$ be a normed space. Any convergent sequence is a Cauchy sequence.
\end{lemma}
\begin{fancyproof}
	Let ${(x_{n})}_{n}\subset X$ be a convergent sequence in $X$ and let $x$ be its limit. By definition, given $\varepsilon>0$, there is $N\in\N$ such that $\norm{x_{n}-x}<\varepsilon$ for any $n\geq N$. Pick then $n,m\geq N$. From the triangle inequality
	\begin{equation*}
		\norm{x_{n}-x_{m}}\leq\norm{x_{n}-x}+\norm{x_{m}-x}<2\varepsilon.
	\end{equation*}
	Since $\varepsilon>0$ is arbitrary, we proved that the sequence is a Cauchy sequence (since we can rename $2\varepsilon$ as $\varepsilon$).
\end{fancyproof}
The opposite is not true... We introduce the set $c_{00}(\N)$ of all sequences of real number that have only finitely many nonzero components, which means
\begin{equation*}
	c_{00}=\left\{x={(x_{n})}_{n}\subset X,\exists N\in\N,x_{k}=0\quad\every k\geq N\right\}.
\end{equation*}
This is a subset of the space $\ell^{1}(\N)$ of finitely summable  sequences and in particular it is a norm space with respect to the norm induced by the one in $\ell^{1}(\N)$ (which is $\sum_{n=0}^{\infty}|x_{n}|$). Now for any $n\in\N$ let $\bm{x}^{(n)}$ be the sequence given by
\begin{equation*}
	\bm{x}^{(n)}=(2^{-1},2^{-2},\ldots,2^{-n},0,0,\ldots)
\end{equation*}
so that $x^{(n)}={\left(x_{k}^{(n)}\right)}_{k}$ with \begin{equation*}
	x_{k}^{(n)}=\begin{cases}
		2^{-k}&\text{if }k\leq n\\
		0&\text{if }k>n.
	\end{cases}
\end{equation*}
Clearly, $\bm{x}^{(n)}\in c_{00}(\N)$ for any $n\in\N$. Therefore the family ${\left(\bm{x}^{(n)}\right)}_{n}$ is a sequence of elements of $c_{00}(\N)$ (it is a sequence of sequences...) and if $m<n$
\begin{equation*}
	\norm{\bm{x}^{(n)}-\bm{x}^{(m)}}_{1}=\sum_{k=1}^{\infty}\left|x_{k}^{(n)}-x_{k}^{(m)}\right|=\sum_{k=m+1}^{n}2^{-k}.
\end{equation*}
${\left(\bm{x}^{(n)}\right)}_{n}$ is a Cauchy sequence in $\left(c_{00}(\N),\dnorm_{1}\right)$. However, it does not converge in $\left(c_{00}(\N),\dnorm_{1}\right)$.
\begin{fancyproof}
	Assume the contrary: there exists 
	\begin{equation*}
		\bm{x}={\left(x_{k}\right)}_{k}\in c_{00}(\N)\text{ such that }\lim\bm{x}^{n}=\bm{x}.
	\end{equation*}
	But then for any $k\in\N$ it must hold
	\begin{equation*}
		\lim_{n\to\infty}x_{k}^{(n)}=\bm{x}_{k}
	\end{equation*}
	and therefore it must hold that
	\begin{equation*}
		\bm{x}_{k}=2^{-k}\qquad\every k\in\N.
	\end{equation*}
	But the sequence $\bm{x}={\left(2^{-k}\right)}_{k}$ does not belong to $c_{00}(\N)$!
\end{fancyproof}
\begin{lemma}
	Let $(X,\dnorm)$ be a given normed space and let ${(x_{n})}_{n}$ be a Cauchy sequence in $X$. If $\seqn{x}$ admits a limit point then it is convergent.
\end{lemma}
\begin{definition}
	A normed space $\normsp{X}$ is said to be \emph{complete} if any Cauchy sequence is convergent in $X$. A complete normed space $\normsp{X}$ is called a \emph{Banach space}.
\end{definition}
The most fundamental example of complete normed space is the set of real numbers.
\begin{theorem}
	If $X=\R$ is endowed with the usual norm $|\cdot|$ for any $x,y\in\R$ then $(\R,|\cdot|)$ is a complete normed space. In other words, any Cauchy sequence in $\R$ is convergent.
\end{theorem}
\begin{definition}
	Let $\normsp{X}$ be a normed space and let $\seqn{x}\in X$ be a given sequence. We define the sequence ${(s_{N})}_{N}\in X$ of partial sums:
	\begin{equation*}
		s_{N}:=\sum_{n=1}^{N}x_{n},\qquad N\in\N.
	\end{equation*} 
	The series $\sum_{n}x_{n}$ is said to be convergent if there exists $x\in X$ such that
	\begin{equation*}
		\lim_{N\to\infty}\norm{s_{N}-x}=0.
	\end{equation*}
	We write then 
	\begin{equation*}
		x=\sum_{n=1}^{\infty}x_{n}.
	\end{equation*}
	The series $\sum_{n}x_{n}$ is said to be absolutely convergent if 
	\begin{equation*}
		\sum_{n}\norm{x_{n}}<\infty.
	\end{equation*}
\end{definition}
\begin{theorem}
	Let $\normsp{X}$ be a normed space. The following are equivalent:
	\begin{enumerate}
		\item $\normsp{X}$ is a Banach space;
		\item Every absolutely convergent series is convergent in $\normsp{X}$, i.e. for any $\seqn{x}\subset X$,
		\begin{equation*}
			\sum_{n=1}^{\infty}\norm{x_{n}}<\infty\implies{\left(\sum_{n=1}^{N}x_{n}\right)}_N\text{ converges in }\normsp{X}.
		\end{equation*}
	\end{enumerate}
\end{theorem}\label{theo213}
\begin{fancyproof}
	\begin{enumerate}
		\item[$(1\Rightarrow 2)$] Let $\seqn{x}$ be a sequence such that $\sum_{n}\norm{x_{n}}<\infty$ and let 
		\begin{equation*}
			S_{n}=\sum_{k=1}^{n}x_{k},\qquad k\in\N.
		\end{equation*}
		We need to prove that ${(S_{N})}_{N}$ converges in $\normsp{X}$. Since, by assumption, $(X,\dnorm)$ is a Banach space, it is enough to show that $\seqn{S}$ is a Cauchy sequence. Thus, for $m>n\geq 1$, we compute
		\begin{equation*}
			\norm{S_{m}-S_{n}}=\norm{\sum_{k=1}^{m}x_{k}-\sum_{k=1}^{n}x_{k}}=\norm{\sum_{k=n+1}^{m}x_{k}}.
		\end{equation*}
		Thanks to the triangle inequality, we deduce that
		\begin{equation*}
			\norm{S_{m}-S_{n}}\leq\sum_{k=n+1}^{m}\norm{x_{k}}\leq\sum_{k=n+1}^{\infty}\norm{x_{k}}\to0\qquad\text{as }n\to\infty.
		\end{equation*}
		since the numerical series $\sum_k\norm{x_{k}}<\infty$. This proves that
		\begin{equation*}
			\lim_{m\to\infty,n\to\infty}\norm{S_{m}-S_{n}}=0
		\end{equation*}
		which means that $\seqn{S}$ is a Cauchy sequence. It is therefore convergent and the absolutely convergent series $\sum_k x_k$ is convergent.
		\item[$(2\Rightarrow1)$] Let us assume now that for any sequence $\seqn{y}\subset X$ one has
		\begin{equation*}
			\sum_k\norm{y_k}<\infty\implies\lim_n\sum_{k=1}^{n}y_k\qquad\text{exists in }(X,\dnorm)\tag*{\faPiedPiper}\label{piedpaper}
		\end{equation*}
		and let $\seqn{x}$ be a Cauchy sequence in $\normsp{X}$. Thus, for any $\varepsilon>0$ there is $N_\varepsilon\in\N$ such that
		\begin{equation*}
			\norm{x_n-x_m}\leq\varepsilon\qquad\every n,m\geq N_\varepsilon.
		\end{equation*}
		Applying this with $\varepsilon=\frac{1}{2^{k}}$, $k\in\N$, it is easy to check that one can construct a strictly increasing sequence $\seqk{n}\subset\N$ of indices such that
		\begin{equation*}
			\norm{x_{n_{k+1}}-x_{n_{k}}}\leq\frac{1}{2^{k}}\qquad k\in\N.
		\end{equation*}
		Therefore, setting $y_k=x_{n_{k+1}-x_{n_{k}}}$ one has $\sum_k\norm{y_k}\leq\sum_k\frac{1}{2^{k}}<\infty$. Applying \ref{piedpaper} we deduce that there exists $y\in X$ such that
		\begin{equation*}
			\lim_{n\to\infty}\norm{\sum_{j=1}^{k}y_{j}-y}=0.
		\end{equation*}
		Observing that, for any $k\in\N$
		\begin{equation*}
			\sum_{j=1}^{k}y_{j}=\sum_{j=1}^{k}\left(x_{n_{j+1}}-x_{n_{j}}\right)=x_{n_{k+1}}-x_{n_{1}},
		\end{equation*}
		one sees that $\seqkk{x_{n_{k+1}}}$ converges to $x=x_{n_{1}}+Y$ in $X$. Thus, $\seqn{x}$ has a subsequence which converges, Since it is a Cauchy sequence, the whole sequence $\seqnn{x}$ actually converges and this proves the completeness of $\normsp{X}$.
	\end{enumerate}
\end{fancyproof}\subsection{Examples revisited}
\begin{proposition}
	The space $\left(\ell^{1}(\N),\dnorm_{1}\right)$ is a Banach space.
\end{proposition}
\begin{fancyproof}
	Let $\seqnn{\bm{x}^{(n)}}$ be a Cauchy sequence in $\ell^{1}(\N)$. Recall that, for any $n\in\N$, $\bm{x}^{(n)}$ is a sequence given by
	\begin{equation*}
		\bm{x}^{(n)}=\seqkk{\bm{x}^{(n)}_{k}}.
	\end{equation*}
	Since it is a Cauchy sequence, for any $\varepsilon>0$ there is $N>0$ such that
	\begin{equation*}
		\norm{\bm{x}^{(n)}-\bm{x}^{(m)}}_{1}=\sum_{k=1}^{\infty}\left|\bm{x}_{k}^{(n)}-\bm{x}^{(m)}\right|<\varepsilon\qquad\every n,m>N.
	\end{equation*}
	It is easy to see that for any $k\in\N$ one has
	\begin{equation*}
		\left|\bm{x}_{k}^{(n)}-\bm{x}_{k}^{(n)}\right|\leq\norm{\bm{x}^{(n)}-\bm{x}^{(m)}}_{1}<\varepsilon
	\end{equation*}
	i.e. for any $k\in\N$ fixed the sequence $\seqnn{\bm{x}^{(n)}_{k}}\subset\R$ is a Cauchy sequence in $\R$ endowed with its natural distance. Since $\R$ is complete by the previous theorem, it must converge which means that for any $k\in\N$
	\begin{equation*}
		x_{k}=\lim_{n\to\infty}\bm{x}^{(n)}_{k}
	\end{equation*}
	exists in $\R$. We introduce the sequence $\bm{x}=\seqk{x}$ and let us show that $\bm{x}\in\ell^{1}(\N)$ and that $\bm{x}=\lim_{n}\bm{x}^{(n)}$ for the $\ell^{1}(\N)$-distance. To do so, we fix $\varepsilon>0$. Because $\seqnn{\bm{x}^{(n)}}$ is a Cauchy sequence in $\ell^{1}(\N)$, there exists $N>0$ such that
	\begin{equation*}
		\norm{\bm{x}^{(n)}-\bm{x}^{(m)}}_{1}<\varepsilon\qquad n,m>N.
	\end{equation*}
	Let us fix $n>N$ and let us fix an integer $k_{0}>0$. Since $\lim_{m\to\infty}\bm{x}^(m)_{k}=\bm{x}_{k}$ for any $k\geq 1$ and since the $k_{0}$ is finite one has
	\begin{equation*}
		\sum_{k=1}^{k_{0}}\left|\bm{x}^{(n)}-\bm{x}_{k}\right|=\sum_{k=1}^{k_{0}}\lim_{m\to\infty}\left|\bm{x}^{(n)}_{k}-\bm{x}_{k}^{(m)}\right|\leq\lim_{m\to\infty}\norm{\bm{x}^{(n)}-\bm{x}^{(m)}}_{1}<\varepsilon.
	\end{equation*}
	Since this is true for any $k_{0}>0$ and 
	\begin{equation*}
		\norm{\bm{x}^{(n)}-\bm{x}}_{1}=\lim_{k_{0}\to\infty}\sum_{k=1}^{k_{0}}\left|\bm{x}^{(n)}_{k}-\bm{x}_{k}\right|
	\end{equation*}
	one deduces that $\norm{\bm{x}^{(n)}-\bm{x}}_{1}<\varepsilon$ for any $n\geq N$ which proves that the sequence is converging to $\bm{x}$ and in particular that $\bm{x}\in\ell^{1}(\N)$.
\end{fancyproof}
\begin{proposition}
	For any $p\in[1,\infty]$ the normed space $\left(\ell^{p}(\N),\dnorm_{p}\right)$ is a Banach space. Moreover, $\left(\ell^{\infty}(\N),\dnorm_{\infty}\right)$ is also a Banach space.
\end{proposition}
\begin{proposition}
	For any compact interval $I\subset\R$, the normed space $\left(\mathcal{C}(I),\dnorm_{\infty}\right)$ is a Banach space.
\end{proposition}
The norm $\dnorm_{\infty}$ in this case measures the max distance between two functions in $\mathcal{C}(I)$.
\begin{proposition}
	For any compact interval $I\subset\R$, the normed space $\left(\mathcal{C}(I),\dnorm_{1}\right)$ is not a Banach space.
\end{proposition}
This is because $\dnorm_{\infty}$ is stronger than $\dnorm_{1}$. There are some functions that are continuous but converge (according to $\ell^{1}$ norm $\norm{f_{n}-f}_{1}\xrightarrow{n\to\infty}0$) to a discontinuous function (not in $\mathcal{C}(I)$). But if our criterion of convergence is $\norm{f_{n}-f}_{\infty}\xrightarrow{n\to\infty}0$ then we cannot find such functions.
\begin{theorem}
	\emph{Fischer-Riesz Theorem.} Let $(S,\Sigma,\mu)$ be a given measure space. For any $1\leq p\leq\infty$, $\Lp(S,\Sigma,\mu)$ is a Banach space.
\end{theorem}
\begin{fancyproof}
	To prove this we need this lemma:
	\begin{lemma}
		Let $(S,\Sigma,\mu)$ be a given measure space and let $1\leq p<\infty$. Assume that $\seqn{f}\subset\Lp(S,\Sigma,\mu)$ is a sequence such that
		\begin{equation*}
			\sum_{n=1}^{\infty}\norm{f_{n}}_{p}<\infty.
		\end{equation*}
		Then the series 
		\begin{equation*}
			\sum_{n=1}^{\infty}f_{n}
		\end{equation*}
		converges almost everywhere and in $\Lp(S,\Sigma,\mu)$ which means that there exists $f\in\Lp(S,\Sigma,\mu)$ such that
		\begin{equation*}
			\lim_{N\to\infty}\norm{f-\sum_{n=1}^{\infty}f_{n}}_{p}=0.
		\end{equation*}
	\end{lemma}
	From now on we simply write $\Lp(S,\mu)$ instead of $\Lp(S,\Sigma,\mu)$. For the proof we distinguish between the cases $1\leq p<\infty$ and $p=\infty$.
	\begin{itemize}
		\item By virtue of the previous Lemma, in such a case any absolutely convergent series $\sum_{n}f_{n}$ in $\Lp(S,\Sigma,\mu)$ is actually convergent. From \hyperref[theo213]{this} theorem we know that this proves that $\left(\Lp(S,\Sigma,\mu)\dnorm_{p}\right)$ is a Banach Space.
		\item Assume that $\seqn{f}$ is a Cauchy sequence in $L^{\infty}(S,\mu)$. Given an integer $k\geq 1$ there is an integer $N_{k}$ such that
		\begin{equation*}
			\norm{f_{m}-f_{n}}_{\infty}\leq\frac{1}{k}\qquad\every n,m\geq N_{k}.\tag*{\faMoneyCheck}\label{novabs}
		\end{equation*}
		Indeed, given $m,n\geq N_{k}$ by definition there is a null set $E_{k}(m,n)$ for which $\left|f_{m}(k)-f_{n}(x)\right|\geq\frac{1}{k}$ for $x\in S\setminus E_{k}(m,n)$. Taking then $E_{k}=\bigcup_{n,m\geq N_{k}}E_{k}(m,n)$ one has $E_{k}$ null set and the above property holds on $S\setminus E_{k}$ for all $n,m\geq N_{k}$. Then we let 
		\begin{equation*}
			E=\bigcup_{k=1}^{\infty}E_{k}
		\end{equation*}
		so that $\mu(E)=0$ and get that, for any $x\in S\setminus E$ the sequence $\seqnn{f_{n}(x)}$ is a Cauchy sequence in $\R$. Thus for any $x\in S\setminus E$, 
		\begin{equation*}
			\lim_{n}f_{n}(x)
		\end{equation*}
		exists and we denote it by $f(x)$. Passing to the limit in \ref{novabs} as $m\to\infty$ we obtain then that 
		\begin{equation*}
			\left|f(x)-f_{n}(x)\right|\leq\frac{1}{k}\qquad\every x\in S\setminus E,\;\every n\geq N_{k}
		\end{equation*}
		from which we see that $f\in L^{\infty}(S,\mu)$ with
		\begin{equation*}
			\norm{f-f_{n}}_{\infty}\leq\frac{1}{k}\qquad\every n\geq N_{k}.
		\end{equation*}
		Letting then $n\to\infty$ and then $k\to\infty$ we get that 
		\begin{equation*}
			\lim_{n}\norm{f-f_{n}}_{\infty}=0.
		\end{equation*}
		This proves that $\left(L^{\infty}(S,\mu),\dnorm_{\infty}\right)$ is complete.
	\end{itemize}
\end{fancyproof}
\begin{proposition}
	Let $1\leq p\leq\infty$ and let $\seqn{f}$ be a Cauchy sequence in $\Lp(S,\mu)$ which converges to $f$. Then there is a subsequence $\left(f_{n_{k}}\right)$ which converges $\mu$-almost everywhere to $f$.
\end{proposition}
\begin{proposition}
	Let $\left(X,\dnorm_{X}\right)$ and $\left(Y,\dnorm_{Y}\right)$ be two normed spaces and assume that $(Y,\dnorm_{Y})$ is a Banach space. Then the space $\left(\mathscr{L}(X,Y),\dnorm_{\mathscr{L}(X,Y)}\right)$.
\end{proposition}
Remember that 
\begin{equation*}
	\norm{L}_{\mathscr{L}(X,Y)}=\norm{L}_{\mathsf{op}}=\sup_{\norm{x}_{X}\leq 1}\norm{L(x)}_{Y}\qquad\every L\in\mathscr{L}(X,Y).
\end{equation*}
This measures how much the linear mapping ``stretches'' a small $x\in X$ when it maps it to $Y$.
\begin{fancyproof}
	Let $\seqn{L}$ be a sequence of linear and continuous applications from $X$ to $Y$. Assume that $\seqn{L}$ is a Cauchy sequence for $\dnorm_{\mathscr{L}(X,Y)}$, i.e. for any $\varepsilon>0$ there exists $N\in\N$ such that
	\begin{equation*}
		\norm{L_{n}-L_{m}}_{\mathscr{L}(X,Y)}<\varepsilon\qquad \every n,m\geq N.
	\end{equation*}
	By definition of the norm, it holds 
	\begin{equation*}
		\norm{L_{n}(x)-L_{m}(x)}_{Y}<\varepsilon\norm{x}_{X}\qquad\every n,m\geq  N\quad\every x\in X.
	\end{equation*}
	In particular, for any $x\in X$ the sequence $\seqnn{L_{n}(x)}$ is a Cauchy sequence in $Y$ and since $Y$ is complete, the sequence converges to $L(x)$. This defines a mapping $L:X\to Y$ and it is not difficult to prove that $L$ is linear. Let us check then that $L$ is continuous and that $\norm{L_{n}-L}_{\mathscr{L}(X,Y)}\to0$ as $n\to\infty$. Taking the limit as $m\to\infty$ in the previous inequality one gets that 
	\begin{equation*}
		\norm{L_{n}(x)-L(X)}_{Y}\leq\varepsilon\norm{x}_{X}\qquad\every n\geq N\quad\every x\in X.\tag*{\faGasPump}\label{gas}
	\end{equation*}
	In particular for any fixed $n\geq N$ and any given $x\in X$
	\begin{equation*}
		\norm{L(x)}_{Y}\leq\norm{L_{n}(x)}_{Y}+\varepsilon\norm{x}_{X}
	\end{equation*}
	and since $L_{n}$ is continuous there exists $C_{n}>0$ such that
	\begin{equation*}
		\norm{L(x)}_{Y}\leq C_{n}\norm{x}_{X}+\varepsilon\norm{x}_{X}.
	\end{equation*}
	This shows that $L\in\mathscr{L}(X,Y)$. Now, \ref{gas} exactly means that $\norm{L_{n}-L}_{\mathscr{L}(X,Y)}<\varepsilon$ for any $n\geq N$ which proves that $\seqn{L}$ converges to $L$ in $\left(\mathscr{L}(X,Y),\dnorm_{\mathscr{L}(X,Y)}\right)$
\end{fancyproof}
\begin{corollary}
	Let $\normspp{X}$ be a Banach space and let $L\in\mathscr{L}(X)$ be such that
	\begin{equation*}
		\norm{L}_{\mathscr{L}(X)}<1.
	\end{equation*}
	Then, the linear mapping $I-L$ is invertible (i.e. $I-L$ is bijective and its inverse $(I-L)^{-1}\in\mathscr{L}(X)$) and
	\begin{equation*}
		(I-L)^{-1}=\sum_{n=0}^{\infty}L^{n}
	\end{equation*}
	where $I$ is the identity mapping of $X$ and $L^{n}$ is defined inductively by $L^{0}=I$ and $L^{n+1}=L\circ L^{n}=L^{n}\circ L$ for any $n\geq 0$.
\end{corollary}
\begin{fancyproof}
	For any $n\in\N$ set
	\begin{equation*}
		A_{n}:=\sum_{k=0}^{n}L^{k}.
	\end{equation*}
	Clearly $A_{n}$ is a linear mapping as a composition and sum of linear mappings. For the same reason it is continuous, i.e. $A_{n}\in\mathscr{L}(X)$. For any $m>n$ it holds
	\begin{align*}
		\norm{A_{m}-A_{n}}_{\mathscr{L}(X)}&=\norm{\sum_{k=n+1}^{m}L^{k}}_{\mathscr{L}(X)}\\
		&\leq\sum_{k=n+1}^{m}\norm{L^{k}}_{\mathscr{L}(X)}.
	\end{align*}
	One easily checks by induction that $\norm{L^{k}}_{\mathscr{L}(X)}\leq\norm{L}_{\mathscr{L}(X)}\norm{L^{k}}_{\mathscr{L}(X)}$ for any $k\geq 0$. Thus
	\begin{equation*}
		\norm{L^{k}}_{\mathscr{L}(X)}\leq\norm{L}^{k}_{\mathscr{L}(X)}\qquad\every m>n.
	\end{equation*}
	Then $\seqn{A}$ is a Cauchy sequence in $\mathscr{L}(X)$. According to the previous theorem, $\seqn{A}$ converges to some linear mapping, say, $A\in\mathscr{L}(X)$. Let us now prove that $I-L$ is necessarily invertible and that $A=(I-L)^{-1}$. For any $n\geq 0$, it holds that
	\begin{equation*}
		(I-L)\circ A_{n}=A_{n}\circ(I-L)=I-L^{n+1}.
	\end{equation*}
	In particular
	\begin{equation*}
		\norm{(I-L)\circ A_{n}-I}_{\mathscr{L}(X)}\leq\norm{L}^{n+1}_{\mathscr{L}(X)}.
	\end{equation*}
	Letting $n\to\infty$ one sees that $(I-L)\circ A_{n}$ converges to $I$ in $\mathscr{L}(X)$. By continuity, this means that 
	\begin{equation*}
		(I-L)\circ A=I.
	\end{equation*}
	In the same way we have $A\circ(I-L)=I$. This shows that $(I-L)$ is invertible and that $A$ is its inverse.
\end{fancyproof}
\begin{definition}
	If $\normsp{X}$ is a normed space, we define the dual space of $X$ denoted by $X^{\star}$ the space of all continuous and linear mappings $\Phi:X\to\R$, i.e.
	\begin{equation*}
		X^{\star}=\mathscr{L}(X,\R).
	\end{equation*}
	We also denote
	\begin{equation*}
		\norm{\Phi}_{\star}=\norm{\Phi}_{\mathscr{L}(X,\R)}=\sup_{\norm{x}_{X}\leq 1}|\Phi(x)|\qquad\every\Phi\in X^{\star}.
	\end{equation*}
\end{definition}
\begin{corollary}
	If $\normsp{X}$ is a normed vector space then $\left(X^{\star},\dnorm_{\star}\right)$ is a Banach space.
\end{corollary}
\begin{theorem}
	\emph{Riesz representation theorem}. Given $1<p<\infty$ and $\Phi\in\left(\Lp(S,\mu)\right)^{\star}$ there exists a unique $g\in\Lq(S,\mu)$ (with $\frac{1}{p}+\frac{1}{q}=1$) such that
	\begin{equation*}
		\Phi(f)=\int_{S}fg\dmu\qquad\every f\in\Lp(S,\mu)
.	\end{equation*}
Moreover,
\begin{equation*}
	\norm{\Phi}_{\star}=\norm{g}_{q}.
\end{equation*}
If $(S,\Sigma,\mu)$ is $\sigma$-finite, then the result is still true for $p=1$.
\end{theorem}
\subsection{Simple consequences of completeness}
\begin{proposition}
	Let $\normsp{X}$ be a Banach space and let $\seqn{x}\subset X$ be such that
	\begin{equation*}
		\sum_{n=1}^{\infty}\norm{x_{n}}<\infty.
	\end{equation*}
	Then the series $\sum_{n=1}^{\infty}x_{n}$ converges in $X$, i.e. there exists $x\in X$ such that
	\begin{equation*}
		\lim_{N\to\infty}\norm{\sum_{n=1}^{N}x_{n}-x}=0.
	\end{equation*}
\end{proposition}
\begin{definition}
	Let $\normspp{X}$ and $\normspp{Y}$ be two normed spaces and $k\in(0,1)$. A function $f:X\to Y$ is said to be a $k$-contraction mapping if
	\begin{equation*}
		\norm{f(x)-f(y)}_{y}\leq k\norm{x-y}_{X}\qquad\every x,y\in X\times X.
	\end{equation*}
\end{definition}
It is obvious that any $k$ contraction mapping is continuous. This is clear that a $k$-contraction mapping is an application which ``contracts'' distances: the images of $x$ and $y$ through $f$ are closer to each other than $x,y$.
\begin{theorem}
	\emph{Banach fixed point theorem}. Let $\normsp{X}$ be a complete normed space and let $f:X\to X$ be a $k$-contraction mapping with $k\in(0,1)$. Then there exists a unique fixed point $\bm{a}\in X$ for $f$, i.e. there is a unique $\bm{a}\in X$ such that
	\begin{equation*}
		f(\bm{a})=\bm{a}.
	\end{equation*}
\end{theorem}
\begin{fancyproof}
	We first prove that $f$ can admit only one fixed point. Assume $\bm{a},\bm{b}$ are both fixed points of $f$. Since $f$ is a $k$ contraction mapping we have
	\begin{equation*}
		\begin{carray}
			\norm{f(\bm{a})-f(\bm{b})}\leq k\norm{\bm{a}-\bm{b}}\\
			\Downarrow\\
		\norm{\bm{a}-\bm{b}}\leq k\norm{\bm{a}-\bm{b}}.
		\end{carray}
	\end{equation*}
	Since $k>0$ and $\norm{\bm{a},\bm{b}}\geq0$ one sees necessarily that $\norm{\bm{a}-\bm{b}}=0$  i.e. $\bm{a}=\bm{b}$. \par
	To prove now the existence of some fixed point, one proves actually that the sequence defined in the above statement converges to some fixed-point of $f$. Let then $x_{0}\in X$ be given and define inductively 
	\begin{equation*}
		x_{n+1}=f(x_{n}).
	\end{equation*}
	Since $f$ is a $k$ contraction mapping we have
	\begin{equation*}
		\norm{x_{n+1}-x_{n}}\leq k\norm{x_{n}-x_{n-1}}
	\end{equation*}
	for any $n\geq 1$ and we easily deduce that 
	\begin{equation*}
		\norm{x_{n+1}-x_{n}}\leq k^{n}\norm{x_{1}-x_{0}}\qquad\every n\in\N.
	\end{equation*}
	If $m>n\geq 1$ are given, one deduces from the above inequality together with the triangle inequality that
	\begin{equation*}
		\norm{x_{n}-x_{m}}\leq\sum_{j=n}^{m-1}\norm{x_{j}-x_{j+1}}\leq\norm{x_{1}-x_{0}}\sum_{j=n}^{m-1}k^{j}.\tag*{\faLaptopCode}\label{laptopcode}
	\end{equation*}
	Since the geometric series $\sum_{j=1}^{\infty}k^{j}$ is convergent, one has
	\begin{equation*}
		\lim_{n,m\to\infty}\sum_{j=n}^{m-1}k^{j}=0
	\end{equation*}
	and therefore the sequence $\seqn{x}$ is a Cauchy sequence in $X$. Since $X$ is complete, $\seqn{x}$ converges to some limit $\bm{a}\in X$. Moreover, being $k$-contracting, $f$ is continuous so that the sequence $\seqn{f(x)}$ converges to $f(\bm{a})$. Since $x_{n+1}=f\left(x_{n}\right)$ one has $\bm{a}=f(\bm{a})$ and the result is proven.
\end{fancyproof}
Notice tat the above inequality \ref{laptopcode} provides the convergence rate of $\seqn{x}$ to $\bm{a}$. Indeed, taking the limit $m\to\infty$ in \ref{laptopcode} we get
\begin{equation*}
	\norm{x_{n}-\bm{a}}\leq\norm{x_{1}-x_{0}}\sum_{j=n}^{\infty}k^{j}=\frac{k^{n}}{1-k}\norm{x_{1}-x_{0}}s\qquad\every n\in\N.
\end{equation*}
\subsection{Fundamental Properties of Banach spaces}
\begin{definition}
	A normed space $\normsp{X}$ is said to have the Baire property if the intersection of any sequence of dense open sets of $X$ is dense in $X$, i.e. for any $\seqn{U}$ open subsets of $X$ with $\overline{U_{n}}=X$ for any $n\in\N$ it holds
	\begin{equation*}
		\overline{\bigcap_{n}U_{n}}=X.
	\end{equation*}
\end{definition}
The Baire property is clearly equivalent to the following property: the union of any sequence of closed substes with empty interior has an empty interior. In particular, the theorem is often used in the following form: if $\seqn{C}$ is a sequence of closed subsets such that 
\begin{equation*}
	\mathsf{Int}\left(\bigcup_{n\in\N}C_{n}\right)=X
\end{equation*}
Then there exists some $n_{0}\in\N$ such that $\mathsf{Int}(C_{n_{0}})\neq\emptyset$.
\begin{theorem}
	Any complete normed space $\normsp{X}$ has the Baire property.
\end{theorem}
\begin{fancyproof}
	Let $\seqn{U}$ be a sequence of open subsets of $\normsp{X}$ with $\overline{U_{n}}=X$ for any $n\in\N$ and let $A=\bigcup_{n\in\N}U_{n}$. To prove that $A$ is dense in $X$ we shall use the previous lemma and prove that 
	\begin{equation*}
		A\cap\mathcal{O}\neq\emptyset
	\end{equation*}
	for any non empty open subset $\mathcal{O}$ of $X$. Pick any $x_{0}\in\mathcal{O}$ and let $r_{0}>0$ be such that
	\begin{equation*}
		B_{c}(x_{0},r_{0})\subset\mathcal{O}.
	\end{equation*}
	Since $U_{1}$ is dense, one has that $B(x_{0},r_{0})$ is open and non-empty. So we can pick $x_{1}\in B(x_{0},r_{0})\cap U_{1}$ and $r_{1}>0$ such that
	\begin{equation*}
		B_{c}(x_{1},r_{1})\subset B(x_{0},r_{0})\cap U_{1}\qquad0<r_{1}<\frac{r_{0}}{2}.
	\end{equation*}
	We iterate this reasoning and construct a sequence $\seqn{x}\subset{X}$ and $\seqn{r}\subset(0,\infty)$ such that
	\begin{equation*}
		B_{c}(x_{n+1},r_{n+1})\subset B(x_{n},r_{n})\cap U_{n+1}\qquad\text{with }0<r_{n+1}<\frac{r_{n}}{2}\quad\every n\in\N.
	\end{equation*}
	In particular, one sees that $r_{n}<\frac{r_{0}}{2^{n}}$ so that the sequence $\seqn{x}$ is a Cauchy sequence. Since $X$ is complete, there is $x\in X$ such that $\lim_{n}x_{n}=x$. Since $x_{n+k}\in B(x_{n},r_{n})$ for any $n\in\N$ and any $k\in\N$ taking the limit as $k\to\infty$ we get that
	\begin{equation*}
		x\in\overline{B(x_{n},r_{n})}\subset B_{c}(x_{n},r_{n})\qquad\every n\in\N.
	\end{equation*}
	In particular, $x\in\mathcal{O}\cap A$ which is non empty and this proves the result.
\end{fancyproof}
\begin{theorem}
	\emph{Banach-Steinhaus Theorem}. Let $\normspp{X}$ be a Banach space and let $\normspp{Y}$ be a normed space. Let $\left(T_{t}\right)_{i\in I}\subset\mathscr{L}(X,Y)$ be a given collection of continuous linear applications. Assume that, for any $x\in X$, there exists $M_{x}>0$ such that
	\begin{equation*}
		\sup_{i\in I}\norm{T_{i}(x)}_{Y}\leq M_{x}.\tag*{\faWindowClose}\label{winmax}
	\end{equation*}
	Then, there exists $M>0$ such that for any $x\in X$ and any $i\in I$ it holds
	\begin{equation*}
		\norm{T_{i}(x)}_{Y}\leq M\norm{x}_{X}
	\end{equation*}
	i.e. $\sup_{i\in I}\norm{T_{i}}_{\mathscr{L}(X,Y)}\leq M\leq\infty$.
\end{theorem}
The Uniform Boundedness Principle can be reformulated as follows: let $\normspp{X}$ be a Banach space and let $\normspp{Y}$ be a normed space. Let $\F\subset\mathscr{L}(X,Y)$ be a given collection of continuous linear applications (here above $\F={\left(T_{i}\right)}_{i\in I}$). Then the following are equivalent:
\begin{itemize}
	\item Pointwise boundedness: for every $x\in X$, the set $\left\{T(x);T\in\F\right\}$ is bounded in $Y$, i.e.
	\begin{equation*}
		\sup_{T\in\F}\norm{T(x)}_{Y}=M_{x}<\infty\qquad\every x\in X.
	\end{equation*}
	\item Uniform boundedness: the operator norms $\left\{\norm{T}_{\mathscr{L}(X,Y)};T\in\F\right\}$ are bounded, i.e.
	\begin{equation*}
		\sup_{T\in\F}\norm{T}_{\mathscr{L}(X,Y)}=M<\infty.
	\end{equation*}
\end{itemize}
\begin{fancyproof}
	For every $n\in\N$ set
	\begin{equation*}
		X_{n}=\left\{x\in X;\norm{T_{i}(x)}_{Y}\leq n\every i\in I\right\}.
	\end{equation*}
	Since for any $i\in I,T$, $T_{i}$ is continuous, one sees that $X_{n}$ is closed as the intersection of the closed subsets of $X$. Moreover, according to \ref{winmax},
	\begin{equation*}
		X=\bigcup_{n}X_{n}.
	\end{equation*}
	It follows from Baire theorem that there exists $n_{0}\in\N$ such that $\mathsf{Int}\left(X_{n_{0}}\right)\neq\emptyset$. Pick then $x_{0}\in X$ and $r>0$ so that $B(x_{0},r)\subset X_{n_{0}}$. By definition it holds
	\begin{equation*}
		\norm{T_{1}(x_{0}+rz)}_{Y}\leq n_{0}\qquad\every i\in I;\every z\in B(0,1).
	\end{equation*}
	By linearity we get
	\begin{equation*}
		\norm{T_{i}(z)}_{Y}\leq\frac{1}{r}(\norm{T_{i}(x_{0}+r_{z})}_{Y}+\norm{T_{i}(x_{0})}_{Y})\leq 2\frac{n_{0}}{r}\qquad\every i\in I;\every z\in B(0,1).
	\end{equation*}
	This clearly gives the result with $M=2\frac{n_{0}}{r}$.
\end{fancyproof}
\begin{theorem}
	\emph{Open mapping theorem}. Let $\normspp{X}$ and $\normspp{Y}$ be two Banach spaces and let 
	\begin{equation*}
		T:X\to Y
	\end{equation*}
	be a continuous linear application which is surjective (i.e. $T\in\mathscr{L}(X,Y)$ is onto). Then $T$ maps open sets of $X$ into open sets of $Y$, i.e. if $U\subset X$ is an open set then $T(U)$ is an open set of $Y$.
\end{theorem}
\begin{corollary}
	Let $\left(X_{i},\dnorm_{i}\right)$ with $i=1,2$ be two Banach spaces and let $T:X_{1}\to X_{2}$ be a continuous linear application which is bijective. Then its inverse $T_{-1}$ is also continuous, i.e. $T^{-1}\in\mathscr{L}(X_{2},X_{1})$.
\end{corollary}
\begin{proposition}
	Let $\normsp{X}$ be a Banach space and let $\dnorm_{0}$ be a norm on $X$ such that
	\begin{itemize}
		\item $\left(X,\dnorm_{0}\right)$ is a Banach space;
		\item there exists $C_{0}>0$ such that
		\begin{equation*}
			\norm{x}_{0}\leq C_{0}\norm{x}\qquad\every x\in X_{0}.
		\end{equation*}
	\end{itemize}
	Then, the two norms $\dnorm$ and $\dnorm_{0}$ are equivalent.
	\end{proposition}
So if a Banach norm dominates another Banach norm, then they are equivalent.
\begin{theorem}
	\emph{Closed graph theorem}. Let $\left(X_{i},\dnorm_{i}\right)$ with $i=1,2$ be two Banach spaces and let $T:X_{1}\to X_{2}$ be a linear application. Assume that the graph of $T$
	\begin{equation*}
		\mathcal{G}(T)=\left\{(x_{1},T(x_{1}))\in X_{1}\times X_{1}\; x_{1}\in X_{1}\right\}
	\end{equation*}
	is closed in $X_{1}\times X_{2}$ (endowed with the norm $\norm{(x_{1},x_{s})}_{\max}=\max\left\{\norm{x_{1}}_{1},\norm{x_{2}}_{2}\right\}$). Then, 
	\begin{equation*}
		T\in\mathscr{L}(X_{1},X_{2}).
	\end{equation*}
\end{theorem}
\begin{fancyproof}
	Consider on $X_{1}$ the norm
	\begin{equation*}
		\norm{x}_{T}=\norm{x}_{1}+\norm{T(x)}_{2}\qquad x\in X_{1}.
	\end{equation*}
	One checks that $\dnorm_{T}$ is indeed a norm. Moreover, since $\mathcal{G}(T)$ is closed in $X_{1}\times X_{2}$, one can check that $(X_{1},\dnorm_{T})$ is a Banach space. Moreover, one clearly has 
	\begin{equation*}
		\norm{x}_{1}\leq\norm{x}_{T}\qquad\every x\in X_{1}.
	\end{equation*}
	Then, the previous proposition asserts that the norms $\dnorm_{1}$ and $\dnorm_{T}$ are equivalent norms on $X_{1}$, so there exists
	\begin{equation*}
		c>0\quad\text{such that }\norm{x}_{T}\leq x\norm{x}_{1}\qquad\every x\in X_{1}.
	\end{equation*}
	This is enough to prove the continuity of $T$.
\end{fancyproof}
\subsection{Complements}
\begin{definition}
	Let $A\in\mathscr{L}(X)$. We say that $A$ is a compact operator if the image of $B_{c}(0,1)$ through $A$ is contained in a compact set of $X$, i.e. for any sequence $\seqn{\bm{x}}\subset X$ with $\norm{x_{n}}\leq1$, the sequence $\seqnn{A(\bm{x}_{n})}$ admits a subsequence which converges. We shall denote by $\mathscr{K}(X)$ the collection of all compact operators in $X$.
\end{definition}
\begin{definition}
	Let $A\in\mathscr{L}(X)$. We say that $A$ is of finite rank if the image of $A$, $R(A)=\left\{A\bm{x},\bm{x}\in X\right\}$ is a linear subspace of finite dimension $X$.
\end{definition}
\begin{lemma}
	If $A\in\mathscr{K}(X)$ and $B\in\mathscr{L}(X)$ then $B\circ A$ and $A\circ B$ belong to $\mathscr{K}(X)$.
\end{lemma}
\begin{proposition}
	Let $\seqn{A}\subset\mathscr{K}(X)$ and let $A\in\mathscr{L}(X)$ be given with
	\begin{equation*}
		\lim_{n\to\infty}\norm{A_{n}-A}_{\mathscr{L}(X)}=0.
	\end{equation*}
	Then, $A\in\mathscr{K}(X)$.
\end{proposition}
\subsection{More properties of the dual space}
\begin{theorem}
	\emph{Zorn's lemma}. Every non empty ordered set that is inductive has a maximal element.
\end{theorem}
\begin{theorem}
	\emph{Helly, Hahn-Banach analytic form}. Let $X$ be a vector space over $\R$. Let $p:X\to\R$ be a function satisfying
	\begin{enumerate}
		\item $p(\lambda x)=\lambda p(x)$ for any $x\in X$ and any $\lambda>0$\\
		\item $p(x+y)\leq p(x)+p(y)$ for any $x,y\in X$.
	\end{enumerate} 
	Let $Y\subset X$ be a linear subspace and $g:Y\to\R$ a linear function such that
	\begin{enumerate}
		\item[3] $g(x)\leq p(x)$ for any $x\in Y$.  
	\end{enumerate}
	Then, there exists a linear function $f:X\to\R$ such that
	\begin{equation*}
		f(x)=g(x)\qquad\every x\in Y
	\end{equation*}
	(we say then that $f$ extends $g$ to $X$) and
	\begin{equation*}
		f(x)\leq p(x)\qquad\every x\in X.
	\end{equation*}
\end{theorem}
\begin{corollary}
	Let $\normsp{X}$ be a normed space and let $Y\subset X$ be a linear subspace. If $g:Y\to\R$ is a continuous linear mapping, then there exsts $f\in X^{\star}$ that extends $g$ and such that
	\begin{equation*}
		\norm{f}_{\star}=\norm{g}_{Y}
	\end{equation*}
\end{corollary}
\begin{corollary}
	Let $\normsp{X}$ be a normed space. For every $x_{0}\in X$ there exists $\Phi_{0}\in X^{\star}$ such that 
	\begin{equation*}
		\norm{\Phi_{0}}_{\star}=\norm{x_{0}}
	\end{equation*}
	and
	\begin{equation*}
		\Phi_{0}(x_{0})=\norm{x_{0}}^{2}.
	\end{equation*}
\end{corollary}
\section{Inner product spaces and Hilbert spaces}
\subsection{General properties}
\begin{definition}
	Let $H$ be a given vector space. An inner product $\indot$ is a mapping from $H\times H$ with values in $\R$ with the following properties:
	\begin{enumerate}
		\item \emph{symmetry}: $\inprod{x,y}=\inprod{y,x}$ for $\every x,y\in H$;
		\item \emph{bilinearity}: $\inprod{x,\alpha y+\beta z}=\alpha\inprod{x,y}+\beta\inprod{x,z}$ for $\every x,y,z\in H$ and $\every\alpha,\beta\in\R$;
		\item $\inprod{x,x}\geq0$ for $\every x\in H$;
		\item $\inprod{x,y}=0$ if and only if $x=0$. 
	\end{enumerate}
	If $H$ is endowed with an inner product, we say that $\insp{H}$ is an inner product space.
\end{definition}
\begin{proposition}
	If $\insp{H}$ is an inner product space then it is a norm space with respect to the norm given by
	\begin{equation*}
		\norm{x}=\sqrt{\inprod{x,x}}.
	\end{equation*}
	The norm $\dnorm$ is called the norm induced by the inner product $\indot$ and it satisfies
	\begin{equation*}
		\norm{x\pm y}^{2}=\norm{x}^{2}+\norm{y}^{2}\pm 2\inprod{x,y}
	\end{equation*}
	and the parallelogram law
	\begin{equation*}
		\norm{x+y}^{2}+\norm{x-y}^{2}=2\norm{x}^{2}+2\norm{y}^{2}.
	\end{equation*}
\end{proposition}
\begin{proposition}
	\emph{Cauchy-schwartz inequality}. If $\insp{H}$ is an inner product space and $\dnorm$ defines the inner norm then it holds
	\begin{equation*}
		\left|\inprod{x,y}\right|\leq\norm{x}\norm{y}.
	\end{equation*}
\end{proposition}
\begin{proposition}
	Let $\insp{H}$ be an inner product space and let $\seqn{x}$ and $\seqn{y}$ be two converging sequencer (in the inner norm) to $x$ and $y$. Then
	\begin{equation*}
		\lim_{n\to\infty}\inprod{x_{n},y_{n}}=\inprod{x,y}.
	\end{equation*}
\end{proposition}
\subsection{Orthogonality}
\begin{definition}
	Let $\insp{H}$ be a given inner product space. Two vector $x,y\in H$ are said to be \emph{orthogonal} if $\inprod{x,y}=0$. Given two linear subspaces $M,N\subset H$ we say that $M\perp N$ if $\inprod{x,y}=0$ for $\every x\in M,\every y\in N$.
\end{definition}
\begin{definition}
	Let $\insp{H}$ be a inner product space. A finite family $\left\{\bm{e}_{1},\ldots,\bm{e}_{N}\right\}\subset H$ is said to be \emph{orthonormal} if
	\begin{equation*}
		\norm{\bm{e}_{k}}=1\qquad\every k=1,\ldots,N
	\end{equation*}
	and
	\begin{equation*}
		\inprod{\bm{e}_{m},\bm{e}_{n}}=0\qquad\every n\neq m.
	\end{equation*}
\end{definition}
\begin{proposition}
	\emph{Gram-Schmidt Procedure}. Let $\insp{H}$ be a inner product space. Then
	\begin{enumerate}
		\item any orthonormal set $\left\{\bm{e}_{1},\ldots,\bm{e}_{N}\right\}\subset H$ is linearly independent;
		\item given a linearly independent subset $\left\{v_{1},\ldots,v_{N}\right\}$ of $H$ and given $S=\mathrm{span}(v_{1},\ldots,v_{N})$ then there exists an orthonormal basis of $S$ $\left\{\bm{e}_{1},\ldots,\bm{e}_{N}\right\}$.
	\end{enumerate}
\end{proposition}
\begin{fancyproof}
	\begin{enumerate}
		\item We need to show that given $\alpha_{1},\ldots,\alpha_{N}$ such that
		\begin{equation*}
			\sum_{k=1}^{N}\alpha_{k}\bm{e}_{k}=0
		\end{equation*}
		then $\alpha_{j}=0$ for every $j$. Consider any $\bm{e}_{j}$ for $j=1,\ldots,N$. Then we have
		\begin{align*}
			0&=\inprod{\bm{e}_{j},	\sum_{k=1}^{N}\alpha_{k}\bm{e}_{k}}\\
			&=\sum_{k=1}^{N}\alpha_{k}\ubracketthin{\inprod{\bm{e}_{j},\bm{e}_{k}}}_{\mathclap{\text{$=0$ for anything but $\inprod{\bm{e}_{j},\bm{e}_{j}}=\norm{\bm{e}_{j}}=1$}}}\\
			&=\alpha_{j}.
		\end{align*}
		\item We argue by induction over $N$. For $N=1$ it is easy: we have $\left\{v_{1}\right\}$ for which we can set $\bm{e}_{1}=\frac{v_{1}}{\norm{v_{1}}}$. We now know this is true for some integer $N\geq1$. We need to find $\{\bm{e}_{1},\ldots,\bm{e}_{N+1}\}$ which is an orthonormal basis for $\mathrm{span}\left\{v_{1},\ldots,v_{N+1}\right\}$. We know for sure that $v_{N+1}\notin\mathrm{span}(v_{1},\ldots,v_{N})$ because they are linearly independent. Construct
		\begin{equation*}
			h_{N+1}=v_{N+1}-\sum_{k=1}^{N}\inprod{v_{N+1},\bm{e}_{k}}\bm{e}_{k}.
		\end{equation*}
		By construction we have $h_{N+1}\neq0$ and $h_{N+1}\in\mathrm{span}\left\{v_{1},\ldots,v_{N+1}\right\}$. Moreover $\inprod{h_{N+1},\bm{e}_{j}}=0$ for $\every j=1,\ldots, N$. Now we just set 
		\begin{equation*}
			\bm{e}_{N+1}=\frac{h_{N+1}}{\norm{h_{N+1}}}.
		\end{equation*}
		Now we have that $\mathrm{span}(\bm{e}_{1},\ldots,\bm{e}_{N+1})\subset\mathrm{span}(v_{1},\ldots,v_{N+1})$ but since we proved that $\left\{\bm{e}_{1},\ldots,\bm{e}_{N}\right\}$ are linearly independent and since these two spans have the same dimensions, we can say
		\begin{equation*}
			\mathrm{span}(\bm{e}_{1},\ldots,\bm{e}_{N+1})=\mathrm{span}(v_{1},\ldots,v_{N+1}
		\end{equation*}
	\end{enumerate}
\end{fancyproof}
\begin{definition}
	Let $\insp{H}$ be an inner product space and let $M$ be a subset of $H$. Then the \emph{orthogonal complement} of $M$ is defined as
	\begin{equation*}
		M^{\perp}=\left\{x\in H; \inprod{x,u}=0\quad\every u\in M\right\}.
	\end{equation*}
	In particular, one sees that $M^{\perp}\perp M$.
\end{definition}
\begin{proposition}
	Let $\insp{H}$ be an inner product space and let $M\subset H$. Then
	\begin{enumerate}
		\item $0\in M^{\perp}$;
		\item if $0\in M$ then $M\cap M^{\perp}=\left\{0\right\}$, otherwise $M\cap M^{\perp}=\emptyset$;
		\item $\left\{0\right\}^{\perp}=H$ and $H^{\perp}=\{0\}$;
		\item if $M$ is a non-empty subset of $H$ then $M^{\perp}=0$;
		\item if $N\subset M$ then $N^{\perp}\subset M^{\perp}$;
		\item $M^{\perp}$ is a closed linear subspace of $H$;
		\item $M\subset\left(M^{\perp}\right)^{\perp}$.
	\end{enumerate}
\end{proposition}
\begin{proposition}
	Let $M$ be a linear subspace of an inner product space $\insp{H}$. Then
	\begin{equation*}
		x\in M^{\perp}\iff\norm{x-y}\geq\norm{y}\qquad\every y\in M.
	\end{equation*}
\end{proposition}
\subsection{Hilbert Spaces and Projection Theorem}
\begin{definition}
	Let $\insp{H}$ be an inner product space and let $\dnorm$ be the associated inner norm. If $(H,\dnorm)$ is complete then $\insp{H}$ is called a \emph{Hilbert space}.
\end{definition}
\begin{definition}
	Let $X$ be a given vector space. A subset $C\subset X$ is convex if for any $x,y\in C$ one has
	\begin{equation*}
		tx+(1-t)y\in C\qquad\every t\in[0,1].
	\end{equation*}
\end{definition}
Linear subspaces of vector spaces are always convex.
\begin{theorem}
	\emph{Projection over closed subspaces}. Let $(H,\indot)$ be a given Hilbert space and let $K\subset H$ be \underline{closed and convex}.
\end{theorem}
%\newpage
\section{Problem sets}
\subsection{Problem set 1}
\begin{equation*}
	N_{1}(\alpha x,\beta y)=\alpha\beta N_{1}(x,y)
\end{equation*}
\begin{align*}
	N_{1}(x_{1}+x_{2},y_{1}+y_{2})&=\max\left\{\sqrt{(x_{1}+x_{2})^{2}+(y_{1}+y_{2})^{2}},\left|x_{1}+x_{2}-y_{1}-y_{2}\right|\right\}\\
	&\leq\max\left\{\sqrt{x_{1}^{2}+y_{1}^{2}},|x_{1}-y_{1}|\right\}+\max\left\{\sqrt{x_{2}^{2}+y_{2}^{2}},|x_{2}-y_{2}|\right\}
\end{align*}
\begin{align*}
	\int_{-1}^{1}\left|f_{n}(t)-f_{p}(t)\right|\dt&\leq\int_{-1}^{1}\left|f_{n}(t)\right|+\left|f_{p}(t)\right|\dt\\
	&=\int_{-1}^{1}\left|f_{n}(t)\right|\dt+\int_{-1}^{1}\left|f_{p}(t)\right|\dt\\
	&=\norm{f_{n}}_{1}+\norm{f_{p}}_{1}
\end{align*}
\begin{align*}
	L(f+g)&=(f+g)(1)\\
	&=f(1)+g(1)
\end{align*}
\begin{align*}
	L(\lambda f)&=(\lambda f)(1)\\
	&=\lambda f(1)
\end{align*}
\subsubsection{Exercise 5} Let $E=\R^{d}$ be endowed with a norm $\dnorm$. We define the \textit{distance} from an element $x_{0}\in E$ to a subset $A\subset E$ and denote by $d(x_{0},A)$ the quantity
\begin{equation*}
	d(x_{0},A)=\inf_{x\in A}\norm{x-x_{0}}.
\end{equation*}\begin{enumerate}
\item Assume that $A$ is compact. Prove that, for any $x_{0}\in E$ there exists $y\in A$ such that $d(x_{0},A)=\norm{y-x_{0}}$.
\item Prove the result is still true if we only assume $A$ to be closed. One can check that for any $B\subset A$, it holds 
\begin{equation*}
	d(x_{0},B)\geq d(x_{0},A).
\end{equation*}
\item Prove that given $A\subset E$ the mapping $\Phi:x_{0}\in E\mapsto\Phi(x_{0})=d(x_{0},A)$ is continuous on $E$.
\item Deduce that if $A$ is a closed subset of $E$ and $B$ is a compact subset of $E$ with $A\cap B=\emptyset$ then there exists $\delta>0$ such that
\begin{equation*}
	\norm{a-b}\geq\delta\qquad\every(a,b)\in A\times B.
\end{equation*}
\item Provide a counter example showing the result is not true if one only assumes that both $A$ and $B$ are closed and disjoint by considering for instance in dimension $d=2$ the sets
\begin{equation*}
	A=\left\{(x,y);x\leq0\right\},\qquad B:=\left\{\left(\frac{1}{n},n\right);n\in\N\right\}.
\end{equation*}
\end{enumerate}
Solution:
\begin{enumerate}
	\item If we manage to show that $f(x\in A)=\norm{x_{0}-x}$ is a continuous function then we can simply apply the theorem for which if $A$ is compact, the any continuous function assumes its maximum and minimum on $A$. So the minimum of $\norm{x_{0}-x}$, that is $\inf_{x\in A}\norm{x-x_{0}}$, is surely in $A$.
	We know that every norm in $\R^{d}$ is continuous (idk I found it online) we know that this is the case.
\end{enumerate}
\end{document}