% !TeX spellcheck = en_US
\documentclass{article}
\usepackage{pacco}

\title{This Should Help Your Lazy Ass In Analisys B}
\author{Charles A. Mongus, world famous jazz composer and bassist}
\date{\today}

\begin{document}

\maketitle
\tableofcontents
\section{Normed Spaces}
\begin{proposition}
    A subset $C\subset X$ is a closed subset of $(X,\norm{\cdot})$ whenever the limit of any convergent sequence ${(x_n)}_n\subset C$ belongs to $C$. 
\end{proposition}
In other words, $C$ is closed if and only if once a sequence of elements of $C$ is converging, the limit cannot escape $C$.
\begin{fancyproof}
    \begin{enumerate}[\circnum]
        \item \textbf{$C$ closed and ${(x_n)}_n\subset C$ $\implies$ $\lim_nx_n=x\in C$}.\\
        We argue by contradiction that $x\not\in C$: then $x$ must be in $X\setminus C$ which must be open. But if that is an open set then there must exist a ball of $x$ for some $r>0$ such that $B(x,r)\subset X\setminus C$. But due to the definition of convergence there must exist a $N$ such that $\norm{x_n-x}<r\;\every n\geq N$, but then that $x_n$ should be outside $C$ which is a contradiction!
        \item \textbf{$\lim_nx_n=x\in C$ $\implies$ $C$ closed  and ${(x_n)}_n\subset C$}. \\
        We argue by contradiction that $C$ is not closed, so $X\setminus C$ must be not open. Not open sets are such that there exist $x\in X\setminus C$ such that for every $r>0$ you can't find a ball that it is entirely in $X\setminus C$ which means $B(x,r)\cap C\neq\emptyset$. Then pick $x_n\in B(x,\frac{1}{n})\cap C$ so that every $x_n$ also belongs to $C$. Clearly this is a sequence of $n$ such that for any $n\geq1$ we have $\norm{x_n9-x}<\frac{1}{n}$ and this means that $\lim_nx_n=x$. We picked $x\in X\setminus B$ so we have found a limit of a sequence of elements of $C$ that doesn't belong to $C$ which is a contradiction!
    \end{enumerate}
\end{fancyproof}
\begin{lemma}
		\emph{Young's inequality.} Let $p>1$ and let $q<1$ be its conjugate exponent. Then, for any nonnegative $a,b\in\R$ it holds
		\begin{equation*}
			ab\leq\frac{1}{p}a^{p}+\frac{1}{b}b^q.
		\end{equation*}
\end{lemma}
\begin{theorem}
    \emph{Holder Inequality.} Let $1\leq p\leq\infty$ and $\frac{1}{p}+\frac{1}{q}=1$. Assume that $f\in \Lp(S,\mu)$ and $g\in\Lq(S,\mu)$. Then $f\cdot g\in\Lone$ and
    \begin{equation*}
        \norm{fg}_1\leq\norm{f}_p\norm{g}_q.
    \end{equation*}
\end{theorem}
\begin{fancyproof}
    The proof is trivial if $p=1$ and $q=1$. Remember Young's inequality 
    \[
    ab\leq\frac{1}{p}a^p+\frac{1}{q}b^q.
    \]
    Now let's say $a=|f(s)|$ and $b=|g(s)|$. Now our inequality becomes
    \[
    |f(s)g(s)|\leq\frac{1}{p}|f(s)|^p+\frac{1}{q}|g(s)|^q\qquad\maes{S}.
    \]
    Now we integrate over $S$ and we get
    \begin{align*}
        &\int_S|f(s)g(s)|\leq\frac{1}{p}\int_S|f(s)|^p+\frac{1}{q}\int_S|g(s)|^q\\
        \implies&\norm{fg}_1\leq\frac{1}{p}\norm{f}_p^p+\frac{1}{q}\norm{g}_q^q
    \end{align*}
    which means that $\norm{fg}_1$ is finite and therefore $fg\in\Lone(S,\mu)$. To end the proof let's substitute $f$ with $\lambda f,\;\every\lambda>0$. We get
    \begin{align*}
        &\lambda\norm{fg}_1\leq\frac{\lambda^p}{p}\norm{f}_p^p+\frac{1}{q}\norm{g}_q^q\qquad\every\lambda>0\\
        \implies&\frac{\lambda}{\color{red}\lambda}\norm{fg}_1\leq\frac{\lambda^p}{p\color{red}\lambda}\norm{f}_p^p+\frac{1}{q\color{red}\lambda}\norm{g}_q^q\qquad\every\lambda>0\\
        \implies&\norm{fg}_1\leq\frac{\lambda^{p-1}}{p}\norm{f}_p^p+\frac{1}{\lambda q}\norm{g}_q^q\qquad\every\lambda>0.\\
    \end{align*}
    Now choose $\lambda=\frac{1}{\norm{f}_p}\cdot\norm{g}^{\frac{q}{p}}_q$. When we substitute this value for \( \lambda \), we get:
    \[
\frac{\lambda^p}{p} \|f\|_p^p = \frac{\left( \frac{1}{\|f\|_p} \cdot \|g\|_q^{\frac{p}{q}} \right)^p}{p} \|f\|_p^p.
\]
Expanding \( \left( \frac{1}{\|f\|_p} \cdot \|g\|_q^{\frac{p}{q}} \right)^p \), we get:
\[
\left( \frac{1}{\|f\|_p} \cdot \|g\|_q^{\frac{p}{q}} \right)^p = \frac{\|g\|_q^p}{\|f\|_p^p}.
\]

Substituting this, we have:
\[
\frac{\lambda^p}{p} \|f\|_p^p = \frac{\frac{\|g\|_q^p}{\cancel{\|f\|_p^p}}}{p} \cancel{\|f\|_p^p} = \frac{\|g\|_q^p}{p}.
\]

Substitute back into the inequality:
\[
\lambda \|fg\|_1 \leq \frac{\|g\|_q^p}{p} + \frac{1}{q} \|g\|_q^q.
\]

Since \( \frac{\|g\|_q^p}{p} + \frac{\|g\|_q^q}{q} = \|g\|_q^p \left( \frac{1}{p} + \frac{1}{q} \right) \) and \( \frac{1}{p} + \frac{1}{q} = 1 \), we get:
\[
\lambda \|fg\|_1 \leq \|g\|_q^p.
\]


Dividing by \( \lambda = \frac{1}{\|f\|_p} \|g\|_q^{\frac{p}{q}} \):
\[
\|fg\|_1 \leq \|f\|_p \|g\|_q.
\]

This is Hölder’s inequality:
\[
\|fg\|_1 \leq \|f\|_p \|g\|_q.
\]

\end{fancyproof}
\begin{theorem}
	For any $1\leq p\leq\infty$, $\Lp(S,\mu)$ is a vector space and $\norm{\cdot}_{p}$ is a norm.
\end{theorem}
\begin{remark}
	For $1<p<\infty$ the triangular inequality
	\begin{equation*}
		\norm{f+g}_p\leq\norm{f}_p+\norm{g}_p\qquad\every f,g\in\Lp(S,\mu)
	\end{equation*}
	is known as \emph{Minkowski's Inequality}.
\end{remark}
\begin{fancyproof}
	We already know that if $f\in\Lp(S,\mu)$ then $\lambda f\in\Lp(S,\mu)$. Homogeneity and uniqueness are also existent for $\norm{\cdot}_p$ so in order to show that $\Lp(S,\mu)$ is a vector space we only need to prove that if $f,g\in\Lp(S,\mu)$ then $f+g\in\Lp(S,\mu)$ \underline{and} $\norm{\cdot}$ is a norm. \\
	Fix $f,g\in\Lp(S,\mu)$. We know that for any $x,y\in\R$ we get
	\begin{equation*}
		\left|\frac{1}{2}x+\frac{1}{2}y\right|^{p}\leq\frac{1}{2}|x|^p+\frac{1}{2}|y|^{p}
	\end{equation*}
	since this mapping $r\to r^p$ is convex. This also means that
	\begin{equation*}
		\left|x+y\right|^p\leq2^{p-1}\left(|x|^{p}+|y|^{p}\right).
	\end{equation*}
	and this implies in particular that
	\begin{equation*}
		\left|f(s)+g(s)\right|^{p}\leq2^{p-1}\left(|f(s)|^{p}+|g(s)|^{p}\right)\qquad\maes{s\in S}.
	\end{equation*}
	If we integrate over $S$ we get:
	\begin{equation*}
		\int_S\left|f(s)+g(s)\right|^{p}\leq2^{p-1}\left(\int_S|f(s)|^{p}+\int_S|g(s)|^{p}\right)
	\end{equation*}
	which means 
	\begin{equation*}
		\norm{f+g}_{p}^{p}\leq2^{p-1}\left(\norm{f}^p_p+\norm{g}^p_p\right)
	\end{equation*}
	which means that $f+g\in\Lp(S,\mu)$. \par
	We now must prove the Minkowski's inequality. We know that
	\begin{equation*}
			\norm{f+g}_{p}^{p}=\int_S\left|f+g\right|^{p}\dmu=\int_S\left|f+g\right|\left|f+g\right|^{p-1}\dmu
	\end{equation*}
	but since we know that $\left|f+g\right|\leq|f|+|g|$ then
	\begin{equation*}
			\norm{f+g}_{p}^{p}=
	\end{equation*}
\end{fancyproof}
\end{document}
