\documentclass[multi=tikzpicture,varwidth=false]{standalone}
\usepackage{../pacco}
\begin{document}

% Create a function for generating inverse normally distributed numbers using the Box–Muller transform
\pgfmathdeclarefunction{invgauss}{2}{%
  \pgfmathparse{sqrt(-2*ln(#1))*cos(deg(2*pi*#2))}%
}
% Code for brownian motion
\makeatletter
\pgfplotsset{
    table/.cd,
    brownian motion/.style={
        create on use/brown/.style={
            create col/expr accum={
                (\coordindex>0)*(
                    max(
                        min(
                            invgauss(rnd,rnd)*0.1+\pgfmathaccuma,
                            \pgfplots@brownian@max
                        ),
                        \pgfplots@brownian@min
                    )
                ) + (\coordindex<1)*\pgfplots@brownian@start
            }{\pgfplots@brownian@start}
        },
        y=brown, x expr={\coordindex},
        brownian motion/.cd,
        #1,
        /.cd
    },
    brownian motion/.cd,
            min/.store in=\pgfplots@brownian@min,
        min=-inf,
            max/.store in=\pgfplots@brownian@max,
            max=inf,
            start/.store in=\pgfplots@brownian@start,
        start=0
}
\makeatother
%

% Initialise an empty table with a certain number of rows
\pgfplotstablenew{201}\loadedtable % How many steps?



\pgfplotsset{
        no markers,
        xmin=0,
        enlarge x limits=false,
        scaled y ticks=false,
        ymin=-1, ymax=1
}
\tikzset{line join=bevel}
\pgfmathsetseed{13}
\begin{tikzpicture}
\begin{axis}[
                xlabel=\empty,
                x axis line style={->,opacity=100},
                ylabel=\empty,
                axis y line=left,
                ymin=-0.6,ymax=2,
                y axis line style={->,opacity=100},
                ytick=\empty,
                xtick={85,170},
                xticklabels={$t_1$,$t_2$},
                axis x line*=bottom
                   ]     
    \addplot[RedViolet,smooth] table [
        brownian motion={%
        start=0.7,
            max=10,
            min=0
        },
    ] {\loadedtable}node[RedViolet,pos=0.3, above=5pt]{\footnotesize$X(\omega_1)$};
    \addplot[Dandelion,smooth] table [
        brownian motion={%
            start=0.3,
            min=0, max=10
        }
    ] {\loadedtable} node[Dandelion,pos=0.3, below=7pt]{\footnotesize$X(\omega_2)$};
    \draw[|-|,ultra thick](170,-0.15)--(170,0.55) node[pos=0,below]{$A_2$};
    \draw[|-|,ultra thick](85,0.25)--(85,0.95) node[pos=0,below]{$A_1$};
    \draw[dotted] (85,-1)--(85,0.9);
    \draw[dotted] (170,-1)--(170,-0.15);
\end{axis}e
\node[below,Dandelion]at(2.9,-0.5){$\scriptscriptstyle X(\omega_2)\not\in C$};
\node[below,RedViolet]at(5.8,-0.5){$\scriptscriptstyle X(\omega_2)\not\in C$};
\end{tikzpicture}
\end{document}