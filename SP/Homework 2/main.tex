% !TeX spellcheck = en_US
\documentclass[12pt]{article}
\usepackage{amsmath,amsthm,amsfonts,amssymb,amscd}
\usepackage[lining,semibold,scaled=1.05]{ebgaramond}      \usepackage[cmintegrals,cmbraces]{newtxmath}
% Use FUCKING GARAMOND
\usepackage{graphicx}           
% Enhanced support for images
\usepackage{float}              
% Improved interface for floating objects
\usepackage{booktabs}        
\usepackage{pdfpages}   
% Publication quality tables
\usepackage[x11names,table]{xcolor}             
% Driver-independent color extensions
\usepackage[margin=1in]{geometry}           
% Customize document dimensions
\usepackage{fullpage}           
% all 4 margins to be either 1 inch or 1.5 cm
\usepackage{comment}            
% Commenting
\usepackage{mathtools}
\usepackage{minted}             
% Highlighted source code. Syntax highlighting
\usepackage{listings}           
% Typeset programs (programming code) within LaTeX
\usepackage{lastpage}           
% Reference last page for Page N of M type footers.
\usepackage{fancyhdr}           
% Control of page headers and footers
\usepackage{hyperref}           
% Cross-referencing 
\usepackage[small,bf]{caption}  
% Captions
\usepackage{multicol}
\usepackage{cancel} 
% Creating graphic elements
\usepackage{circuitikz}         
% Creating circuits
\usepackage{verbatim}          
% Print exactly what you type in
\usepackage{cite}               
% Citation
\usepackage[us]{datetime} 
% Various time format
\usepackage{blindtext}
% Generate blind text
\usepackage[utf8]{inputenc}
\usepackage{array}
\usepackage{makecell}
\usepackage{tabularx}
\usepackage{titlesec}
\usepackage{bbding}
\usepackage[scr=euler,cal=dutchcal,frak=euler,bb=dsfontserif,bbsymbols]{mathalfa}
\usepackage{bbm}
\usepackage{bm}
%\usepackage{amsmath,amssymb,bbm}
\usepackage{graphicx}
\usepackage[x11names,table]{xcolor}
\usepackage[colorlinks=true, allcolors=DarkOrchid4]{hyperref}
\usepackage{lipsum} 
\usepackage[shortlabels]{enumitem}   
\usepackage{array}
\usepackage{pdfpages}
%\usepackage{tipa} %perchè si stai zitto 
\usepackage{minted}
\usepackage{mleftright}
\usepackage{fancybox}
\usepackage{standalone}
\usepackage{microtype}
\usepackage{csquotes}
\usepackage{titlesec}
\usepackage{forest}
\usepackage{censor}
\usepackage{movement-arrows}
\usepackage{algorithm}
\usepackage{algpseudocode}
\usepackage[makeroom]{cancel}
\usepackage[many]{tcolorbox}
\tcbuselibrary{skins,raster,theorems}
\usepackage{mathtools}
\usepackage{soul}
\usepackage{multicol}
\usepackage{caption}
\usepackage{wrapfig}
\usepackage{authblk}
\usepackage{sectsty,titling}
\usepackage{float}
\usepackage{fontawesome5}
\usepackage{tikz}
\usetikzlibrary{shapes,arrows,positioning,patterns,arrows.meta, bending, shadings,intersections,shadows,decorations,fit,backgrounds,circuits.ee.IEC,shapes.callouts}
\usepackage{pgfplots,pgfplotstable}
\pgfplotsset{compat=1.16}
\usepgfplotslibrary{fillbetween}


\newcounter{dummy}


\theoremstyle{remark}
\newtcbtheorem[number within=section]{rema}{Remark}{
	enhanced, breakable, sharp corners,
	colframe=white, coltitle=black, colbacktitle=gray!24,
	colback=white,
	interior style={top color=LightSteelBlue2, bottom color=white},
	boxed title style={interior style={left color=LightSteelBlue2,right color=white!35}, sharpish corners,
		frame style={right color=LightSteelBlue2,left color=white!35},drop fuzzy shadow=gray},
	attach boxed title to top left,
	drop fuzzy shadow=gray,
	boxed title style={boxrule=0.5mm, sharp corners},
	separator sign={:},
	description formatter=\newline,
	theorem hanging indent/.try=0pt,
}{}

\newenvironment{remark}[1][]{
	\ifstrempty{#1}{%   
		\begin{rema*}{}{}%                     
		}{%        
			\begin{rema*}{}{}%                 
			}%
		}{%
		\end{rema*}% 
	}
	
	
	\newminted{python}{mathescape, linenos, numbersep=5pt, frame=single, breaklines,bgcolor=gray!9, fontfamily=courier,python3,rulecolor=red!39!black,numbersep=5pt,label=Python,xleftmargin=0pt}
	\newenvironment{py}{\VerbatimEnvironment\begin{pythoncode}}{\end{pythoncode}}
	
	\newcounter{defi}
	
	\newtcbtheorem[number within=section]{nition}{Definition}{
		enhanced, 
		breakable, sharp corners,
		interior style={left color=SkyBlue4!60,right color=white!29},
		title style={left color=white,right color=white!35},
		% boxed title style={boxrule=5mm, colframe=black,sharp corners},
		colframe=white, coltitle=SkyBlue4!60!black, colbacktitle=blue!34,
		colback=white,
		titlerule=0mm,
		drop fuzzy shadow=gray,
		borderline={0.5mm}{0mm}{SkyBlue4},
		separator sign={:},
		description formatter=\newline,
		theorem hanging indent/.try=0pt,
	}{}
	
	\newenvironment{definition}[1][]{
		\stepcounter{defi}
		\ifstrempty{#1}{%   
			\begin{nition}{}{\thedefi\thedummy}%                     
			}{%        
				\begin{nition}{}{\thedefi eheh}%                 
				}%
			}{%
			\end{nition}% 
		}{\addvspace{\baselineskip}}
		
		\newcounter{theo}
		
		\newtcbtheorem[number within=section]{thrm}{Theorem}{
			enhanced, 
			breakable,
			sharp corners,
			interior style={left color=PaleTurquoise3!60,right color=Turquoise4!50},
			title style={right color=LemonChiffon1,left color=Aquamarine4!90!black},
			frame style={right color=SpringGreen4,left color=Turquoise4},
			colframe=white, coltitle=white, colbacktitle=SkyBlue4,
			colback=SkyBlue4,
			drop fuzzy shadow=Cyan4!50!black,
			titlerule=0mm,
			separator sign={:},
			description formatter=\newline,
			theorem hanging indent/.try=0pt,
		}{}
		
		\newenvironment{theorem}[1][]{ 
			\stepcounter{theo}
			\stepcounter{dummy}
			\ifstrempty{#1}{%   
				\begin{thrm}{}{\thetheo,\thedummy}%                     
				}{%        
					\begin{thrm}{}{\thetheo aa}%                 
					}%
				}{%
				\end{thrm}% 
			}
			
			\newtcolorbox{corollary}{
				enhanced, 
				breakable,
				sharp corners,
				interior style={left color=Thistle1,right color=LightSteelBlue4!50},
				title style={right color=LightSteelBlue4,left color=Thistle1},
				frame style={right color=LightSteelBlue4,left color=Thistle1},
				colframe=white, coltitle=LightSteelBlue4, colbacktitle=SkyBlue4,
				colback=SkyBlue4,
				drop fuzzy shadow=Cyan4!50!black,
				titlerule=0mm,
				separator sign={:},
				description formatter=\newline,
				title={Corollary}
			}{}
			
			%lemma lemma lemmaaa
			\newcounter{lem}
			
			\newtcbtheorem[number within=section]{lemm}{Lemma}{
				enhanced, 
				breakable,
				interior style={left color=SkyBlue4!20,right color=Cyan4!10},
				title style={left color=SkyBlue4!70,right color=Cyan4!10},
				frame style={left color=SkyBlue4!70,right color=Cyan4!10},
				colframe=white, coltitle=white, colbacktitle=SkyBlue4,
				colback=SkyBlue4,
				drop fuzzy shadow=gray,
				titlerule=0mm,
				separator sign={:},
				description formatter=\newline,
				theorem hanging indent/.try=0pt,
			}{}
			
			\newenvironment{lemma}[1][]{ 
				\stepcounter{lem}
				\stepcounter{dummy}
				\ifstrempty{#1}{%   
					\begin{lemm}{}{\thetheo,\thedummy}%                     
					}{%        
						\begin{lemm}{}{\thetheo aa}%                 
						}%
					}{%
					\end{lemm}% 
				}
				
				
				
				\newcounter{propos}
				%now making the proposition
				\newtcbtheorem[number within=section]{prp}{Proposition}{
					enhanced, 
					breakable,
					sharp corners,
					interior style={left color=RoyalBlue4!40!Aquamarine1,right color=RoyalBlue4!40},
					frame style={left color=RoyalBlue4!80!Aquamarine1,right color=RoyalBlue4},
					colframe=white, coltitle=RoyalBlue4, title style={left color=white!60, right color=RoyalBlue4!80},
					colback=Maroon1,
					drop fuzzy shadow=gray,
					titlerule=0mm,
					separator sign={:},
					description formatter=\newline,
					theorem hanging indent/.try=0pt,
				}{}
				
				\newenvironment{proposition}[1][]{ 
					\ifstrempty{#1}
					{%   
						\begin{prp}{}{\thepropos\thedummy eeeheheh\thedefi}%                     
						}{%        
							\hfill
							\begin{prp}{}{\thetheo,\thepropos ahah}%                 
							}%
						}{%
						\end{prp}% 
					}
					
					
					\newenvironment{diagram}[1][]{
						\ifstrempty{#1}
						{%   
							\begin{tikzpicture}[circuit ee IEC,x=2cm,y=0.8cm,semithick,
								every info/.style={font=\small},
								set resistor graphic=var resistor IEC graphic,
								set diode graphic=var diode IEC graphic,
								set make contact graphic= var make contact IEC graphic]
							}{%        
								\begin{tikzpicture}[circuit ee IEC,x=2cm,y=0.8cm,semithick,
									every info/.style={font=\small},
									set resistor graphic=var resistor IEC graphic,
									set diode graphic=var diode IEC graphic,
									set make contact graphic= var make contact IEC graphic]  
								}%
							}{
							\end{tikzpicture}
						}
						
						
						
						\newcounter{algor}
						%now making the proposition
						\newtcbtheorem{algo}{Complexity analysis}{
							enhanced, 
							breakable,
							interior style={left color=red!30,right color=white!29},
							title style={left color=red!30,right color=red!80},
							colframe=white, coltitle=black, colbacktitle=red!60,
							colback=red!80!black,
							titlerule=0mm,
							separator sign={:},
							description formatter=\newline,
							theorem hanging indent/.try=0pt,
						}{}
						
						\newenvironment{complexity}[1][]{
							{\par\noindent\ignorespaces}
							{\ignorespacesafterend} 
							\hfill   
							\stepcounter{algor}
							\stepcounter{dummy}
							\ifstrempty{#1}
							{%   
								\hfill
								\begin{algo}{}{\thealgor\thedummy eeeheheh\thedefi}%                     
								}{%        
									\hfill
									\begin{algo}{}{\thetheo,\thealgor ahah\thedummy}%                 
									}%
								}{%
								\end{algo}% 
							}
							
							
							\newcounter{exampl}
							
							\newtcbtheorem[number within=section]{exa}{Example}{
								enhanced jigsaw, 
								frame empty, breakable,
								colframe=white, coltitle=black,
								colback=white,
								opacityback=0,
								sharp corners,
								borderline west={1mm}{0mm}{CadetBlue4},
								attach boxed title to top left,  
								boxed title style={boxrule=0.5mm,drop fuzzy shadow=gray, colframe=CadetBlue4, sharp corners, interior style={left color=white, right color=CadetBlue4!50}},
								separator sign={:},
								opacityframe=.5,
								description formatter=\newline,
								theorem hanging indent/.try=0pt,
							}{}
							
							\newenvironment{example}[1][]{
								\stepcounter{exampl}
								\ifstrempty{#1}{%   
									\begin{exa}{}{\theexer\theexampl}%                     
									}{%       
										\begin{exa}{}{\theexampl}%                 
										}%
									}{%
									\end{exa}% 
								}
								
								\newcounter{exer}
								\newlength{\hatchspread}
								\newlength{\hatchthickness}
								\newlength{\hatchshift}
								\newcommand{\hatchcolor}{}
								% declaring the keys in tikz
								\tikzset{hatchspread/.code={\setlength{\hatchspread}{#1}},
									hatchthickness/.code={\setlength{\hatchthickness}{#1}},
									hatchshift/.code={\setlength{\hatchshift}{#1}},% must be >= 0
									hatchcolor/.code={\renewcommand{\hatchcolor}{#1}}}
								% setting the default values
								\tikzset{hatchspread=3pt,
									hatchthickness=0.4pt,
									hatchshift=0pt,% must be >= 0
									hatchcolor=black}
								% declaring the pattern
								\pgfdeclarepatternformonly[\hatchspread,\hatchthickness,\hatchshift,\hatchcolor]% variables
								{custom north east lines}% name
								{\pgfqpoint{\dimexpr-2\hatchthickness}{\dimexpr-2\hatchthickness}}% lower left corner
								{\pgfqpoint{\dimexpr\hatchspread+2\hatchthickness}{\dimexpr\hatchspread+2\hatchthickness}}% upper right corner
								{\pgfqpoint{\dimexpr\hatchspread}{\dimexpr\hatchspread}}% tile size
								{% shape description
									\pgfsetlinewidth{\hatchthickness}
									\pgfpathmoveto{\pgfqpoint{0pt}{\dimexpr\hatchspread+\hatchshift}}
									\pgfpathlineto{\pgfqpoint{\dimexpr\hatchspread+0.15pt+\hatchshift}{-0.15pt}}
									\ifdim \hatchshift > 0pt
									\pgfpathmoveto{\pgfqpoint{0pt}{\hatchshift}}
									\pgfpathlineto{\pgfqpoint{\dimexpr0.15pt+\hatchshift}{-0.15pt}}
									\fi
									\pgfsetstrokecolor{\hatchcolor}
									%    \pgfsetdash{{1pt}{1pt}}{0pt}% dashing cannot work correctly in all situation this way
									\pgfusepath{stroke}
								}
								\newcounter{exercise}[subsection]
								\newtcolorbox[use counter=exercise]{exercise}{
									enhanced standard jigsaw,
									sharp corners,
									breakable=false,
									interior titled code={
										\path[
										draw=white,
										sharp corners,
										pattern=custom north east lines,
										hatchspread=12pt,
										hatchthickness=4pt,
										hatchcolor=SkyBlue1!10,
										]
										(interior.south east) rectangle (interior.north west);
									},
									title={Exercise~\thetcbcounter},
									coltitle=black,
									fonttitle=\bfseries,
									colbacktitle=SkyBlue1!20}
								
								\newtcolorbox{revise}{
									enhanced jigsaw,
									colback=red!5!white,sharp corners, colframe=LightSteelBlue4!75!black,
									breakable,
									drop shadow,
									fonttitle=\bfseries,
									interior style={left color=CadetBlue3!30,right color=white!29},
									frame style={left color=DodgerBlue4,right color=DodgerBlue1},
									title={Revise with Kotatsu!}}
								\newtcolorbox{insult}{
									enhanced jigsaw,
									colback=red!5!white,sharp corners, colframe=red,
									breakable,
									drop shadow={red!80!black},
									fonttitle=\bfseries,
									interior style={left color=red!70,right color=white!29},
									title={You fucking prick}}
								
								\newtcolorbox{notation}{
									enhanced jigsaw,
									opacityback=0,
									colback=white, colframe=Aquamarine4!80!SteelBlue2,
									attach boxed title to top left ={yshift=-\tcboxedtitleheight/2,yshifttext=-\tcboxedtitleheight/2,xshift=0.3cm},
									boxed title style={%
										colback=Aquamarine4!80!SteelBlue2,
										opacityback=1
									},
									boxrule=0.5mm,top=0mm,bottom=0mm,
									rounded corners,
									coltitle=white,
									drop shadow,
									fonttitle=\bfseries,
									title={Note the notation!}}	
								
								
								
								\newtcolorbox{proofbox}{
									enhanced,
									attach boxed title to top left,
									colback=red!5!white,sharp corners, colframe=white,
									breakable,
									drop shadow,
									boxed title style={interior style={fill overzoom image=scrapped},sharp corners},
									coltitle=black,
									fonttitle=\bfseries\itshape,
									interior style={fill tile image*={width=1.1\linewidth}{squares}},
									frame style={white},
									title={Proof}}	
								
								\newenvironment{fancyproof}[1][]{
									\ifstrempty{#1}{%   
										\begin{proofbox} \begin{proof}[\unskip\nopunct]                   
											}{%        
												\begin{proofbox} \begin{proof}[\unskip\nopunct]%                 
													}%
												}{%
											\end{proof}\end{proofbox}% 
										}					
										\let\oldemptyset\emptyset
										\let\emptyset\varnothing
										
										
										\renewcommand{\labelitemi}{$\textcolor{SteelBlue4}{\bullet}$}
										\renewcommand{\labelitemii}{$\textcolor{blue}{\cdot}$}
										\renewcommand{\labelitemiii}{$\textcolor{SkyBlue4}{\diamond}$}
										\renewcommand{\labelitemiv}{$\textcolor{SkyBlue4}{\ast}$}
										
										\newcommand{\enf}[1]{\textcolor{DodgerBlue3}{\textbf{#1}}} %enf sta per enfasi
										\let\emph\enf
										\newcommand{\sott}[1]{\setulcolor{SkyBlue3}\ul{#1}}
										\let\underline\sott
										
										\newcommand{\prob}{\mathbb{P}}
										\newcommand\independent{\protect\mathpalette{\protect\independenT}{\perp}}
										\newcommand{\ev}{\mathbb{E}}
										\def\independenT#1#2{\mathrel{\rlap{$#1#2$}\mkern2mu{#1#2}}}
										\newcommand{\Z}{\mathbb{Z}}
										\newcommand{\R}{\mathbb{R}}
										\newcommand{\N}{\mathbb{N}}
										\newcommand{\T}{\mathbb{T}}
										\newcommand{\Tbar}{\overline{\T}}
										\newcommand{\A}{\mathscr{A}}
										\newcommand{\Nstar}{\N^{*}}
										\newcommand{\Rext}{\overline{\R}}
										\newcommand{\io}{\text{ i.o.}}
										\newcommand{\normale}{\mathcal{N}}
										\newcommand{\equalexpl}[1]{%
											\underset{\substack{\uparrow\\\mathrlap{\text{\vspace{-3cm}\hspace{-1em}#1}}}}{=}}
										\newcommand{\dif}{\mathop{}\!\mathrm{d}}
										\newcommand{\convas}{\xrightarrow[]{\text{a.s.}}}
										\newcommand{\convpr}{\xrightarrow[]{\pr}}
										\newcommand{\convd}{\xrightarrow[]{\text{d}}}
										\newcommand{\convlp}{\xrightarrow[]{\lp}}
										\newcommand{\lone}{L^{1}}
										\newcommand{\convw}{\xrightarrow[]{\mathrm{weak}}}
										\newcommand{\as}{\text{ a.s.}}
										\newcommand{\asstnr}{\sim N(0,1)}
										%\def\checkmark{\tikz\fill[scale=0.4](0,.35) -- (.25,0) -- (1,.7) -- (.25,.15) -- cycle;} 
										\newcommand{\dx}{\dif x}
										\newcommand{\dy}{\dif y}
										\newcommand{\dt}{\dif t}
										\newcommand{\du}{\dif u}
										\newcommand{\ds}{\dif s}
										\newcommand{\dw}{\dif\omega}
										\newcommand{\dz}{\dif z}
										\newcommand{\dmu}{\dif\mu}
										\newcommand{\dpr}{\dif\pr}
										\newcommand{\E}{\mathscr{E}}
										\newcommand{\B}{\mathscr{B}}
										\newcommand{\F}{\mathscr{F}}
										\newcommand{\G}{\mathscr{G}}
										\newcommand{\HS}{\mathscr{H}}
										\newcommand{\Zn}{\mathscr{Z}}
										\newcommand{\D}{\mathscr{D}}
										\newcommand{\xbar}{\overline{X}}
										\newcommand{\rbar}{\overline{\R}}
										\newcommand{\ybar}{\overline{Y}}
										\newcommand{\xxbar}{\overline{\xbar}}
										\newcommand{\xtilde}{\widetilde{X}}
										\newcommand{\ifonly}{\underline{if and only if}}
										
										\newcommand{\ubracketthin}[1]{\underbracket[0.3pt]{#1}}
										
										\newcommand*\circled[1]{\tikz[baseline=(char.base)]{
												\node[shape=circle,draw,
												shading=ball, ball color=SkyBlue1!70,
												,inner sep=1.5pt] (char) {\scriptsize\bfseries #1};}}
										\newcommand{\circnum}{\protect\circled{\arabic*}}
										\newcommand{\circlet}{\protect\circled{\alph*}}	
										\newcommand{\bbg}[1]{%
											\ooalign{$#1$\cr\raisebox{-.2pt}{$#1$}\cr\raisebox{.2pt}{$#1$}\cr\textcolor{white}{$\mkern0.2mu#1$}}%
										}
										\newcommand{\leb}{\bbg{\lambda}}
										
										%i comandi di MERDA di stefano
										\newcommand{\ra}{\rightarrow}
										\newcommand{\iy}{\infty}
										\newcommand{\mE}{\mathbb{E}}
										\newcommand{\mP}{\mathbb{P}}
										\newcommand{\mB}{\mathcal{B}(\mathbb{R})}
										\newcommand{\mL}{\mathbb{L}}
										\newcommand{\ninfty}{n\to$\infty$}
										\newcommand{\tinfty}{t\to$\infty$}
										\newcommand{\xinfty}{x\to$\infty$}
										\newcommand{\yinfty}{y\to$\infty$}
										\newcommand{\mLL}{\mathbb{L}^2(0,T)}
										\newcommand{\mR}{\mathbb{R}}
										\newcommand{\mC}{\mathcal{C}}
										\newcommand{\mRd}{\mathbb{R}^d}
										\newcommand{\mN}{\mathbb{N}}
										%\newcommand{\m1}{\mathbf{1}}
										\newcommand{\mF}{\mathcal{F}}
										\newcommand{\mM}{\mathcal{M}}
										\newcommand{\mW}{\mathcal{W}}
										\newcommand{\mV}{\mathcal{V}}
										\newcommand{\mA}{\mathcal{A}}
										\newcommand{\mez}{\frac{1}{2}}
										\newcommand{\intT}{\int_{0}^{T}}
										\newcommand{\nN}{n \in \mathbb{N}}
										\newcommand{\mmN}{\mathcal{N}}
										\newcommand{\BM}{(B_t)_{t\geq 0}}
										\newcommand{\PX}{(X_t)_{t\geq 0}}
										\newcommand{\var}{\mathbb{V}\!\mathrm{ar}\,}
										\newcommand{\cov}{\mathbb{C}\!\mathrm{ov}\,}
										\newcommand{\lp}{L^{p}}
										\newcommand{\BMF}{\left((B_t)_{t\geq 0},\mathcal{F}_t \right)}
										\newcommand{\mFt}{\mathcal{F}_t}
										\newcommand{\every}{\forall\,}
										\newcommand{\dyadic}{\mathcal{d}}
										\newcommand{\norm}[1]{\left|\negthinspace\left|#1\right|\negthinspace\right|}
										\newcommand{\pr}{\mathbb{P}}
										\newcommand{\indi}{\mathbb{1}}
										\newcommand{\trns}[1]{{#1}^{\scriptscriptstyle\mathsf{T}}}
										\newcommand{\iid}{\stackrel{\mathrm{iid}}{\sim}}
										\newcommand{\unmezz}{\frac{1}{2}}
										\newcommand{\sa}{$\sigma$-algebra}	\newcommand{\rv}{random variable}
										\titleformat{\section}{\color{SkyBlue4}}{\color{red}\thesection}{1em}{}
										\subsectionfont{\color{SkyBlue4!90!black}\bfseries}
										\pretitle{\begin{center}\LARGE\color{SkyBlue2}}
											\sectionfont{\color{Cyan2!50!black}\bfseries}
											\pretitle{\begin{center}\LARGE\color{SkyBlue2}}
												\newcommand{\prefacename}{Preface}
												\newcommand{\cinlar}{Çinlar }
												\newenvironment{preface}{
													\vspace*{\stretch{2}}
													{\noindent \bfseries \huge\color{SkyBlue4} \prefacename}
													\begin{center}
														% \phantomsection \addcontentsline{toc}{chapter}{\prefacename} % enable this if you want to put the preface in the table of contents
														\thispagestyle{plain}
													\end{center}%
												}
												{\vspace*{\stretch{5}}}
												
												
												\newenvironment{closethedeal}{
													\vspace*{\stretch{2}}
													{\noindent \bfseries \huge\color{SkyBlue4} That's all, folks}
													\begin{center}
														% \phantomsection \addcontentsline{toc}{chapter}{\prefacename} % enable this if you want to put the preface in the table of contents
														\thispagestyle{plain}
													\end{center}%
												}
												{\vspace*{\stretch{5}}}
												\pgfkeys{%
													/calloutquote/.cd,
													width/.code                   =  {\def\calloutquotewidth{#1}},
													position/.code                =  {\def\calloutquotepos{#1}}, 
													author/.code                  =  {\def\calloutquoteauthor{#1}},
													/calloutquote/.unknown/.code   =  {\let\searchname=\pgfkeyscurrentname
														\pgfkeysalso{\searchname/.try=#1,                                
															/tikz/\searchname/.retry=#1},\pgfkeysalso{\searchname/.try=#1,
															/pgf/\searchname/.retry=#1}}
												}  
												
												
												\newcommand\calloutquote[2][]{%
													\pgfkeys{/calloutquote/.cd,
														width               = 5cm,
														position            = {(0,-1)},
														author              = {}}
													\pgfqkeys{/calloutquote}{#1}                   
													\node [rectangle callout,draw,callout relative pointer={\calloutquotepos},text width=\calloutquotewidth,/calloutquote/.cd,
													#1] (tmpcall) at (0,0) {#2};
													\node at (tmpcall.pointer){\calloutquoteauthor};    
												}  		
												
												
												\newcommand\lemmethink[2][]{%
													\pgfkeys{/calloutquote/.cd,
														width               = 5cm,
														position            = {(0,-1)},
														author              = {}}
													\pgfqkeys{/calloutquote}{#1}                   
													\node [cloud callout,draw,callout relative pointer={\calloutquotepos},text width=\calloutquotewidth,/calloutquote/.cd,
													#1] (tmpcall) at (0,0) {#2};
													\node at (tmpcall.pointer){\calloutquoteauthor};    
												}  											
												
												
												\renewcommand\leq\leqslant
												\renewcommand\geq\geqslant			
												
												\usepackage[
												backend=biber,
												style=authoryear,
												sorting=nyt
												]{biblatex}			\posttitle{\par\end{center}\vskip 0.5em}

\setlength\parindent{0pt}

%%%%%%%%%%%%%%%%%%%%%%%%%%%%%%%%%%%%%%%%%%%%%%%%%%%%%%%%%%%%%%
\titleformat{\section}
{\color{UM_DarkBlue}\normalfont\large\bfseries}
{\color{blue!80!black}\thesection}{1em}{}

%%%%%%%%%%%%%%%%%%%%%%%%%%%%%%%%%%%%%%%%%%%%%%%%%%%%%%%%%%%%%%
\hypersetup{
    draft=false,
    final=true,
    colorlinks=true,
    citecolor=UM_DarkBlue,
    anchorcolor=yellow,
    linkcolor=UM_DarkBlue,
    urlcolor=UM_DarkBlue,
    filecolor=green,      
    pdfpagemode=FullScreen,
    bookmarksopen=false
    }
    
%%%%%%%%%%%%%%%%%%%%%%%%%%%%%%%%%%%%%%%%%%%%%%%%%%%%%%%%%%%%%%
\lstdefinestyle{Fortran}{
basicstyle=\scriptsize,        % the size of the fonts that are used for the code
  breakatwhitespace=false,         % sets if automatic breaks should only happen at whitespace
  breaklines=false,                 % sets automatic line breaking
  captionpos=b,                    % sets the caption-position to bottom
  commentstyle=\color{mygreen},    % comment style
  extendedchars=true,              % lets you use non-ASCII characters; for 8-bits encodings only, does not work with UTF-8
  keepspaces=true,                 % keeps spaces in text, useful for keeping indentation of code (possibly needs columns=flexible)
  keywordstyle=\color{blue},       % keyword style
  language=[95]Fortran,                 % the language of the code
  numbers=left,                    % where to put the line-numbers; possible values are (none, left, right)
  numbersep=5pt,                   % how far the line-numbers are from the code
  numberstyle=\tiny\color{mygray}, % the style that is used for the line-numbers
  rulecolor=\color{black},         % if not set, the frame-color may be changed on line-breaks within not-black text (e.g. comments (green here))
  showspaces=false,                % show spaces everywhere adding particular underscores; it overrides 'showstringspaces'
  showstringspaces=false,          % underline spaces within strings only
  showtabs=false,                  % show tabs within strings adding particular underscores
  stepnumber=1,                    % the step between two line-numbers. If it's 1, each line will be numbered
  stringstyle=\color{mymauve},     % string literal style
  tabsize=4,                       % sets default tabsize to 2 spaces
  title=\lstname                   % show the filename of files
}

%%%%%%%%%%%%%%%%%%%%%%%%%%%%%%%%%%%%%%%%%%%%%%%%%%%%%%%%%%%%%%%
\definecolor{UM_Brown}{HTML}{3D190D}
\definecolor{UM_DarkBlue}{HTML}{2264B0}
\definecolor{UM_LightBlue}{HTML}{1CA9E1}
\definecolor{UM_Orange}{HTML}{fEB415}





\tikzstyle{startstop} = [rectangle, rounded corners, minimum width=3cm, minimum height=1cm,text centered, draw=black, fill=red!30]
\tikzstyle{io} = [trapezium, trapezium left angle=70, trapezium right angle=110, minimum width=3cm, minimum height=1cm, text centered, draw=black, fill=blue!30]
\tikzstyle{process} = [rectangle, minimum width=3cm, minimum height=1cm, text centered, draw=black, fill=orange!30]
\tikzstyle{decision} = [diamond, minimum width=3cm, minimum height=1cm, text centered, draw=black, fill=green!30]
\tikzstyle{arrow} = [thick,->,>=stealth]

\begin{document}
	\textcolor{UM_Brown}{
		\begin{center}
			\textbf{\Large Stochastic Processes}\\
			\vspace{5pt}
			Homework 2 \\
			\vspace{5pt}
			\textbf{M.S. in Stochastics and Data Science}\\
			\vspace{20pt}
			\textit{Niccolò Disassonia} 863128 \\
			\textit{Varaga Haghoubians} 1088805\\
			\textit{Lorenzo Sala} 943481\\
			\vspace{5pt}
			\today
		\end{center}
		\vspace{10pt}
		\hrule
	}
	
	
	
	
	%%%%%%%%%%%%%%% NEW SECTION %%%%%%%%%%%%%%% 
	\section*{Exercise 1}
	Consider the one-dimensional diffusion process $\{X(t)\}$ defined on the interval $(0,\infty)$ with drift $\mu(x)=\frac{1}{2x}$ and infinitesimal variance $\sigma^{2}(x)=1$. Let $a=0$ and $b=\infty$ be the boundaries of the state space.
	\begin{enumerate}
		\item Compute the \emph{scale function} $s(x)$ up to a constant.
		\item Compute the \emph{speed measure} $m(x)$ up to a constant.
		\item Determine the \emph{nature of boundaries} $0$ and $\infty$. For each boundary decide whether it is: 
		\begin{itemize}
			\item regular;
			\item exit;
			\item entrance;
			\item natural.
		\end{itemize}
		\item Interpret the classification in terms of the behavior of the process near $0$ and $\infty$. Is the process \emph{recurrent} or \emph{transient}?
	\end{enumerate}
	\subsection*{Solution}
	\begin{enumerate}
		\item We know that the scale function (or measure) $S(x)$ for a process $x$ is defined as
		\begin{equation*}
			S(x)=\int^{x}s(\eta)\dif\eta=\int^{x}\exp\left\{-\int^{\eta}\frac{2\mu(\xi)}{\sigma^{2}(\xi)}\dif\xi\right\}\dif\eta.
		\end{equation*} 
		We start by computing the scale \textit{density} $s(\eta)$:
		\begin{align*}
			s(\eta)&=\exp\left\{-\int^{\eta}\frac{2\mu(\xi)}{\sigma^{2}(\xi)}\dif\xi\right\}\\
			&=\exp\left\{\int^{\eta}\frac{2\cdot\frac{1}{2\eta}}{1}\dif\xi\right\}\\
			&=\exp\left\{-\int^{\eta}\frac{1}{\xi}\dif\xi\right\}\\
			&=\exp\left\{-\ln(\eta)\right\}=\frac{1}{\eta}.
		\end{align*}
		We can now plug this into the formula for $S(x)$ so that we get
		\begin{equation*}
			S(x)=\int^{x}\frac{1}{\eta}\dif\eta=\ln |x|+c
		\end{equation*}
		where $c\in\R$ is an additive constant.
		\item We know that the speed measure $M(x)$ for a process $x$ is
		\begin{equation*}
			M(x)=\int^{x}m(y)\dy=\int^{x}\frac{1}{\sigma^{2}(x)s(x)}\dy.
		\end{equation*} 
		We need to compute the speed \textit{density} $m(y)$:
		\begin{align*}
			m(y)&=\frac{1}{\sigma^{2}(y)s(y)}\\
			&=\frac{1}{1\cdot\frac{1}{y}}=y.
		\end{align*}
		So by plugging this result into the formula for $M(x)$ we get:
		\begin{equation*}
			M(x)=\int^{x}y\dy=\frac{1}{2}x^{2}+c
		\end{equation*}
		where $c\in\R$ is an additive constant.
		\item Recall the scheme for the classification of boundaries (in this case, for the lower boundary $a$):
		\begin{table}[H]
			\centering
			\resizebox{1.02\textwidth}{!}{
				\begin{tabular}{|>{\columncolor[gray]{0.8}}c>{\columncolor[gray]{0.8}}c>{\columncolor[gray]{0.8}}c>{\columncolor[gray]{0.8}}c|c|c|c|c|}
					\hline
					\multicolumn{4}{|c|}{\textbf{Criteria}} &\multicolumn{4}{c|}{\textbf{Terminology}}\\
					\hline\rowcolor{red}
					$S(a,x]$&$M(a,x]$&$\Sigma(a)$&$N(a)$&\textbf{Feller} & \multicolumn{3}{c|}{\textbf{Gikhman and Skorokhod}} \\
					\hline
					$<\infty$ & $<\infty$ & $<\infty$ & $<\infty$ & \multicolumn{2}{c|}{Regular}& \multirow{3}{*}{Attracting}&\multirow{2}{*}{Attainable}\\ \cline{1-6}
					$<\infty$ & $=\infty$ & $<\infty$ & $=\infty$&\multicolumn{2}{c|}{Exit-Trap-Absorbing}& &\\ \cline{1-6}\cline{8-8}
					$<\infty$ & $=\infty$ & $=\infty$ & $=\infty$ &\multirow{3}{*}{\makecell{Natural\\ $(\Sigma(l)=\infty,N(l)=\infty)$}}&\makecell{Attracting,\\ unattainable}& &\multirow{4}{*}{Unattainable}\\ \cline{6-7}\cline{1-4}
					$=\infty$ & $<\infty$ & $=\infty$ & $=\infty$&&\multirow{3}{*}{\makecell{Natural\\$(S(l,x]=\infty$}}&\multirow{3}{*}{Nonattracting}&\\ \cline{1-4}
					$=\infty$ &	$=\infty$ &	$=\infty$ &	$=\infty$ &&&&\\\cline{5-5}\cline{1-4}
					$=\infty$ &	$<\infty$ &	$=\infty$ &	$<\infty$ &Entrance&&&\\\hline
			\end{tabular}}
		\end{table}
		In our case, $a=0$ so we are left with the task of computing 
		\begin{equation*}
			S(0,x]\quad M(0,x]\quad \Sigma(0) \quad N(0).
		\end{equation*}
		\begin{itemize}
			\item	We can compute $M(0,x]$straight away:
			\begin{align*}
				M(0,x]&=\int_{0}^{x}y\dy=\unmezz y^{2}\bigg|^{x}_{0}=\unmezz x^{2}<\infty.
			\end{align*}
			\item On the other hand, $S(0,x]$ presents itself as an improper integral which does not converge (since we cannot compute $\ln0$ directly):
			\begin{align*}
				S(0,x]&=\int_{0}^{x}\frac{1}{\eta}\dif\eta\\
				&=\lim_{\varepsilon\to0^{+}}\int_{\varepsilon}^{x}\frac{1}{\eta}\dif\eta\\
				&=\lim_{\varepsilon\to0^{+}}\left(\ln x-\ln \varepsilon\right)=\infty.
			\end{align*}
			\item We now have to compute $\Sigma(0)$. We defined the function $\Sigma(l)$ as
			\begin{align*}
				\Sigma(l)&=\lim_{a\searrow l}\int_{a}^{x}S[a,\xi]\dif M(\xi)\\
				&=\int_{l}^{x}M[\eta,x]\dif S(\eta).
			\end{align*}
			We know that
			\begin{equation*}
				M[\eta,x]=\int_{\eta}^{x}y\dy=\unmezz y^{2}\bigg|^{x}_{\eta}=\unmezz \left(x^{2}-\eta^{2}\right).
			\end{equation*}
			We also know that $\dif S(\eta)=s(\eta)\dif\eta$ and since we already know that $s(\eta)=\frac{1}{\eta}$
			\begin{equation*}
				\dif S(\eta)=s(\eta)\dif\eta=\frac{1}{\eta}\dif\eta.
			\end{equation*}
			Putting all together in the formula for $\Sigma(l)$ with $l=0$ we get
			\begin{align*}
				\Sigma(0)&=\int_{0}^{x}\left(\unmezz\left(x^{2}-\eta^{2}\right)\right)\cdot\frac{1}{\eta}\dif\eta\\
				&=\unmezz\int_{0}^{x}\left(x^{2}-\eta^{2}\right)\cdot\frac{1}{\eta}\dif\eta\\
				&=\unmezz\biggl[x^{2}\cdot\ubracketthin{\int^{x}_{0}\frac{1}{\eta}\dif\eta}_{=\infty}-\ubracketthin{\int_{0}^{x}\eta\dif\eta}_{\unmezz x^{2}<\infty}\biggr]
			\end{align*}
			Since the first term of the subtraction is infinite, we have that
			\begin{equation*}
				\Sigma(0)=\infty.
			\end{equation*}
			\item We defined $N(l)$ as
			\begin{align*}
				N(l)&=\int_{l}^{x}S[\eta,x]\dif M(x)\\
				&=\int_{l}^{x}M(l,\xi]\dif S(\xi).
			\end{align*}
			We already know that, for $l=0$ then
			\begin{equation*}
				M(0,\xi]=\int_{0}^{\xi}m(y)\dy=\int_{0}^{\xi}y\dy=\frac{\xi^{2}}{2}
			\end{equation*}
			and
			\begin{equation*}
				\dif S(\xi)=\frac{1}{\xi}\dif\xi.
			\end{equation*}
			So putting all together we get
			\begin{align*}
				N(0)&=\int_{0}^{x}\frac{\xi^{2}}{2}\cdot\frac{1}{\xi}\dif\xi\\
				&=\unmezz\int_{0}^{x}\xi\dif\xi\\
				&=\unmezz\cdot\frac{x^{2}}{2}=\frac{x^{2}}{4}<\infty\qquad \every x\in\R.
			\end{align*}
		\end{itemize}
		So our situation for the boundary $a=0$ is
		\begin{equation*}
			S(0,x]=\infty\quad M(0,x]<\infty\quad \Sigma(0)=\infty \quad N(0)<\infty.
		\end{equation*}
		So, according to the Feller terminology this boundary is:
		\begin{itemize}[$\diamond$]
			\item an entrance boundary (or a ``natural boundary'' according to the Skorokhod terminology);
			\item non-attracting;
			\item unattainable.
		\end{itemize}
		We now have to do the same exact procedure for the upper boundary $b=\infty$ for which we have to compute
		\begin{equation*}
			S(x,\infty]\quad M(x,\infty]\quad \Sigma(\infty) \quad N(\infty).
		\end{equation*}
		\begin{itemize}
			\item  We know that 
			\begin{align*}
				S(x,\infty)&=\int_{x}^{\infty}\frac{1}{\eta}\dif\eta\\
				&=\lim_{\varepsilon\to\infty}\int_{x}^{b}\frac{1}{\eta}\dif\eta\\
				&=\lim_{\varepsilon\to\infty}\left(\ln b-\ln x\right)=\infty.
			\end{align*}
			\item We know that 
			\begin{equation*}
				M(x,\infty]=\int_{x}^{\infty}y\dif y=\unmezz y^{2}\bigg|^{\infty}_{x}=\infty.
			\end{equation*}
			\item We know that
			\begin{equation*}
				\Sigma(\infty)=\int_{x}^{\infty}M[x,\xi]\dif S(\xi)
			\end{equation*}
			and since $M[x,\xi]=\int_{x}^{\xi}\eta\dif\eta=\frac{\xi^{2}-x^{2}}{2}$ and $\dif S(\xi)=\frac{1}{\xi}\dif\xi$ we get that
			\begin{align*}
				\Sigma(\infty)&=\int_{x}^{\infty}\frac{\xi^{2}-x^{2}}{2}\cdot\frac{1}{\xi}\dif\xi\\
				&=\unmezz\int_{x}^{\infty}\left(\xi-\frac{x^{2}}{\xi}\right)\dif\xi\\
				&=\unmezz\ubracketthin{\int_{x}^{\infty}\xi\dif\xi}_{=\infty}-x^{2}\ubracketthin{\int_{x}^{\infty}\frac{1}{\xi}\dif\xi}_{=\infty}.
			\end{align*}
			Since $\int_{x}^{\infty}\frac{1}{\xi}\dif\xi$ grows logarithmically and this it is slower than $\int_{x}^{\infty}\xi\dif\xi$ then we have
			\begin{equation*}
				\Sigma(\infty)=\infty.
			\end{equation*}
			\item For this problem it is easier to use the alternative definition of $N(l)$. We know that
			\begin{equation*}
				N(\infty)=\int_{x}^{\infty}S[x,\xi]\dif M(\xi).
			\end{equation*}
			Since $\dif M(\xi)=m(\xi)\dif\xi$ we can write
			\begin{equation*}
				N(\infty)=\int_{x}^{\infty}S[x,\xi]\cdot m(\xi)\dif\xi.
			\end{equation*}
			Moreover, we know that $$S[x,\xi]=\int_{x}^{\xi}\frac{1}{\eta}\dif\eta=\ln\xi-\ln x$$ and that $m(\xi)=\xi$ so we can write
			\begin{align*}
				N(\infty)&=\int_{x}^{\infty}\xi(\ln\xi-\ln x)\dif\xi\\
				&=\int_{x}^{\infty}\xi\ln\xi\dif\xi-\ln x\int_{x}^{\infty}\xi\dif\xi.
			\end{align*}
			We want to analyze the behavior of both $\int_{x}^{\infty}\xi\ln\xi\dif\xi$ and $\int_{x}^{\infty}\xi\dif\xi$.
			\begin{itemize}
				\item In $\int_{x}^{\infty}\xi\ln\xi\dif\xi$ we have $\xi$ that grows faster than $\ln\xi$, so $\xi\ln\xi\xrightarrow{\xi\to\infty}\infty$ so the integral diverges to $\infty$.
				\item In $\int_{x}^{\infty}\xi\dif\xi$ we have
				\begin{equation*}
					\int_{x}^{\infty}\xi\dif\xi=\frac{\xi^{2}}{2}\bigg|_{x}^{\infty}
				\end{equation*}
				but since $\frac{\xi^{2}}{2}\xrightarrow{\xi\to\infty}\infty$ then the integral diverges to $\infty$.
			\end{itemize}
			So we have
			\begin{equation*}
				N(\infty)=\infty-\ln x\cdot\infty=\infty
			\end{equation*}
			since $\ln x$ is finite.
		\end{itemize}
		So our situation for the boundary $b=\infty$ is
		\begin{equation*}
			S(0,x]=\infty\quad M(0,x]=\infty\quad \Sigma(0)=\infty \quad N(0)=\infty.
		\end{equation*}
		So, according to the Feller terminology this boundary is:
		\begin{itemize}[$\diamond$]
			\item a natural boundary (according to both terminologies);
			\item non-attracting;
			\item unattainable.
		\end{itemize}
		\item We know that $0$ is an entrance boundary: this means that the process cannot ``enter'' 0 but if it starts from 0 then it can exit and drift away. On the other hand $\infty$ is a natural boundary, so the process can neither start from it or reach it in finite time from the interior of the diffusion interval. 
		So this means that the process will drift away from 0 and will never return to it. This, together with the fact that $\infty$ is unattainable and non-attracting, makes the process \emph{transient.}
	\end{enumerate}
	\section*{Exercise 2}
	Show that:
	\begin{enumerate}
		\item the equilibrium distribution for the Wiener process between two reflecting barriers is a truncated exponential if the drift is non-zero and a uniform distribution if the drift is zero;
		\item the equilibrium distribution for the Ornstein-Uhlenbeck process between two reflecting barriers is a truncated normal distribution;
		\item $\{X(t)\}$ is not a Markov process when $\{X(t)\}$ is the displacement process of a particle whose velocity $U(t)$ follows the Ornstein-Uhlenbeck process.
	\end{enumerate}
	
	\subsection*{Solution}
	\begin{enumerate}
		\item If Brownian motion has a real drift and is locked between two reflective barriers, its overall conduct (the distribution of the equilibrium) depends on the drift. \par
		We know that the stationary distribution $\psi(x)$ satisfies the Fokker-Planck equation
		\begin{equation*}
			0=\unmezz\frac{\partial^{2}}{\partial y^{2}}\left[\sigma^{2}(y)\psi(y)\right]-\frac{\partial}{\partial y}\left[\mu(y)\psi(y)\right]
		\end{equation*} which can be solved to yield
		\begin{equation*}
			\psi(x)=m(x)\left[c_{1}S(x)+c_{2}\right].
		\end{equation*}
		Let's analyze the two different cases, when $\mu(x)=0$ and when $\mu(x)\neq0$.
		\begin{itemize}
			\item \emph{No drift}. We know that
				\begin{align*}
				m(y)=\frac{1}{\sigma^{2}(y)s(y)}
			\end{align*}
			and in this case
			\begin{align*}
				s(y)&=\exp\left\{-\int^{y}\frac{2\mu(\xi)}{\sigma^{2}(\xi)}\dif\xi\right\}\\
				&=\exp\left\{-\int^{y}\frac{2\cdot 0}{1}\dif\xi\right\}\\
				&=e^{0}=1
			\end{align*}
			so we get
			\begin{equation*}
				\mu(y)=1.
			\end{equation*}
			We can also calculate $S(x):$
			\begin{align*}
					S(x)&=\int^{x}s(\eta)\dif\eta\\
					&=\int^{x}1\dif\eta=x.
			\end{align*}
			This means that the solution to the Fokker-Planck equation becomes
			\begin{equation*}
				\psi(x)=c_{1}\cdot x+c_{2}.
			\end{equation*}
			To determine the values of $c_{1}$ and $c_{2}$ we must take into account the conditions on the stationary distribution:
			\begin{equation*}
				\begin{cases}
					\psi(x)\geq0\\
					\int_{I}\psi(x)\dx=1.
				\end{cases}
			\end{equation*}
		For the conditions of non-negativity to hold, we try setting $c_{1}=0\implies\psi(x)=c_{2}$.
		Knowing that the Wiener process between two reflecting boundaries $a$ and $b$ has diffusion interval $I=(a,b)$, the total probability condition becomes
		\begin{align*}
			&\int_{I}\psi(x)\dx=1\\
			\implies&\int_{a}^{b}c_{2}\dx=1\\
			\implies&c_{2}(b-a)=1\\
			\implies& c_{2}=\frac{1}{b-a}.
		\end{align*}
		So what we get is 
		\begin{equation*}
			\psi(x)=\frac{1}{b-a}\qquad x\in[a,b]
		\end{equation*}
		which is exactly the uniform distribution we were looking for.
		\item \emph{With drift}. Since we are talking about a Wiener process, we have $\mu(x)=\mu$ constant. Here the values for $m(x)$ and $S(x)$ change and so does the solution to the Fokker-Planck equation. We have
		\begin{align*}
			s(y)&=\exp\left\{-\int^{y}\frac{2\mu}{1}\dif\xi\right\}\\
			&\exp\left\{-2\mu\int^{y}\dif\xi\right\}=e^{-2\mu x}
		\end{align*}
		so our speed density becomes
		\begin{equation*}
			m(y)=\frac{1}{1\cdot e^{-2\mu x}}=e^{2\mu x}.
		\end{equation*}
		The scale measure $S(x)$ becomes
		\begin{align*}
			S(x)&=\int^{x}e^{-2\mu\eta}\dif\eta.
		\end{align*}
		If we integrate from $-\infty$ then this integral is not well-defined for $\mu>0$ but in this case we know that the lower bound is our reflecting boundary $a$. So we get
		\begin{equation*}
			S(x)=\frac{e^{-2\mu\eta}}{-2\mu}\bigg|_{a}^{x}=\frac{1}{2\mu}\left(e^{-2\mu x}-e^{-2\mu a}\right).
		\end{equation*}
		This means that the solution to the Fokker-Planck equation becomes
		\begin{align*}
			\psi(x)&=e^{2\mu x}\left[c_{1}\frac{1}{2\mu}\left(e^{-2\mu x}-e^{-2\mu a}\right)+c_{2}\right]\\
			&=e^{2\mu x}\left[\frac{c_{1}}{2\mu}\left(e^{-\mu x}-e^{-\mu a}\right)+c_{2}\right]\\
			&=\frac{c_{1}}{2\mu}\left(e^{2\mu x}e^{-2\mu x}-e^{2\mu x}e^{2\mu a}\right)+c_{2}e^{2\mu x}\\
			&=\frac{c_{1}}{2\mu}\left(1-e^{2\mu(x-a)}\right) +c_{2}e^{2\mu x}.
		\end{align*}
		As before, we try setting $c_1=0$ so that the equilibrium distribution becomes
		\begin{equation*}
			\psi(x)=c_2e^{2\mu x}
		\end{equation*}
		and so that the normalization condition becomes
		\begin{align*}
			\int_{a}^{b}\psi(x)\dx=\int_{a}^{b}c_{2}e^{2\mu x}\dx=1.
		\end{align*}
		So we get
		\begin{align*}
			c_2\int_{a}^{b}e^{2\mu x}\dx&=\frac{c_2}{2\mu}\left(e^{2\mu b}-e^{2\mu a}\right)=1 \qquad\text{this is okay because }\mu\neq0\\
			\implies&c_2=\frac{2\mu}{e^{2\mu b}-e^{2\mu a}}.
		\end{align*}
		This gives us
		\begin{equation*}
			\psi(x)=\frac{2\mu\cdot e^{2\mu x}}{e^{2\mu b}-e^{2\mu a}},\qquad x\in[a,b],\mu \neq0
		\end{equation*}
		which is a truncated exponential distribution with parameter $\lambda=-2\mu$.
	\end{itemize}
	\item For the Ornstein-Uhlenbeck process we have
	\begin{equation*}
		\begin{cases}
			\mu(x)=-\frac{x}{\vartheta}\\
			\sigma^{2}(x)=\sigma^{2}.	\end{cases}
	\end{equation*}
	We need to perform the calculations. We have that
	\begin{align*}
		s(y)&=\exp\left\{-\int^{y}\frac{-2\frac{\xi}{\vartheta}}{\sigma^{2}}\dif\xi\right\}\\
		&=\exp\left\{\int^{y}\frac{2\xi}{\vartheta\sigma^{2}}\dif\xi\right\}\\
		&=\exp\left\{\frac{1}{\vartheta\sigma^{2}}\int^{y}2\xi\dif\xi\right\}\\
		&=\exp\left\{\frac{y^{2}}{\vartheta\sigma^{2}}\right\}.
	\end{align*}
	We get that
	\begin{equation*}
		m(y)=\frac{1}{\sigma^{2}\exp\left\{\frac{y^{2}}{\vartheta\sigma^{2}}\right\}}=\frac{1}{\sigma^{2}}\exp\left\{-\frac{y^{2}}{\vartheta\sigma^{2}}\right\}.
	\end{equation*}
	Moving on to $S(x)$:
	\begin{align*}
		S(x)&=\int^{x}\exp\left\{\frac{\eta^{2}}{\vartheta\sigma^{2}}\right\}\dif\eta.
	\end{align*}
	This is not solvable in closed form... Anyway, we get that our stationary distribution satisfies
	\begin{align*}
		\psi(x)&=m(x)\left[c_{1}S(x)+c_{2}\right]\\
		&=\frac{1}{\sigma^{2}}\exp\left\{-\frac{x^{2}}{\vartheta\sigma^{2}}\right\}\left[c_1\int^{x}\exp\left\{\frac{\eta^{2}}{\vartheta\sigma^{2}}\right\}\dif\eta+c_2\right]\\
		&=c_1\frac{1}{\sigma^{2}}\exp\left\{-\frac{x^{2}}{\vartheta\sigma^{2}}\right\}\int^{x}\exp\left\{\frac{\eta^{2}}{\vartheta\sigma^{2}}\right\}\dif\eta+c_2\frac{1}{\sigma}\exp\left\{-\frac{x^{2}}{\vartheta\sigma^{2}}\right\}.
	\end{align*}
	We gladly set $c_1=0$ to get
	\begin{equation*}
		\psi(x)=c_2\frac{1}{\sigma^{2}}\exp\left\{-\frac{x^{2}}{\vartheta\sigma^{2}}\right\}.
	\end{equation*}
	We use the normalization constraint (with respect to the diffusion interval with boundaries $[a,b]$) to obtain
	\begin{align*}
		&\int_{a}^{b}c_2\frac{1}{\sigma^{2}}\exp\left\{-\frac{x^{2}}{\vartheta\sigma^{2}}\right\}\dx=1\\
		\implies&c_2\frac{1}{\sigma^{2}}\int_{a}^{b}\exp\left\{-\frac{x^{2}}{\vartheta\sigma^{2}}\right\}\dx=1\\
		\implies&c_2=\frac{\sigma^{2}}{\int_{a}^{b}\exp\left\{-\frac{x^{2}}{\vartheta\sigma^{2}}\right\}\dx}.
	\end{align*}
	If we plug this back into the solution of the Fokker-Planck equation and we get
	\begin{equation*}
		\psi(x)=\frac{\exp\left\{-\frac{x^{2}}{\vartheta\sigma^{2}}\right\}}{\int_{a}^{b}\exp\left\{-\frac{x^{2}}{\vartheta\sigma^{2}}\right\}\dx}\qquad x\in[a,b]
	\end{equation*}
	which is the density function of truncated Gaussian distribution at $a$ and $b$.
	
\item If $X(t)$ is the displacement process with velocity $U(t)$ then we have that
\begin{equation*}
	X(t)=\int_{0}^{t}U(y)\dy.
\end{equation*}
We need to show that 
\begin{equation*}
	\pr(X(t+s)\in A|\F_{t})\neq\pr(X(t+s)\in A|X(t)).
\end{equation*}
An alternative definition of Markov property that can be found in \cite[64]{rene} is that any $d$-dimensional $\F_{t}$ adapted stochastic process ${\{X_{t}\}}_{t\geq0}$ with right continuous paths is a Markov process if it satisfies
\begin{equation*}
	\ev\left[u(X(t+s))\mid\F_{t}\right]=\ev\left[u(X(t+s)\mid X(t))\right]
\end{equation*}
for $\every s,t\geq0, u\in\mathcal{B}(\R^{d})$.
We can write $X(t+s)$ as 
\begin{equation*}
	X(t+s)=X(t)+\int_{t}^{t+s}U(y)\dy.
\end{equation*}
This means that 
\begin{align*}
	\ev\left[X(t+s)|\F_{t}\right]&=\ev\left[\left.X(t)+\int_{t}^{t+s}U(y)\dy\right|\F_{t}\right]\\
	&=\ubracketthin{X(t)}_{\mathclap{\in  \F_{t}}}+\ev\left[\left.\int_{t}^{t+s}U(y)\dy\right|\F_{t}\right]
\end{align*}
and we know that Ornstein-Uhlenbeck processes are Markov, therefore
\begin{equation*}
	\ev[U(y)|\F_{t}]=\ev[U(y)|U(t)]\qquad\text{for }y>t.
\end{equation*}
So by linearity of expectation and Fubini's theorem we have
\begin{equation*}
	\ev\left[X(t+s)|\F_{t}\right]=X(t)+\int_{t}^{t+s}\ev\left[U(y)|U(t)\right]\dy.
\end{equation*}
We now want to compute $\ev\left[u(X(t+s)\mid X(t))\right]$. Again, we start from the ``split'' version of $X(t+s)$.
\begin{align*}
	\ev\left[X(t+s)\mid X(s)\right]&=\ev\left[\left.X(t)+\int_{t}^{t+s}U(y)\dy\right|X(t)\right]\\
	&=X(t)+\ev\left[\left.\int_{t}^{t+s}U(y\dy)\right|X(t)\right]\\
	&=X(t)+\int_{t}^{t+s}\ev\left[U(y)|X(t)\right]\dy.
\end{align*}
But now we cannot say that $\ev\left[U(y)|X(t)\right]=\ev\left[U(y)|U(t)\right]$ as before, so we got that 
\begin{equation*}
	\ev\left[u(X(t+s))\mid\F_{t}\right]\neq\ev\left[u(X(t+s)\mid X(t))\right]
\end{equation*}
And so the process $X(t)$ is \textit{not} Markov. This makes sense intuitively because $X$, being the integral of $U$, ``smooths out'' the velocity. This makes the information generated by $X$ at time $t$ insufficient to know about the displacement at time $t+s$.
	\end{enumerate}
	
	\section*{Exercise 3}
	Consider a variant ${\{S_{n}\}}_{n\leq0}$ of the classical random walk such that $S_{n}=\sum_{i=0}^{n}X_{i}$ but $\{X_{i}\}$ is a sequence i.i.d. random variables with $\pr(X_{i}=1)=\pr(X_{i}=-1)=\frac{1}{4}$ and $\pr(X_{i}=0)=\frac{1}{2}$.
	\begin{enumerate}
		\item Determine the diffusion limit of this random walk.
		\item Determine its diffusion interval and the nature of its boundaries.
		\item Does this process admit stationary distribution? Justify the answer.
	\end{enumerate}
	\subsection*{Solution}
	\begin{enumerate}
		\item By diffusion limit here we mean investigating the distribution of ${\{S_{n}\}}_{n\leq0}$ as the step size tends to 0. We want to apply the Central Limit Theorem and for this we need the mean and the variance of $X_i$.
		We know that
		\begin{equation*}
			\ev\left[{X_{i}}\right]=1\cdot\frac{1}{4}+(-1)\cdot\frac{1}{4}+0\cdot\frac{1}{2}=0
		\end{equation*}
		and that
		\begin{align*}
			\var(X_{i})&=\ev\left[X_{i}^{2}\right]-\ev\left[X_{i}\right]^{2}\\
			&=\left(1^{2}\cdot\frac{1}{4}+(-1)^{2}\cdot\frac{1}{4}+0^{2}\cdot\unmezz\right)-0^{2}\\
			&=\unmezz.
		\end{align*}
		If we turn to the variance of $S_{n}$ we see that, due to the fact that the $X_{i}$'s are i.i.d., 
		\begin{equation*}
			\var(S_{n})=\sum_{i=0}^{n}\var(X_{i})=(n+1)\cdot\frac{1}{2}.
		\end{equation*}
		This is not something we should be surprised of: variances have an additive nature in sums, so we could say that, having a step in time (that in this case is 1 unit of time between each step of the random walk),
		\begin{equation*}
			\mathrm{variance}\propto n.
		\end{equation*}
		In diffusion processes we have that the time step is not fixed as it is here, but it is continuous. This is problematic because it causes the variance to explode as $n\to\infty$ if it is directly proportional to the number of time steps. The variance must be proportional to the time step so that changing the size of the step $\Delta t$ doesn't change the behavior of the process.
		\begin{equation*}
			\mathrm{variance}\propto \mathrm{time~step}.
		\end{equation*}
		Each time step we will basically add a fixed variance $\sigma^{2}$ to the random step length: after $n$ steps we will have $n\sigma^{2}$ cumulative variance; we want this variance to be the same when we take a single time step of length $n\Delta t$ and this corresponds to scaling by $\sqrt{n}$ so that the variance does not explode when letting $n\to\infty$. For this purpose we define the scaled random walk
		\begin{equation*}
			S^{\star}_{(n)}(t)=\frac{1}{\sqrt{n}}\sum_{i=1}^{nt}X_{i}.
		\end{equation*}
		Here the random walk is rescaled by \textit{space} by a factor of $\frac{1}{\sqrt{n}}$ and by \textit{time} by ``fitting'' $nt$ steps of the random walk in a time $t$. This means that instead of 1 unit of time per step, we make each step take time $\Delta t=\frac{1}{n}$. The effect of this is that as $n\to\infty$ the steps will become smaller (as effect of the space rescaling) and they will happen ever more often in a time $t$ (by effect of the time rescaling). \par
		We can now apply the CLT. We know that since $\ev\left[X_{i}\right]=0$ then
		\begin{equation*}
			\ev\left[S^{\star}_{(n)}(t)\right]=0
		\end{equation*}
		and since $\var(X_{i})=\unmezz$ then
		\begin{align*}
			\var\left(S^{\star}_{(n)}(t)\right)&=\frac{1}{n}\cdot\sum_{i=1}^{nt}\var(X_{i})\\
			&=\frac{1}{\cancel{n}}\cdot \cancel{n}t\cdot\unmezz=\frac{t}{2}.
		\end{align*}
		This means that by means of the CLT
		\begin{equation*}
			S^{\star}_{(n)}(t)\convd\mathsf{N}\left(0,\frac{t}{2}\right)\qquad\text{as }n\to\infty
		\end{equation*}
		and this hints at the fact that the limiting process may be a Brownian motion with diffusion coefficient $\sigma^{2}=\unmezz$. We can check the 4 properties of Brownian motion:
		\begin{enumerate}[\circnum]
			\item initial condition:
			\begin{equation*}
				S^{\star}_{(n)}(0)=0
			\end{equation*}
			for every $n$, even when $n\to\infty$;
			\item independent increments: this is true since the random walk had independent increments;
			\item stationary and normally distributed increments: for $t\leq s$ we have
			\begin{equation*}
				S^{\star}_{(n)}(t)-S^{\star}_{(n)}(s)=\frac{1}{\sqrt{(n)}}\sum_{i=ns+1}^{nt}X_{i}\xrightarrow[n\to\infty]{d}\mathsf{N}\left(0,\frac{t-s}{2}\right)
			\end{equation*}
			since this is always a sum of $(t-s)n$ i.i.d. \rv s $X_{i}$ with variance $\unmezz$;
			\item continuity of paths: by Donsker's theorem\footcite[198]{rene} we know that, if the increments of the random walk have zero mean and finite variance, then 
			\begin{equation*}
				S^{\star}_{(n)}(t)\xrightarrow[n\to\infty]{d}B_{t}\qquad\every t\in(0,1].
			\end{equation*}
			This convergence gives pointwise but not uniform results: to get the latter, we need the \textit{invariance principle} by Donsker\footcite[199]{rene}: if $\Phi:\mathcal{C}([0,1],\R)\to\R$ (the space of continuous real-valued functions defined on $[0,1]$) is a uniformly continuous bounded functional then
			\begin{equation*}
				\lim_{n\to\infty}\ev\left[\Phi\left(S^{\star}_{(n)}(\cdot)\right)\right]=\ev\left[\Phi(B(\cdot))\right].
			\end{equation*}
			This implies that the limit process lives in $\mathcal{C}([0,1],\R)$ and therefore the limit (which is the Brownian motion) has continuous sample paths almost surely.
			\[
			S^{\star}_{(n)}(\cdot) \Rightarrow B(\cdot)
			\quad \text{in } \mathcal{C}([0,1], \mathbb{R}).
			\]
		\end{enumerate}
		\item The state space of the random walk is $\mathbb{Z}$ so the diffusion interval is $(-\infty,+\infty)$. To classify these borders we need to compute, as before, the four quantities of interest. In this case $\mu(x)=0$ and $\sigma^{2}=\unmezz$.
		\begin{enumerate}[i)]
			\item \textit{scale measure} $S(x)$:
			\begin{align*}
				S(x)&=\int^{x}_{-\infty}\exp\left\{-\int^{\eta}\frac{0}{\unmezz}\dif\xi\right\}\dif\eta\\
				&=\int^{x}_{-\infty}\exp\left\{0\right\}\dif\eta\\
				&=\int^{x}_{-\infty}1\dif\eta=\infty.
			\end{align*}
			\item \textit{speed measure} $M(x)$:
			\begin{align*}
				M(x)&=\int_{-\infty}^{x}\frac{1}{\sigma^{2}(y)\cdot s(y)}\dy\\
				&=\int_{-\infty}^{x}\frac{1}{\unmezz\cdot 1}\dy\\
				&=\int_{-\infty}^{x}2\dy=\infty.
			\end{align*}
		\end{enumerate}
		We already have that $S(-\infty,x)=S(x,+\infty)=\infty$ and $M(-\infty,x)=M(x,+\infty)=\infty$. Since we defined
		\begin{equation*}
			\Sigma(l)=\int_{l}^{x}M[\eta,x]\dif S(\eta)\quad\mathrm{and}\quad N(l)=\int_{l}^{x}M(l,\xi]\dif S(\xi)
		\end{equation*}
		then we will have also that $\Sigma(-\infty)=\Sigma(\infty)=\infty$ and $N(-\infty)=N(\infty)=\infty$. According to our table, this means that both boundaries will be 
		\begin{itemize}
			\item natural;
			\item non-attracting;
			\item unattainable.
		\end{itemize}
		\item A natural boundary is not accessible from the interior of the state space and it is neither reflecting or absorbing. We know that an hypothetical stationary distribution $\psi(x)$ should satisfy
		\begin{equation*}
			\psi(x)=m(x)\left[c_{1}S(x)+c_{2}\right]
		\end{equation*}
		and that if we cannot find the constants $c_{1}$ and $c_{2}$ such that $\psi(x)$ is a density then the stationary distribution doesn't exist. In our case we have $m(x)=2$ and $\left[c_{1}S(x)+c_{2}\right]=\left[c_{1}\cdot\infty+c_{2}\right]$ so we will never find constants such that $\psi(x)$ integrates to 1! Even choosing $c_{1}=0$ we have
		\begin{equation*}
			\int_{I}\psi(x)\dx=\int_{-\infty}^{\infty}2c_{2}\dx=\infty.
		\end{equation*}
		This makes sense because the process can move arbitrarily in any direction (there is no drift) and it will never return with certainty to any compact region.
	\end{enumerate}
	\printbibliography
\end{document}
