\usepackage[T1]{fontenc}
\usepackage[utf8]{inputenc}
\usepackage{amsmath,amsthm,amsfonts,amssymb,amscd}
\usepackage[lining,semibold,scaled=1.05]{ebgaramond}     
%\usepackage[cmintegrals,cmbraces]{newtxmath}
% Use FUCKING GARAMOND
\usepackage[ebgaramond]{newtxmath}
\usepackage{graphicx}        
\usepackage{fancyvrb}    
% Enhanced support for images
\usepackage{float}              
% Improved interface for floating objects
\usepackage{booktabs}        
\usepackage{pdfpages}   
% Publication quality tables          
% Driver-independent color extensions
\usepackage[margin=1in]{geometry}           
% Customize document dimensions
\usepackage{fullpage}           
% all 4 margins to be either 1 inch or 1.5 cm
\usepackage{comment}      
% Commenting
\usepackage{mathtools}       
% Highlighted source code. Syntax highlighting
\usepackage{listings}           
% Typeset programs (programming code) within LaTeX
\usepackage{lastpage}           
% Reference last page for Page N of M type footers.
\usepackage{fancyhdr}                  
% Cross-referencing 
\usepackage[small,bf]{caption}  
% Captions
\usepackage{multicol} 
% Creating graphic elements
\usepackage{circuitikz}         
% Creating circuits        
% Print exactly what you type in
\usepackage{cite}               
% Citation
\usepackage[us]{datetime} 
% Various time format
\usepackage{blindtext}
% Generate blind text
\usepackage[utf8]{inputenc}
\usepackage{array}
\usepackage{makecell}
\usepackage{tabularx}
\usepackage{titlesec}
\usepackage{bbding}
\usepackage[scr=euler,cal=dutchcal,frak=euler,bb=dsfontserif,bbsymbols]{mathalfa}
\usepackage{bbm}
\usepackage{bm}
%\usepackage{amsmath,amssymb,bbm}
\usepackage{graphicx}
\usepackage[x11names,table]{xcolor}
\usepackage[colorlinks=true, allcolors=DarkOrchid4]{hyperref}
\usepackage[shortlabels]{enumitem}   
\usepackage{array}
\usepackage{pdfpages}
%\usepackage{tipa} %perchè si stai zitto 
\usepackage{mleftright}
\usepackage{fancybox}
\usepackage{standalone}
\usepackage{microtype}
\usepackage{csquotes}
\usepackage{titlesec}
\usepackage{forest}
\usepackage{censor}
\usepackage{courier}
\usepackage{movement-arrows}
\usepackage{algorithm}
\usepackage{algpseudocode}
\usepackage[makeroom]{cancel}
\usepackage[many]{tcolorbox}
\tcbuselibrary{skins,raster,theorems}
\usepackage{mathtools}
\usepackage{soul}
\usepackage{multicol}
\usepackage{caption}
\usepackage{wrapfig}
\usepackage{authblk}
\usepackage{sectsty,titling}
\usepackage{float}
\usepackage{fontawesome5}
\usepackage{tikz}
\usetikzlibrary{positioning}
\usetikzlibrary{shapes.geometric,arrows,patterns,arrows.meta, bending,calc,matrix, shadings,intersections,shadows,decorations,fit,backgrounds,circuits.ee.IEC,shapes.callouts}
\newcounter{dummy}




\definecolor{dkgreen}{rgb}{0,0.6,0}
\definecolor{dred}{rgb}{0.545,0,0}
\definecolor{dblue}{rgb}{0,0,0.545}
\definecolor{lgrey}{rgb}{0.9,0.9,0.9}
\definecolor{gray}{rgb}{0.4,0.4,0.4}
\definecolor{darkblue}{rgb}{0.0,0.0,0.6}
\lstset{ 
	backgroundcolor=\color{black},  
	basicstyle=\footnotesize \ttfamily \color{white} \bfseries,   
	breakatwhitespace=false,       
	breaklines=true,   
	belowcaptionskip=1em,
	belowskip=-2\baselineskip, 
	numberstyle=\ttfamily,         
	captionpos=b,       
	numbers=left,framexleftmargin=5mm,            
	commentstyle=\color{dkgreen},   
	deletekeywords={...},          
	escapeinside={\%*}{*)},                  
	frame=single,                  
	keywordstyle=\color{purple},  
	morekeywords={BRIEFDescriptorConfig,string,TiXmlNode,DetectorDescriptorConfigContainer,istringstream,cerr,exit}, 
	identifierstyle=\color{white},
	stringstyle=\color{blue},      
	language=C,                                 
	numbersep=5pt,                  
	rulecolor=\color{white},        
	showspaces=false,               
	showstringspaces=false,        
	showtabs=false,                
	stepnumber=1,                   
	tabsize=5,                     
	title=\lstname,                 
}


\theoremstyle{remark}
\newtcbtheorem[number within=section]{rema}{Remark}{
	enhanced, breakable, sharp corners,
	colframe=white, coltitle=black, colbacktitle=gray!24,
	colback=white,
	interior style={top color=LightSteelBlue2, bottom color=white},
	boxed title style={interior style={left color=LightSteelBlue2,right color=white!35}, sharpish corners,
		frame style={right color=LightSteelBlue2,left color=white!35},drop fuzzy shadow=gray},
	attach boxed title to top left,
	drop fuzzy shadow=gray,
	boxed title style={boxrule=0.5mm, sharp corners},
	separator sign={:},
	description formatter=\newline,
	theorem hanging indent/.try=0pt,
}{}


\newcounter{defi}
\newenvironment{remark}[1][]{
	\ifstrempty{#1}{%   
		\begin{rema*}{}{}%                     
		}{%        
			\begin{rema*}{}{}%                 
			}%
		}{%
		\end{rema*}% 
	}
	
	
	\newtcbtheorem[number within=section]{nition}{Definition}{
		enhanced, 
		breakable, sharp corners,
		interior style={left color=SkyBlue4!60,right color=white!29},
		title style={left color=white,right color=white!35},
		% boxed title style={boxrule=5mm, colframe=black,sharp corners},
		colframe=white, coltitle=SkyBlue4!60!black, colbacktitle=blue!34,
		colback=white,
		titlerule=0mm,
		drop fuzzy shadow=gray,
		borderline={0.5mm}{0mm}{SkyBlue4},
		separator sign={:},
		description formatter=\newline,
		theorem hanging indent/.try=0pt,
	}{}
	
	\newenvironment{definition}[1][]{
		\stepcounter{defi}
		\ifstrempty{#1}{%   
			\begin{nition}{}{\thedefi\thedummy}%                     
			}{%        
				\begin{nition}{}{\thedefi eheh}%                 
				}%
			}{%
			\end{nition}% 
		}{\addvspace{\baselineskip}}
		
		\newcounter{theo}
		
		\newtcbtheorem[number within=section]{thrm}{Theorem}{
			enhanced, 
			breakable,
			sharp corners,
			interior style={left color=PaleTurquoise3!60,right color=Turquoise4!50},
			title style={right color=LemonChiffon1,left color=Aquamarine4!90!black},
			frame style={right color=SpringGreen4,left color=Turquoise4},
			colframe=white, coltitle=white, colbacktitle=SkyBlue4,
			colback=SkyBlue4,
			drop fuzzy shadow=Cyan4!50!black,
			titlerule=0mm,
			separator sign={:},
			description formatter=\newline,
			theorem hanging indent/.try=0pt,
		}{}
		
		\newenvironment{theorem}[1][]{ 
			\stepcounter{theo}
			\stepcounter{dummy}
			\ifstrempty{#1}{%   
				\begin{thrm}{}{\thetheo,\thedummy}%                     
				}{%        
					\begin{thrm}{}{\thetheo aa}%                 
					}%
				}{%
				\end{thrm}% 
			}
			
			\newtcolorbox{corollary}{
				enhanced, 
				breakable,
				sharp corners,
				interior style={left color=Thistle1,right color=LightSteelBlue4!50},
				title style={right color=LightSteelBlue4,left color=Thistle1},
				frame style={right color=LightSteelBlue4,left color=Thistle1},
				colframe=white, coltitle=LightSteelBlue4, colbacktitle=SkyBlue4,
				colback=SkyBlue4,
				drop fuzzy shadow=Cyan4!50!black,
				titlerule=0mm,
				separator sign={:},
				description formatter=\newline,
				title={Corollary}
			}{}
			
			%lemma lemma lemmaaa
			\newcounter{lem}
			
			\newtcbtheorem[number within=section]{lemm}{Lemma}{
				enhanced, 
				breakable,
				interior style={left color=SkyBlue4!20,right color=Cyan4!10},
				title style={left color=SkyBlue4!70,right color=Cyan4!10},
				frame style={left color=SkyBlue4!70,right color=Cyan4!10},
				colframe=white, coltitle=white, colbacktitle=SkyBlue4,
				colback=SkyBlue4,
				drop fuzzy shadow=gray,
				titlerule=0mm,
				separator sign={:},
				description formatter=\newline,
				theorem hanging indent/.try=0pt,
			}{}
			
			\newenvironment{lemma}[1][]{ 
				\stepcounter{lem}
				\stepcounter{dummy}
				\ifstrempty{#1}{%   
					\begin{lemm}{}{\thetheo,\thedummy}%                     
					}{%        
						\begin{lemm}{}{\thetheo aa}%                 
						}%
					}{%
					\end{lemm}% 
				}
				
				
				
				\newcounter{propos}
				%now making the proposition
				\newtcbtheorem[number within=section]{prp}{Proposition}{
					enhanced, 
					breakable,
					sharp corners,
					interior style={left color=RoyalBlue4!40!Aquamarine1,right color=RoyalBlue4!40},
					frame style={left color=RoyalBlue4!80!Aquamarine1,right color=RoyalBlue4},
					colframe=white, coltitle=RoyalBlue4, title style={left color=white!60, right color=RoyalBlue4!80},
					colback=Maroon1,
					drop fuzzy shadow=gray,
					titlerule=0mm,
					separator sign={:},
					description formatter=\newline,
					theorem hanging indent/.try=0pt,
				}{}
				
				\newenvironment{proposition}[1][]{ 
					\ifstrempty{#1}
					{%   
						\begin{prp}{}{\thepropos\thedummy eeeheheh\thedefi}%                     
						}{%        
							\hfill
							\begin{prp}{}{\thetheo,\thepropos ahah}%                 
							}%
						}{%
						\end{prp}% 
					}
					
					
					\newenvironment{diagram}[1][]{
						\ifstrempty{#1}
						{%   
							\begin{tikzpicture}[circuit ee IEC,x=2cm,y=0.8cm,semithick,
								every info/.style={font=\small},
								set resistor graphic=var resistor IEC graphic,
								set diode graphic=var diode IEC graphic,
								set make contact graphic= var make contact IEC graphic]
							}{%        
								\begin{tikzpicture}[circuit ee IEC,x=2cm,y=0.8cm,semithick,
									every info/.style={font=\small},
									set resistor graphic=var resistor IEC graphic,
									set diode graphic=var diode IEC graphic,
									set make contact graphic= var make contact IEC graphic]  
								}%
							}{
							\end{tikzpicture}
						}
						
						
						
						\newcounter{algor}
						%now making the proposition
						\newtcbtheorem{algo}{Complexity analysis}{
							enhanced, 
							breakable,
							interior style={left color=red!30,right color=white!29},
							title style={left color=red!30,right color=red!80},
							colframe=white, coltitle=black, colbacktitle=red!60,
							colback=red!80!black,
							titlerule=0mm,
							separator sign={:},
							description formatter=\newline,
							theorem hanging indent/.try=0pt,
						}{}
						
						\newenvironment{complexity}[1][]{
							{\par\noindent\ignorespaces}
							{\ignorespacesafterend} 
							\hfill   
							\stepcounter{algor}
							\stepcounter{dummy}
							\ifstrempty{#1}
							{%   
								\hfill
								\begin{algo}{}{\thealgor\thedummy eeeheheh\thedefi}%                     
								}{%        
									\hfill
									\begin{algo}{}{\thetheo,\thealgor ahah\thedummy}%                 
									}%
								}{%
								\end{algo}% 
							}
							
							
							\newcounter{exampl}
							
							\newtcbtheorem[number within=section]{exa}{Example}{
								enhanced jigsaw, 
								frame empty, breakable,
								colframe=white, coltitle=black,
								colback=white,
								opacityback=0,
								sharp corners,
								borderline west={1mm}{0mm}{CadetBlue4},
								attach boxed title to top left,  
								boxed title style={boxrule=0.5mm,drop fuzzy shadow=gray, colframe=CadetBlue4, sharp corners, interior style={left color=white, right color=CadetBlue4!50}},
								separator sign={:},
								opacityframe=.5,
								description formatter=\newline,
								theorem hanging indent/.try=0pt,
							}{}
							
							\newenvironment{example}[1][]{
								\stepcounter{exampl}
								\ifstrempty{#1}{%   
									\begin{exa}{}{\theexer\theexampl}%                     
									}{%       
										\begin{exa}{}{\theexampl}%                 
										}%
									}{%
									\end{exa}% 
								}
								
								\newcounter{exer}
								\newlength{\hatchspread}
								\newlength{\hatchthickness}
								\newlength{\hatchshift}
								\newcommand{\hatchcolor}{}
								% declaring the keys in tikz
								\tikzset{hatchspread/.code={\setlength{\hatchspread}{#1}},
									hatchthickness/.code={\setlength{\hatchthickness}{#1}},
									hatchshift/.code={\setlength{\hatchshift}{#1}},% must be >= 0
									hatchcolor/.code={\renewcommand{\hatchcolor}{#1}}}
								% setting the default values
								\tikzset{hatchspread=3pt,
									hatchthickness=0.4pt,
									hatchshift=0pt,% must be >= 0
									hatchcolor=black}
								% declaring the pattern
								\pgfdeclarepatternformonly[\hatchspread,\hatchthickness,\hatchshift,\hatchcolor]% variables
								{custom north east lines}% name
								{\pgfqpoint{\dimexpr-2\hatchthickness}{\dimexpr-2\hatchthickness}}% lower left corner
								{\pgfqpoint{\dimexpr\hatchspread+2\hatchthickness}{\dimexpr\hatchspread+2\hatchthickness}}% upper right corner
								{\pgfqpoint{\dimexpr\hatchspread}{\dimexpr\hatchspread}}% tile size
								{% shape description
									\pgfsetlinewidth{\hatchthickness}
									\pgfpathmoveto{\pgfqpoint{0pt}{\dimexpr\hatchspread+\hatchshift}}
									\pgfpathlineto{\pgfqpoint{\dimexpr\hatchspread+0.15pt+\hatchshift}{-0.15pt}}
									\ifdim \hatchshift > 0pt
									\pgfpathmoveto{\pgfqpoint{0pt}{\hatchshift}}
									\pgfpathlineto{\pgfqpoint{\dimexpr0.15pt+\hatchshift}{-0.15pt}}
									\fi
									\pgfsetstrokecolor{\hatchcolor}
									%    \pgfsetdash{{1pt}{1pt}}{0pt}% dashing cannot work correctly in all situation this way
									\pgfusepath{stroke}
								}
								\newcounter{exercise}[subsection]
								\newtcolorbox[use counter=exercise]{exercise}{
									enhanced standard jigsaw,
									sharp corners,
									breakable=false,
									interior titled code={
										\path[
										draw=white,
										sharp corners,
										pattern=custom north east lines,
										hatchspread=12pt,
										hatchthickness=4pt,
										hatchcolor=SkyBlue1!10,
										]
										(interior.south east) rectangle (interior.north west);
									},
									title={Exercise~\thetcbcounter},
									coltitle=black,
									fonttitle=\bfseries,
									colbacktitle=SkyBlue1!20}
								
								\newtcolorbox{revise}{
									enhanced jigsaw,
									colback=red!5!white,sharp corners, colframe=LightSteelBlue4!75!black,
									breakable,
									drop shadow,
									fonttitle=\bfseries,
									interior style={left color=CadetBlue3!30,right color=white!29},
									frame style={left color=DodgerBlue4,right color=DodgerBlue1},
									title={Revise with Kotatsu!}}
								\newtcolorbox{insult}{
									enhanced jigsaw,
									colback=red!5!white,sharp corners, colframe=red,
									breakable,
									drop shadow={red!80!black},
									fonttitle=\bfseries,
									interior style={left color=red!70,right color=white!29},
									title={You fucking prick}}
								
								\newtcolorbox{notation}{
									enhanced jigsaw,
									opacityback=0,
									colback=white, colframe=Aquamarine4!80!SteelBlue2,
									attach boxed title to top left ={yshift=-\tcboxedtitleheight/2,yshifttext=-\tcboxedtitleheight/2,xshift=0.3cm},
									boxed title style={%
										colback=Aquamarine4!80!SteelBlue2,
										opacityback=1
									},
									boxrule=0.5mm,top=0mm,bottom=0mm,
									rounded corners,
									coltitle=white,
									drop shadow,
									fonttitle=\bfseries,
									title={Note the notation!}}	
								
								
								
								\newtcolorbox{proofbox}{
									enhanced,
									attach boxed title to top left,
									colback=red!5!white,sharp corners, colframe=white,
									breakable,
									drop shadow,
									boxed title style={interior style={fill overzoom image=scrapped},sharp corners},
									coltitle=black,
									fonttitle=\bfseries\itshape,
									interior style={fill tile image*={width=1.1\linewidth}{squares}},
									frame style={white},
									title={Proof}}	
								
								\newenvironment{fancyproof}[1][]{
									\ifstrempty{#1}{%   
										\begin{proofbox} \begin{proof}[\unskip\nopunct]                   
											}{%        
												\begin{proofbox} \begin{proof}[\unskip\nopunct]%                 
													}%
												}{%
											\end{proof}\end{proofbox}% 
										}					
										\let\oldemptyset\emptyset
										\let\emptyset\varnothing
										
										
										\renewcommand{\labelitemi}{$\textcolor{SteelBlue4}{\bullet}$}
										\renewcommand{\labelitemii}{$\textcolor{blue}{\cdot}$}
										\renewcommand{\labelitemiii}{$\textcolor{SkyBlue4}{\diamond}$}
										\renewcommand{\labelitemiv}{$\textcolor{SkyBlue4}{\ast}$}
										
										\newcommand{\enf}[1]{\textcolor{DodgerBlue3}{\textbf{#1}}} %enf sta per enfasi
										\let\emph\enf
										\newcommand{\sott}[1]{\setulcolor{SkyBlue3}\ul{#1}}
										\let\underline\sott
										
										\newcommand{\prob}{\mathbb{P}}
										\newcommand\independent{\protect\mathpalette{\protect\independenT}{\perp}}
										\newcommand{\ev}{\mathbb{E}}
										\def\independenT#1#2{\mathrel{\rlap{$#1#2$}\mkern2mu{#1#2}}}
										\newcommand{\Z}{\mathbb{Z}}
										\newcommand{\R}{\mathbb{R}}
										\newcommand{\N}{\mathbb{N}}
										\newcommand{\T}{\mathbb{T}}
										\newcommand{\Tbar}{\overline{\T}}
										\newcommand{\A}{\mathscr{A}}
										\newcommand{\Nstar}{\N^{*}}
										\newcommand{\Rext}{\overline{\R}}
										\newcommand{\io}{\text{ i.o.}}
										\newcommand{\normale}{\mathcal{N}}
										\newcommand{\equalexpl}[1]{%
											\underset{\substack{\uparrow\\\mathrlap{\text{\vspace{-3cm}\hspace{-1em}#1}}}}{=}}
										\newcommand{\dif}{\mathop{}\!\mathrm{d}}
										\newcommand{\convas}{\xrightarrow[]{\text{a.s.}}}
										\newcommand{\convpr}{\xrightarrow[]{\pr}}
										\newcommand{\convd}{\xrightarrow[]{\text{d}}}
										\newcommand{\convlp}{\xrightarrow[]{\lp}}
										\newcommand{\lone}{L^{1}}
										\newcommand{\convw}{\xrightarrow[]{\mathrm{weak}}}
										\newcommand{\as}{\text{ a.s.}}
										\newcommand{\asstnr}{\sim N(0,1)}
										%\def\checkmark{\tikz\fill[scale=0.4](0,.35) -- (.25,0) -- (1,.7) -- (.25,.15) -- cycle;} 
										\newcommand{\dx}{\dif x}
										\newcommand{\dy}{\dif y}
										\newcommand{\dt}{\dif t}
										\newcommand{\du}{\dif u}
										\newcommand{\ds}{\dif s}
										\newcommand{\dw}{\dif\omega}
										\newcommand{\dz}{\dif z}
										\newcommand{\dmu}{\dif\mu}
										\newcommand{\dpr}{\dif\pr}
										\newcommand{\E}{\mathscr{E}}
										\newcommand{\B}{\mathscr{B}}
										\newcommand{\F}{\mathscr{F}}
										\newcommand{\G}{\mathscr{G}}
										\newcommand{\HS}{\mathscr{H}}
										\newcommand{\Zn}{\mathscr{Z}}
										\newcommand{\D}{\mathscr{D}}
										\newcommand{\xbar}{\overline{X}}
										\newcommand{\rbar}{\overline{\R}}
										\newcommand{\ybar}{\overline{Y}}
										\newcommand{\xxbar}{\overline{\xbar}}
										\newcommand{\xtilde}{\widetilde{X}}
										\newcommand{\ifonly}{\underline{if and only if}}
										
										\newcommand{\ubracketthin}[1]{\underbracket[0.3pt]{#1}}
										
										\newcommand*\circled[1]{\tikz[baseline=(char.base)]{
												\node[shape=circle,draw,
												shading=ball, ball color=SkyBlue1!70,
												,inner sep=1.5pt] (char) {\scriptsize\bfseries #1};}}
										\newcommand{\circnum}{\protect\circled{\arabic*}}
										\newcommand{\circlet}{\protect\circled{\alph*}}	
										\newcommand{\bbg}[1]{%
											\ooalign{$#1$\cr\raisebox{-.2pt}{$#1$}\cr\raisebox{.2pt}{$#1$}\cr\textcolor{white}{$\mkern0.2mu#1$}}%
										}
										\newcommand{\leb}{\bbg{\lambda}}
										
										%i comandi di MERDA di stefano
										\newcommand{\ra}{\rightarrow}
										\newcommand{\iy}{\infty}
										\newcommand{\mE}{\mathbb{E}}
										\newcommand{\mP}{\mathbb{P}}
										\newcommand{\mB}{\mathcal{B}(\mathbb{R})}
										\newcommand{\mL}{\mathbb{L}}
										\newcommand{\ninfty}{n\to$\infty$}
										\newcommand{\tinfty}{t\to$\infty$}
										\newcommand{\xinfty}{x\to$\infty$}
										\newcommand{\yinfty}{y\to$\infty$}
										\newcommand{\mLL}{\mathbb{L}^2(0,T)}
										\newcommand{\mR}{\mathbb{R}}
										\newcommand{\mC}{\mathcal{C}}
										\newcommand{\mRd}{\mathbb{R}^d}
										\newcommand{\mN}{\mathbb{N}}
										%\newcommand{\m1}{\mathbf{1}}
										\newcommand{\mF}{\mathcal{F}}
										\newcommand{\mM}{\mathcal{M}}
										\newcommand{\mW}{\mathcal{W}}
										\newcommand{\mV}{\mathcal{V}}
										\newcommand{\mA}{\mathcal{A}}
										\newcommand{\mez}{\frac{1}{2}}
										\newcommand{\intT}{\int_{0}^{T}}
										\newcommand{\nN}{n \in \mathbb{N}}
										\newcommand{\mmN}{\mathcal{N}}
										\newcommand{\BM}{(B_t)_{t\geq 0}}
										\newcommand{\PX}{(X_t)_{t\geq 0}}
										\newcommand{\var}{\mathbb{V}\!\mathrm{ar}\,}
										\newcommand{\cov}{\mathbb{C}\mathrm{ov}\,}
										\newcommand{\lp}{L^{p}}
										\newcommand{\BMF}{\left((B_t)_{t\geq 0},\mathcal{F}_t \right)}
										\newcommand{\mFt}{\mathcal{F}_t}
										\newcommand{\every}{\forall\,}
										\newcommand{\dyadic}{\mathcal{d}}
										\newcommand{\norm}[1]{\left|\negthinspace\left|#1\right|\negthinspace\right|}
										\newcommand{\pr}{\mathbb{P}}
										\newcommand{\indi}{\mathbb{1}}
										\newcommand{\trns}[1]{{#1}^{\scriptscriptstyle\mathsf{T}}}
										\newcommand{\iid}{\stackrel{\mathrm{iid}}{\sim}}
										\newcommand{\unmezz}{\frac{1}{2}}
										\newcommand{\sa}{$\sigma$-algebra}	\newcommand{\rv}{random variable}
										\titleformat{\section}{\color{SkyBlue4}}{\color{red}\thesection}{1em}{}
										\subsectionfont{\color{SkyBlue4!90!black}\bfseries}
										\pretitle{\begin{center}\LARGE\color{SkyBlue2}}
											\sectionfont{\color{Cyan2!50!black}\bfseries}
											\pretitle{\begin{center}\LARGE\color{SkyBlue2}}
												\newcommand{\prefacename}{Preface}
												\newcommand{\cinlar}{Çinlar }
												\newenvironment{preface}{
													\vspace*{\stretch{2}}
													{\noindent \bfseries \huge\color{SkyBlue4} \prefacename}
													\begin{center}
														% \phantomsection \addcontentsline{toc}{chapter}{\prefacename} % enable this if you want to put the preface in the table of contents
														\thispagestyle{plain}
													\end{center}%
												}
												{\vspace*{\stretch{5}}}
												
												
												\newenvironment{closethedeal}{
													\vspace*{\stretch{2}}
													{\noindent \bfseries \huge\color{SkyBlue4} That's all, folks}
													\begin{center}
														% \phantomsection \addcontentsline{toc}{chapter}{\prefacename} % enable this if you want to put the preface in the table of contents
														\thispagestyle{plain}
													\end{center}%
												}
												{\vspace*{\stretch{5}}}
												\pgfkeys{%
													/calloutquote/.cd,
													width/.code                   =  {\def\calloutquotewidth{#1}},
													position/.code                =  {\def\calloutquotepos{#1}}, 
													author/.code                  =  {\def\calloutquoteauthor{#1}},
													/calloutquote/.unknown/.code   =  {\let\searchname=\pgfkeyscurrentname
														\pgfkeysalso{\searchname/.try=#1,                                
															/tikz/\searchname/.retry=#1},\pgfkeysalso{\searchname/.try=#1,
															/pgf/\searchname/.retry=#1}}
												}  
												
												
												\newcommand\calloutquote[2][]{%
													\pgfkeys{/calloutquote/.cd,
														width               = 5cm,
														position            = {(0,-1)},
														author              = {}}
													\pgfqkeys{/calloutquote}{#1}                   
													\node [rectangle callout,draw,callout relative pointer={\calloutquotepos},text width=\calloutquotewidth,/calloutquote/.cd,
													#1] (tmpcall) at (0,0) {#2};
													\node at (tmpcall.pointer){\calloutquoteauthor};    
												}  		
												
												
												\newcommand\lemmethink[2][]{%
													\pgfkeys{/calloutquote/.cd,
														width               = 5cm,
														position            = {(0,-1)},
														author              = {}}
													\pgfqkeys{/calloutquote}{#1}                   
													\node [cloud callout,draw,callout relative pointer={\calloutquotepos},text width=\calloutquotewidth,/calloutquote/.cd,
													#1] (tmpcall) at (0,0) {#2};
													\node at (tmpcall.pointer){\calloutquoteauthor};    
												}  											
												
												
												\renewcommand\leq\leqslant
												\renewcommand\geq\geqslant			
													\posttitle{\par\end{center}\vskip 0.5em}

\setlength\parindent{0pt}

%%%%%%%%%%%%%%%%%%%%%%%%%%%%%%%%%%%%%%%%%%%%%%%%%%%%%%%%%%%%%%
\titleformat{\section}
{\color{UM_DarkBlue}\normalfont\large\bfseries}
{\color{blue!80!black}\thesection}{1em}{}

%%%%%%%%%%%%%%%%%%%%%%%%%%%%%%%%%%%%%%%%%%%%%%%%%%%%%%%%%%%%%%
\hypersetup{
    draft=false,
    final=true,
    colorlinks=true,
    citecolor=UM_DarkBlue,
    anchorcolor=yellow,
    linkcolor=UM_DarkBlue,
    urlcolor=UM_DarkBlue,
    filecolor=green,      
    pdfpagemode=FullScreen,
    bookmarksopen=false
    }
    
%%%%%%%%%%%%%%%%%%%%%%%%%%%%%%%%%%%%%%%%%%%%%%%%%%%%%%%%%%%%%%
\lstdefinestyle{Fortran}{
basicstyle=\scriptsize,        % the size of the fonts that are used for the code
  breakatwhitespace=false,         % sets if automatic breaks should only happen at whitespace
  breaklines=false,                 % sets automatic line breaking
  captionpos=b,                    % sets the caption-position to bottom
  commentstyle=\color{mygreen},    % comment style
  extendedchars=true,              % lets you use non-ASCII characters; for 8-bits encodings only, does not work with UTF-8
  keepspaces=true,                 % keeps spaces in text, useful for keeping indentation of code (possibly needs columns=flexible)
  keywordstyle=\color{blue},       % keyword style
  language=[95]Fortran,                 % the language of the code
  numbers=left,                    % where to put the line-numbers; possible values are (none, left, right)
  numbersep=5pt,                   % how far the line-numbers are from the code
  numberstyle=\tiny\color{mygray}, % the style that is used for the line-numbers
  rulecolor=\color{black},         % if not set, the frame-color may be changed on line-breaks within not-black text (e.g. comments (green here))
  showspaces=false,                % show spaces everywhere adding particular underscores; it overrides 'showstringspaces'
  showstringspaces=false,          % underline spaces within strings only
  showtabs=false,                  % show tabs within strings adding particular underscores
  stepnumber=1,                    % the step between two line-numbers. If it's 1, each line will be numbered
  stringstyle=\color{mymauve},     % string literal style
  tabsize=4,                       % sets default tabsize to 2 spaces
  title=\lstname                   % show the filename of files
}

%%%%%%%%%%%%%%%%%%%%%%%%%%%%%%%%%%%%%%%%%%%%%%%%%%%%%%%%%%%%%%%
\definecolor{UM_Brown}{HTML}{3D190D}
\definecolor{UM_DarkBlue}{HTML}{2264B0}
\definecolor{UM_LightBlue}{HTML}{1CA9E1}
\definecolor{UM_Orange}{HTML}{fEB415}




