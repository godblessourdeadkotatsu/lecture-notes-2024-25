% !TeX spellcheck = en_US
\documentclass[12pt]{report}
\usepackage{paccosi}
	\makeindex
\begin{document}
	\title{Statistical Inference}
	\author{Kotatsu}
	\date{\small Com'è possibile non abbia ancora dato sto esame?}
	\maketitle
	\pagenumbering{Roman}
	\begin{preface}
		These are the notes of the Statistical Inference course for the Academic Year 2025-2026 with Professor Favaro.\par
		I took these notes personally and I integrated the (many) unclear parts and passages using online resources and occasionally the fucking shitty books that this course has. \par
		I am not particularly enthusiast about this course but, as a person with no mathematical background from bachelor degree, I firmly believe in the necessity to understand what you are fucking doing and therefore I went out of my way to make these notes as understandable as possible for other wretched people that have not taken a single analysis exam in their whole life and now have to face integral equations. 
		
		GOD I HATE THIS UNIVERSITY
		\vskip1.2cm
		
		\hfill Kotatsu
		\vskip1.2cm
		Check the source code for this and other notes in my GitHub repo $\to$ \href{https://github.com/godblessourdeadkotatsu/lecture-notes-2024-25/tree/main/SI}{\faGithubSquare}
	\end{preface}
	\clearpage
	\tableofcontents
	\pagenumbering{arabic}
\chapter{Statistical Inference}
\section{Introduction}
When we talk about statistics we typically talk about data. Take $n\geq1$ numbers
\begin{equation*}
	(X_{1},\ldots,X_{n}).
\end{equation*}
We want to extract information from those objects. We can summarize them with plots, means, standard variation and so on: this is an application of \emph{descriptive statistics}. If we want to learn something about the whole population and not only these numbers, we must apply \emph{statistical inference}. We always assume that the sample is a subset of a population. \\
We know that:
\begin{enumerate}
	\item $(X_{1},\ldots,X_{n})$ is a realization of a random sample $(X_{1},\ldots,X_{n})$ which is a \rv;
	\item we assume some kind of symmetry: most common symmetries are i.i.d. properties, Markov Chains, autoregressive processes, exchangeability (whose i.i.d. is a specific case);
	\item we assume that the \rv s follow a certain distribution:
	\begin{equation*}
		X_{1},\ldots,X_{n}\sim F_{\theta}\qquad\theta\in\Theta
	\end{equation*}
	where $\theta$ is a parameter living in the parameter space $\Theta$.
\end{enumerate}
In this setting $\Theta$ is finite-dimensional. This assumption has the implicit consequence that the information about the population can be represented by a finite-dimensional vector of $\Theta$ and this is actually a pretty strong assumption that puts us in the realm of \emph{parametric statistics}. To learn about $\Theta$ we will use:
\begin{itemize}
	\item point estimation;
	\item testing;
	\item confidence intervals.
\end{itemize}
\section{Random samples}
\subsection{Random vectors}
We will use columns to denote vectors:
\begin{equation*}
	X=\begin{bmatrix}
		X_{1}\\
		\vdots\\
		X_{k}
	\end{bmatrix}.
\end{equation*}
This is a \rv{} on a $k$=dimensional space with
\begin{equation*}
	\ev{X}=\begin{bmatrix}\ev{X_{1}}\\\vdots\\\ev{X_{k}}\end{bmatrix}.
\end{equation*}
We also have a variance structure that use the covariance matrix:
\begin{equation*}
	\var(X)=\begin{bmatrix}
		\var(X_{1})&\cov(X_{1},X_{2})&\cdots&\cov(X_{1},X_{k})\\
		\cov(X_{2},X_{1})&\var(X_{2})&\cdots&\cov(X_{2},X_{k})\\
		\vdots&\vdots&\ddots&\vdots\\
		\cov(X_{k},X_{1})&\cdots&\cdots&\var(X_{k})
	\end{bmatrix}
\end{equation*}.
We can compute expectation and variance on a linear map of the \rv: take a non random matrix $n\times k$ and a $n\times 1$ vector $b$:
\begin{align*}
	\mathbf{A}&=(a_{ij})\\
	b&=(b_{1},\ldots,b_{n})^{\trsp}.
\end{align*}
Take a \rv{} $X=(X_{1},\ldots,X_{n})^{\trsp}$ with $\mu=\ev{X}$ and $\mathbf{V}=\var(X)$. Define 
\begin{equation*}
	Y=\mathbf{A}X+b
\end{equation*}
as a linear transformation of $X$ which has
\begin{align*}
	\ev{Y}&=\mathbf{A}\mu+b\\
	\var(Y)&=\mathbf{AVA}^{\trsp}.
\end{align*}
\begin{proposition}
	The variance matrix $\mathbf{V}$ of a \rv{} $X$ is positive semidefinite; it is positive definite if there exists no vector $b$ other than $b=0$ such that $b^{\trsp}X$ is a degenerate \rv.
	\end{proposition}
	\begin{proposition}
		If $\mathbf{V}=\var(X)$ is positive definite, there exists a square matrix $\mathbf{C}$ such that $Y=\mathbf{C}X$ has uncorrelated components with $\var(Y)=\mathbf{I}_{k}$ (no covariance).
	\end{proposition}
	\begin{proposition}
		If $\mathbf{A}=(a_{ij})$ is a $k\times k$ matrix then
		\begin{equation*}
			\ev{X^{\trsp}\mathbf{A}X}=\mu^{\trsp}\mathbf{A}\mu+\trace{\mathbf{AV}}.
		\end{equation*}
	\end{proposition}
We recognize $X^{\trsp}\mathbf{A}X$ as a \emph{quadratic form}.
\subsection{Multivariate Gaussian (dimension $k$)}
A \emph{standard multivariate Gaussian \rv{}} is just a collection of i.i.d $\mathsf{N}(0,1)$. Consider a vector 
\begin{equation*}
	Z=(Z_{1},\ldots,Z_{k})^{\trsp}\qquad \begin{array}{l}
		Z_{i}\distnorm{0,1}\quad i=1,\ldots,k\\
		Z_{i}\indep Z_{j}\quad\every i\neq j.
	\end{array}
\end{equation*} Remember that this has variance structure $\mathbf{I}_{k}$.
Define $Y=\mathbf{A}Z+\mu$ for some non singular $k\times k$ matrix $\mathbf{A}$ and a $k\times 1$ vector $\mu$. This is just a mapping $\R^{k}\to\R^{k}$ but we want to know the distribution of $Y$. We know the distribution of $Z$
\begin{equation*}
	f_{Z}(t)=\frac{1}{(2\pi)^{\frac{k}{2}}}\expg{-\unmezz t^{\trsp}t}\indi_{\R^{k}}(t).
\end{equation*}
The general rule to change the variable is
\begin{equation*}
	f_{Y}(y)=F_{Z}(t)\left|\frac{\partial Z}{\partial Y}\right|
\end{equation*}
The inverse transformation of $Y$ is
\begin{equation*}
	Z=\mathbf{A}^{-1}(Y-\mu).
\end{equation*}
Since $Z$ is linearly dependent from $Y$, the Jacobian matrix of partial derivatives is 
\begin{equation*}
	\left|\frac{\partial Z_{i}}{\partial Y_{j}}\right|=\left|\left(\mathbf{A}^{-1}\right)\right|=|\mathbf{A}|^{-1}.
\end{equation*}
We know that
\begin{align*}
	\var(Y)=\mathbf{V}&=\mathbf{A}\mathbf{V}_{Z}\mathbf{A}^{\trsp}\\
	&=\mathbf{A}\mathbf{I}_{k}\mathbf{A}^{\trsp}\\
	&=\mathbf{A}\mathbf{A}^{\trsp}\\
	\implies&|\mathbf{A}|^{-1}=|\mathbf{V}|^{-\unmezz}
\end{align*}
Now we have everything we need to compute the distribution of $Y$. We can write
\begin{equation*}
	t^{\trsp}t=\left\{\mathbf{A}^{-1}(Y-\mu)\right\}^{\trsp}\left\{\mathbf{A}^{-1}(Y-\mu)\right\}
\end{equation*}
But since the transposition of the inverse is the inverse of the transposition we get
\begin{align*}
	t^{\trsp}t&=\left\{\mathbf{A}^{-1}(Y-\mu)\right\}^{\trsp}\left\{\mathbf{A}^{-1}(Y-\mu)\right\}\\
	&=(Y-\mu)^{\trsp}\ubracketthin{\left(\mathbf{A}^{-1}\right)^{\trsp}\mathbf{A}^{-1}}_{\mathclap{\left(\mathbf{A}^{-1}\mathbf{A}\right)=\mathbf{V}^{-1}}}(Y-\mu)\\
	&=(Y-\mu)^{\trsp}\mathbf{V}^{-1}(Y-\mu).
\end{align*}
So substituting into the density of $Z$ we get
\begin{equation*}
	f_{Y}(y)=\frac{1}{(2\pi)^{\frac{k}{2}}|\mathbf{V}|^{\unmezz}}\expg{-\unmezz(y-\mu)^{\trsp}\mathbf{V}^{-1}(y-\mu)}\indi_{\R^{k}}(y).
\end{equation*}
Many of those things were not explained by Favaro, great job! Now we can write
\begin{align*}
	\mu&=\ev{Y}\\
	&=\ev{\mathbf{A}Z+\mu}\\
	&=\mathbf{A}\ubracketthin{\ev{Z}}_{0}+\mu\\
	&=\mu
\end{align*}
and
\begin{align*}
	\mathbf{V}&=\var(Y)\\
	&=\var(\mathbf{A}Z+\mu)\\
	&=\mathbf{A}^{\trsp}\mathbf{I}_{k}\mathbf{A}\\
	&=\mathbf{A}^{\trsp}\mathbf{A}=\mathbf{V}.
\end{align*}
\begin{exercise}
	Compute the distribution of $X=\mathbf{B}Y+b$ where $\mathbf{B}$ is a non singular $k\times k$ matrix and $b$ is a $k\times 1$ vector.
\end{exercise}
\subsection{Chi-squared distribution}
If $Z=(Z_{1},\ldots,Z_{k})^{\trsp}\distnormk{0,\mathbf{I}_{k}}$ set then 
\begin{equation*}
	U_{k}=Z^{\trsp}Z=\sum_{i=1}^{n}Z_{i}^{2}.
\end{equation*}
This is the centered chi-squared \rv:
\begin{equation*}
	U_{k}\sim\chi^{2}_{k}
\end{equation*}
which is often notated
\begin{equation*}
	W\sim c\chi^{2}_{k}\iff \frac{W}{c}\sim\chi^{2}_{k}.
\end{equation*}
For $k=1$ we have 
\begin{align*}
	\pr(U_{1}\leq t)&=\pr\left(Z^{2}_{i}\right)\\
	&=\pr\left(\sqrt{t}\leq Z_{1}\leq\sqrt{t}\right)\\
	&=2\Phi\left(\sqrt{t}\right)-1
\end{align*}
so
\begin{align*}
	f_{U}(t)&=\frac{\dif}{\dt}\left(2\Phi\left(\sqrt{t}\right)\right)\\
	&=\frac{1}{\sqrt{2\pi}}\ubracketthin{t^{-\unmezz}e^{-\frac{t}{2}}}_{\mathrlap{\text{$\Gamma$ distr. function}}}\indi_{\R^{+}}(t)\\
	&=\frac{\left(\unmezz\right)^{\unmezz}}{\Gamma\left(\unmezz\right)}t^{\unmezz-1}e^{-\frac{t}{2}}\indi_{\R^{+}}(t)\distgamma{\frac{1}{2},\unmezz}.
\end{align*}
\begin{revise}
	This is clear if we think about the Gamma distribution function:
\begin{equation*}
	f(t)=\frac{\beta^{\alpha}}{\Gamma(\alpha)}t^{\alpha-1}e^{-\beta t}\distgamma{\alpha,\beta}.
\end{equation*}
\end{revise}
Since the Gamma distribution is closed under convolution if the scale parameter (the second one) is the same, it is closed also under addition. If we take 
\begin{equation*}
	T_{i}\distgamma{\unmezz,\unmezz}\qquad\begin{array}{l}
		i=1,\ldots,k\\
		T_{i}\indep T_{j}
	\end{array}
\end{equation*}
then the sum is
\begin{equation*}
	U_{k}=\sum_{i=1}^{n}Z_{i}^{2}=\sum_{i=1}^{n}T_{i}\distgamma{\frac{k}{2},\unmezz}.
\end{equation*}
So the density of $U_{k}$ is
\begin{equation*}
	f_{U}(t)=\frac{\left(\unmezz\right)^{\frac{k}{2}}}{\Gamma\left(\frac{k}{2}\right)}t^{\frac{k}{2}}+e^{-\frac{t}{2}}\indi_{\R^{+}}(t).
\end{equation*}
We have
\begin{equation*}
	\ev{U_{k}}=k\qquad\var(U_{k})=2k.
\end{equation*}
Also, $\chi^{2}_{k}$ is closed under certain conditions:
\begin{equation*}
	\begin{array}{l}
		W_{r}\sim\chi^{2}_{r}\\
		U_{k}\sim\chi^{2}_k\\
		U_{k}\indep W_{r}
	\end{array}\implies W_{r}+U_{k}\sim\chi^{2}_{r+k}.
\end{equation*}
Fix a $n\geq1$ and take a collection of Gaussian i.i.d \rv s:
\begin{equation*}
	Y\distnorm{\mu,\sigma^{2}}\qquad\begin{array}{l}
		\mu\in\R\\
		\sigma^{2}\in\R^{+}.
	\end{array}
\end{equation*}
We define the \emph{sample mean} as
\begin{equation*}
	\Ybar_{n}=n^{-1}Y^{\trsp}\mathbf{1}_{n}=\frac{1}{n}\sum_{i=1}^{n}Y_{i}
\end{equation*}
And we define the \emph{corrected sample variance} as
\begin{align*}
	S^{2}_{n}&=(n-1)^{-1}\left(Y-\mathbf{1}_{n}\Ybar_{n}\right)^{\trsp}\left(Y-\mathbf{1}_{n}\Ybar_{n}\right)\\
	&=\frac{1}{n-1}\sum_{i=1}^{n}(Y_{i}-\Ybar_{n})^{2}.
\end{align*}
We also provided the matrix notation because it is useful in some situations. Consider the $n\times n$ matrix
\begin{equation*}
	\mathbf{A}=\begin{bmatrix}
		\frac{1}{\sqrt{n}}&\frac{1}{\sqrt{n}}&\frac{1}{\sqrt{n}}&\frac{1}{\sqrt{n}}&\ldots&\frac{1}{\sqrt{n}}\\
		\frac{1}{\sqrt{1\cdot2}}&-\frac{1}{\sqrt{1\cdot2}}&0&0&\ldots&0\\
		\frac{1}{\sqrt{2\cdot 3}}&\frac{1}{\sqrt{2\cdot 3}}&-\frac{2}{\sqrt{2\cdot 3}}&0&\ldots&0\\
		\vdots\\
		\frac{1}{\sqrt{n\cdot (n-1)}}&	\frac{1}{\sqrt{n\cdot (n-1)}}&	\frac{1}{\sqrt{n\cdot (n-1)}}&	\frac{1}{\sqrt{n\cdot (n-1)}}&\ldots&	-\frac{n-1}{\sqrt{n\cdot (n-1)}}
	\end{bmatrix}
\end{equation*}
This matrix is constructed in such a way that every rows sums to 0 except for the first column, that sums to $\sqrt{n}$. It's a orthonormal matrix where the first row is proportional to vector $\mathbf{1}$. Now we have
\begin{equation*}
	Z=\mathbf{A}Y\sim\mathsf{N}_{n}\left(\mu_{Z},\sigma^{2}\mathbf{I}_{k}\right)
\end{equation*}
where
\begin{equation*}
	\mu_{Z}=\left(\mu\sqrt{n},0,0\ldots,0\right)^{\trsp}
\end{equation*}
We know that
\begin{align*}
	\ubracketthin{Z_{2}^{2}+\ldots+Z_{n}^{2}}_{n-1}&=Z^{\trsp}Z-Z_{1}^{2}\\
	&=\sum_{j=1}^{1}Y^{2}_{j}-n\Ybar_{n}^{2}\\
	&=(n-1)S^{2}_{n}
\end{align*}
so we proved that the sample \rv{} $Z_{2}^{2}+\ldots+Z^{2}_{n}$ is equal to the scaled sample variance. This means that
\begin{equation*}
	\frac{n-1}{\sigma}S^{2}_{n}\sim\chi^{2}_{n-1}
\end{equation*}
since it is a sum of $n-1$ Gaussian variables.
Why are the sample mean and the variance independent? We have that the sample mean is
\begin{equation*}
	\Ybar_{n}\distnorm{\mu,\frac{\sigma^{2}}{n}}
\end{equation*}
and therefore is independent from sample variance.
\end{document}
%THIS IS THE DARK AGE OF LOVE   $\cdot$