% Latex template for submission to the XXth International Conference on the use of Computers in Radiation therapy
% (ICCR 2024)
%
% Date:   Oct 2023
%
% Modified from
% https://github.com/gschramm/fully3d_templates
% 
% To build this document, we recommend to use latexmk via:
% ```latexmk -pdf iccr2024_template.tex```
% Building in the online editor overleaf also works.

\documentclass[11pt]{article}
\usepackage{iccr2024}

%%%%%% add your extra packages here (if needed)                                        %%%%%

%%%%%% before, have a look which packages are already imported by the iccr2024 package %%%%%
%\usepackage{mypackage}
%%%%********************************************************************


%%%%% add your bibtex file that contains the bibtex entries here %%%%%
%%%%% please include DOIs in the bibtex entries if possible      %%%%%
\addbibresource{references.bib}

\begin{document}
\twocolumn[{

%-------------------------------------------------------------------------------------------
%%%%% add your title here %%%%%
\title{Anthony Braxton: vibrazioni e de-costruzioni} 

%%%%% add authors and affiliations here %%%%%
\author[1]{Lorenzo~Sala}

%%%%% don't change these 2 lines %%%%%
\maketitle
\thispagestyle{fancy}



%-------------------------------------------------------------------------------------------
%%%%% add your summary (abstract) here               %%%%%%
%%%%% use footnotesize for this section              %%%%%%
%%%%% please stick to the customabstract environment %%%%%% 

\tableofcontents
\bigskip
\medskip}]

%-------------------------------------------------------------------------------------------
%%%%% main text                                                %%%%%    
%%%%% remove the dummy content and put your own content here   %%%%% 
%%%%% feel free to choose your own section titles              %%%%% 
%%%%% you don't need to put the content in a separate tex file %%%%%

% dummy_content.tex shows how to add sections, figures, tables, formulas, and references
% remove the following line, it just adds dummy content

\input{twocol2}


%-------------------------------------------------------------------------------------------
\printbibliography[title={Bibliografia}]

\end{document}
